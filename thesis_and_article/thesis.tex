% https://en.wikibooks.org/wiki/LaTeX/Document_Structure
% \documentclass[twoside,openright]{report}
\documentclass{sapthesis}

\usepackage{graphics}

\title{Demystifying Back-Propagation}
\subtitle{Matrix calculus for the working computer scientist}
\author{Diego Bellani}
\IDnumber{1838645}
\course{Computer Science}
\courseorganizer{Facoltà di Ingegneria dell'informazione, informatica e statistica}
\AcademicYear{2023/2024}
\advisor{Prof. Massini}
\authoremail{bellani.1838645@studenti.uniroma1.it}
\copyyear{2024}
%\date{2024}
\examdate{24 May 2024}
\examiner{Casalicchio}
\examiner{Gorla}
\examiner{Massini}
\examiner{Piperno}
\examiner{Samory}
\examiner{Tronci}
\examiner{Zuliani}

\newcommand{\TODO}[1]{TODO: #1}
% https://tex.stackexchange.com/a/284054
\newcommand{\argmin}{\mathop{\mathrm{arg\,min}}}
\newcommand{\trace}{\mathrm{tr}}
\newcommand{\expected}{\mathop\mathsf E}
\newcommand{\partialfrac}[2]{\frac{\partial #1}{\partial #2}}
\newcommand{\Diag}{\mathrm{Diag}} % diagonalize vector to matrix
\newcommand{\diag}{\mathbf{diag}} % diagonalize matrix to vector
\newcommand{\vect}{\mathbf{vec}} % vectorize
\newcommand{\krone}{\otimes} % Kronecker product
\newcommand{\hadam}{\odot} % Hadamard product
% https://tex.stackexchange.com/a/163041
\newcommand{\dif}{\mathop{}\!\mathsf d} % differential
\newcommand{\der}{\mathop{}\!\mathsf D} % derivative
\newcommand{\pluseq}{\mathrel{{+}{=}}}
\renewcommand\vec[1]{\mathbf{#1}}
% https://ctan.org/pkg/leftidx
\newcommand\leftidx[3]{{\vphantom{#2}}#1#2#3}
\newcommand\cmat[3]{% constant scalar matrix
    \leftidx{_{#2}}{\vec{#1}}{}%
    \,% small space to aid readability
    \leftidx{_{#3}}{\vec{#1}}{'}%
}
\newcommand\matr{\mathrm{Mat}}

% https://tex.stackexchange.com/questions/133147/how-to-use-pdfcolorstack
% 0 is the main color stack
\chardef\Color=0
\newcommand{\comment}[1]{\pdfcolorstack \Color push {0 1 0 rg 0 1 0 RG} #1 \pdfcolorstack \Color pop {}}
%\newcommand{\article}[1]{\pdfcolorstack \Color push {0 0 1 rg 0 0 1 RG} #1 \pdfcolorstack \Color pop {}}

% https://tex.stackexchange.com/a/638348
% http://latexref.xyz/_005cDeclareGraphicsRule.html
\DeclareGraphicsRule{*}{mps}{*}{}

% \setcounter{chapter}{-1}

\iffalse
Quando jitto levo le branch per seguire propagare il gradiente possono essere
eliminate.

Fondere i cicli è lo stesso che fondere i kernel? I miei nuclei sono per la
maggior parte cicli perfettamente nidificati. Quindi se ho lo stesso spazio di
iterazione la fusione dovrebbe essere banale.
\fi

% Useful Wikipedia pages:
%   * https://en.wikipedia.org/wiki/Vectorization_(mathematics)
%   * https://en.wikipedia.org/wiki/Frobenius_inner_product (I know how to put the needed citation)
%   * https://en.wikipedia.org/wiki/Adjoint

\begin{document}

\maketitle

\begin{abstract}
% The setting
In deep learning libraries, the back-propagation algorithm is typically
implemented to automatically calculate gradients, relieving users of the burden
of understanding the underlying mechanics. However, when implementing a
optimized GPU kernel for a custom operation, understanding how gradients are
propagated becomes necessary.

% The problem
To propagate gradients through batched (or matrix) operations one needs to grasp
enough automatic differentiation and matrix calculus, which can be challenging
for various reasons. Matrix results in the automatic differentiation literature
usually are roughly explained and rely on advanced mathematical concepts. The
matrix calculus literature typically is not concerned with derivations for
back-propagation. Finally, the lack of a standard definition for matrix
derivative makes it hard to translate between different resources.

% Our solution/contribution
In this work, we present the back-propagation algorithm in matrix form together
with a simple matrix calculus, providing the readers with the necessary tools to
derive gradient propagation formulas. Our work constitutes the first short and
self-contained explanation of both of these topics, addressing a gap in existing
literature.
\end{abstract}

\tableofcontents

\chapter{Prelude}

We introduce the general context in which this work is located in informal terms to
justify the various topics.

\section{Machine Learning and Deep Learning}

Machine learning is a subset of artificial intelligence (AI) in which
statistical models and training algorithms are developed, such that, given a set
of data, the training algorithm can specialize the statistical model to solve a
certain task\footnote{If we could specialize a certain model to solve ``any
task'' we would call this artificial general intelligence. Quotations are
necessary around ``any task'' since algorithms can't solve literally any
problem(e.g. the halting problem).}.

The training algorithm iteratively \emph{teaches} the model at solving the task,
each iteration should improve the ability of the model to perform such task,
this iterative improvement can be thought of as a form of \emph{learning}. Hence
the name of this field of study.

This is a fundamental shift of perspective from the Good Old-Fashioned AI
(i.e. the one that is based on symbolical logic) were so called ``expert systems''
relied on a hand cured series of facts and inference rules that allowed the
system to reason about a certain problem in the hope of solving it.

The simplest example of a task commonly performed with machine learning is the
one of image classification. In this task you have a set of labels for pictures
(e.g. animal, car, book, etc\dots) and given an input image you have to assign
the correct label to it.

Machine learning in some form or another exits from a lot of time (just think of
inferential statistics). The reason it only recently became a viable method, and
ultimately the most effective one in many fields, is due to the abundance of
easily available data in digital form (in the order of gibibyte and tebibyte)
and the huge computing power that the development of semi-conductor design and
manufacturing has brought us (again in the order of tera-floating-point
operations per second).

Machine learning has many sub-fields itself, but the one that contributed the
most to its success and the end of the AI winter\footnote{This is a term used in
the history of AI to identify periods in which there is a reduction in the
funding and the interest in AI research. This used to happen after AI research
failed to meet the inflated expectations that it set multiple times.} (and the
start of the AI spring) is the so called deep learning. This sub-field concerns
itself with a specific kind of model: deep artificial neural networks. These
models are composed of layers and are deep in the sense that they have many of
them. Having many layers allows these models to perform their tasks with minimal
pre-processing of the data and no manual feature extraction. This is opposed to
more classical machine learning models which expect input of data as a vector of
features.

% TODO: neural networks may deserve a little introduction as DAGs (feed-forward)
% and connected graphs (recurrent). Of importance to us is the organization in
% layers and the use of linear algebra in these layers.

If we think about the image classification example it is already non obvious
which feature to extract from the images to aid a classification algorithm
perform its task. With the deep learning approach we provide the images to the
algorithm, almost, as is and the various layers of the model learn to perform
the various feature extraction sub-tasks that are ultimately used by the
network itself to perform the classification.

The family of training algorithms that made much of this progress, with deep
neural networks models, possible are the ones based on mini-batch stochastic
gradient descent (together with the back-propagation algorithm\footnote{The main
topic of this work.} to efficiently calculate the gradient). The main idea of
this family of algorithms is to teach the model by correcting it based on the
errors that it makes. This error is quantified with a loss function which the
algorithm tries to minimize, during its execution, with a stochastic
approximation of the gradient of the loss function.

The various kinds of deep learning models are often called \emph{architectures}.
These architectures have characteristics that allow them to better solve certain
kinds of tasks. For example the convolutional neural networks are particularly
well suited to solve image related tasks (like our classification example).

\subsection{Brief history of the recent notable results}
\label{sec:notable_results}

The field of deep learning has accomplished many notable results in recent
years. We have selected the events that in our opinion give the best idea of
what these modern systems are capable of. Most of these things were unheard of
(or even unthinkable) just a few years ago and the pace at which we are seeing
this progress does not seem to stop (or slow down for that matter).

The progresses are not limited to solving or advancing old problems but we now
have a totally new kind of computer programs such as ones that can generate
images from prompts, transfer an artist's style to a photo, chat programs that
can perform question answering tasks in a way that previous search engines never
could.

The first big accomplishment for deep learning was 2012 when AlexNet (a
convolutional neural network) won the ImageNet Large Scale Visual Recognition
Challenge (ILSVRC2012) with a significant margin. To give an idea of how hard
this task is the number of classes in this dataset is more than 10,000.

In 2016 DeepMind with AlphaGo, which is a program that uses deep learning to
guide its Monte Carlo tree search with a ``value network'' and a ``policy
network'', was the first computer program to defeat the Go world champion.
While defeating humans at chess could be done with GOFAI techniques Go is a much
harder problem to solve. If we consider the number of possible board
states as a proxy for the
% https://math.stackexchange.com/a/1407631
hardness of this games chess has roughly $10^{45}$ possible board states while
% https://en.wikipedia.org/wiki/Go_(game)
Go has $10^{170}$. For reference the number of atoms in the observable universe
is estimated to be in the order of $10^{80}$.

In 2020 DeepMind with AlphaFold 2 archived astounding results in the Critical
Assessment of Structure Prediction (CASP14) competition protein folding
prediction problem.

In 2021 OpenAI released DALL-E a model that allowed to enter a prompt and
generate an image based on that prompt.

In 2022 OpenAI released the famous ChatGPT chatbot, the model on which this
program is based on a paper released just a few years prior (2017).

In 2024 (the year of the writing of this work) OpenAI announced Sora, a model
able to generate video from a prompt.

Many more examples could be given about the impressing capabilities of modern
deep learning systems, that are left out for the sake of brevity. The ones
reported were not necessarily the first to do said things and there have of
course been various improvements on the cited systems. We have decided to select
the ones that we felt were the most famous and representative ones (especially
at the time of their release or announcement). It is also notable how little
time passes from the release of a research paper (Attention is All You Need) to
the commercial implementation of its ideas (ChatGPT).

Another interesting fact is that while most of the cited systems are closed
source the open source community is (with a bit of lag) keeping up with
research and commercial systems. This we feel is very important for the
democratization of these technologies. Almost each of the cited technologies has
an open source counter, Autopilot has OpenPilot, GPT-3 has LLAMA, ChatGPT has
OpenAssistant and DALL-E has StableDiffusion.

\section{Hardware Accelerators}

Computers are equipped with a CPU that is capable of performing all kinds of
calculations by itself. By its general purpose nature the CPU cannot excel at
all tasks. Some of these tasks are performed often enough which makes it worth
implementing specialized hardware to offload these calculations so that
they can be performed faster. These specialized hardware implementations are
called hardware accelerators.

Machine learning algorithms require lots of data to be properly trained and the
bigger the dataset used to train them is, the better they perform at the given
task. Therefore training these algorithms requires processing a lot of data and
the later task to be done in a reasonable amount of data requires a lot of
computing power.

Moreover, most machine learning algorithms and all deep learning ones have at
their core linear algebra operations (among which the most important one is
\emph{matrix multiplication}), which have a big compute requirement to data
ratio. In particular linear algebra requires floating-point point operations.

These two facts combined, big amounts of data and floating-point intensive
operations, require the use of specialized hardware to accelerate the training
time so that are completed in a reasonable amount of time.

Training machine learning algorithms therefore requires hardware accelerators
that maximize the throughput. The first, most common and successful hardware
accelerators for machine learning are, without a doubt, Graphical Processing
Units (GPUs).

The reasons for the great use of GPUs in machine learning are going to be
explained in section~\ref{sec:gpus}. They are by no means the only hardware
accelerators used (especially in deep learning), there is, in fact, a lot of
research for ``better hardware accelerators'' that use ideas different from the
ones in GPUs. One of the most notable examples is certainly Google with its
Tensor Processing Units (TPU). These hardware accelerators are even more
specialized to perform matrix multiplication.

But GPUs are also improving at machine learning-related tasks (despite their name),
and they are definitely here to stay, especially for consumer hardware.

\subsection{GPUs}\label{sec:gpus}

% https://en.wikipedia.org/wiki/Graphics_processing_unit

GPUs, as their name may suggest, come from the gaming world. Initially, they
were just used to composite data together and generate a video signal to spare
RAM and CPU cycles. For early consumers, hardware RAM was at a premium and
dedicating a portion of it to represent every pixel on a screen\footnote{CRT
displays of early computers with graphic capabilities did not really have
pixels, but let us just imagine they had them for the sake of simplicity.} would
not have left any memory for the program itself. Skipping a lot of history and
simplifying a lot, to solve this issue, the screen was divided into equal
rectangular regions called \emph{tiles} and the graphic of games was decomposed
in this form.

A simple example of how this helped to spare memory was for drawing backgrounds.
You could have specified a single tile with the color of the sky and a cloud
pattern that then could have been repeated, by the hardware, across the entire
screen. Another simple usage could be flipping or changing the color of the
sprite of a character to signify its death.

Over time GPUs started to implement more and more fixed functions for 2D and 3D
graphics and got the ability to display more granular primitives like polygon
(represented as a list of points.) The great variety  of this kind of device in
commercial computers together with the complexity of sending them instructions,
justified the development of a standardized way to do so. These are generally
called Application Programming Interfaces (APIs) and OpenGL, before
% https://www.khronos.org/opengl/wiki/History_of_OpenGL#OpenGL_1.5_(2003)
version 2.0, was a common API for this purpose.

In 2001 Nvidia released the first GPU with programmable shading. This GPU
allowed for small programs (without looping capabilities) to be executed for
each pixel in order to decide its color. The rapid advancement in technology
quickly allowed to run programs with looping capabilities on GPUs, so a few
researchers started to gain advantage of the computing capabilities of these
devices for general-purpose calculations and not only for coloring pixels. This
was the start of the General Purpose GPU (GPGPU) programming.

Doing GPGPU with shader programs was contrived, therefore in 2007 Nvidia
released CUDA, an API dedicated to GPGPU, which allowed to run so-called compute
kernels on GPU, which are fully fledged programs capable of performing many kinds
of calculations. This started a literal revolution in the numerical software
world, so much so that GPUs have started to add functionality that are not
needed by graphics but help with general purpose compute capability.

With all of these developments GPUs have become invaluable accelerators for tasks
involving floating-point calculations\footnote{We want to emphasize this point
because CPUs are still better in many tasks.}.

% TODO: a few figures would help this sections

\section{Related Work}

Modern deep learning architectures involve loss functions of matrix argument,
this implies the use of matrix algebra for calculating the loss and the use of
matrix calculus for the gradient estimation. For learning matrix algebra there
are many resources~\cite{strang2019,searle2017,magnus2005}, sadly the same
cannot be said for matrix calculus resources focused on  deep learning\dots
Which is exactly the gap in the literature that this work tries to cover.

The problems that plague many resources are stopping at the vector case or
treating the matrix case too vaguely or too
complexly~\cite{giering1998,buithanh2023} or just giving results with little to
no explanation~\cite{giles2008}.

Therefore we are left either with resources focused on matrix
calculus~\cite{magnus2019,turkington2001}, but with no regard for
back-propagation or resources focused on automatic differentiation (of which
back-propagation is a specific case) that do not talk about the matrix
case~\cite{griewank2008, naumann2011}.

\part{Theory}

\iffalse
\chapter{Contribution}

We provide the first self contained explanation of the back-propagation
algorithm for scalar functions containing matrix expressions and the matrix
calculus necessary to derive the formulas needed for an efficient implementation
of the gradient accumulation. To keep the explanation short and focused on the
deep learning use case we avoid introducing unnecessary concepts from the
automatic differentiation literature (i.e. forward-mode gradient accumulation.)
Since we present back-propagation together with matrix calculus we always use a
layout convention for derivatives and focus our description on the
back-propagation use-case. The matrix calculus literature has not reached a
consensus on which layout to use, hence making it difficult to translate results
between resources. We also include a discussion on broadcasting which is not
usually discussed in either automatic differentiation or matrix calculus
literature but it is commonly used in deep learning.
\fi

\chapter{Mathematical Preliminaries}

Here we introduce all the notation and conventions that we are going to use for
the rest of this work, this chapter can be skimmed or skipped since it is meant
just to avoid any confusion the reader can just come back to clarify when
something is not clear in the rest of this work.

\section{Notation}

First of all our functions and variables are going to be defined in terms of
real numbers.

Lowercase Latin and Greek letters are used for scalars e.g. \(a,\) \(b,\)
\(c,\) etc\dots This allows us to use this notation consistently even for common
constant like \(e = 2.71\ldots\) Boldface Latin and Greek letters are used for
vectors, which are always considered as column matrices, e.g. \(\vec a,\) \(\vec
b,\) \(\vec c,\) etc\dots Finally for matrices we are going to use capital Latin
and Greek letters e.g. \(A,\) \(B,\) \(C,\) etc\dots The same convention is used
for function of a certain value or argument, of which all possible combinations
are shown in table~\ref{tab:valXarg2}.

\begin{table}
\centering
% https://latexref.xyz/_005cvline.html
\begin{tabular}{rccc}
    & \(x\)     & \(\vec x\)     & \(X\) \\[2pt]
    \cline{2-4}
    \multicolumn{1}{r|}{\(y =\)} & \(f(x)\) & \(f(\vec x)\) & \(f(X)\) \\
    \multicolumn{1}{r|}{\(\vec y =\)} & \(\vec f(x)\) & \(\vec f(\vec x)\) & \(\vec f(X)\) \\
    \multicolumn{1}{r|}{\(Y =\)} & \(F(x)\) & \(F(\vec x)\) & \(F(X)\)
\end{tabular}
\caption{All possible combinations for function argument and value.}
\label{tab:valXarg2}
\end{table}

When we have a sequence of scalar, vectors or matrices we are going to index it
with an upper script number in square brackets, this is to avoid confusion with
the exponentiation operator, e.g. \(a^{[1]},\ldots,a^{[n]}\) is a sequence of
\(n\) scalars.

When we specify the dimensions of a matrix we are going to say that it is \(n\)
by \(m\), written \(n \times m,\) meaning that it has \(n\) rows and \(m\)
columns.  When we say that a vector has size \(n\) we mean that it is a matrix
of size \(n \times 1\), this is handy also because the first subscript always
indicates a row in both matrices and vectors.

When we want to talk about a scalar in a vector we are going to use a scalar
subscript, e.g. if \(\vec a\) has size \(n\) then \[\vec a =
\left[\begin{array}{c} \vec a_1 \\ \vdots \\ \vec a_n \end{array}\right].\]

For matrices we are going to use a pair of scalar subscripts, separated by a
coma, one of the scalar subscripts can be substituted by a colon to indicate an
entire row or column,\footnote{Trivially \(A = A_{:,:},\) \(\vec a_{i,1} = \vec
a_i,\) for every \(i\) and \(\vec a = \vec a_:.\)} e.g.  if \(A\) is \(n \times
m\) then \[
    A
    = \left[\begin{array}{c}A_{1,:} \\ \vdots \\ A_{n,:}\end{array}\right]
    = [\begin{array}{ccc}A_{:,1} & \cdots & A_{:,m}\end{array}]
    = \left[\begin{array}{ccc}
        A_{1,1} & \cdots & A_{1,m} \\
        \vdots & \ddots & \vdots \\
        A_{n,1} & \cdots & A_{n,m}
    \end{array}\right].
\]

% For ease of typesetting i.e. avoiding \left( and \right) and to avoid any
% confusion with exponentiation.
The transpose of a matrix \(A\) is going to be denoted with a prime superscript
\(A'\).

% https://en.wikipedia.org/wiki/Notation_for_differentiation
% % https://en.m.wikipedia.org/wiki/Matrix_calculus#Layout_conventions
When we are going to talk about derivatives we are going to use \(\der\) for the
total derivative, \(\partialfrac\relax\relax\) for the partial derivative (which
are going to be laid out using the numerator convention) and \(\dif\) for the
differential. We are going to elaborate further on this part in
section~\ref{sec:matcalc}.

\section{Common Functions and Constants}

Together with the definition of some operations and some useful constants, we
are also going to state some useful properties of said operations (without
proving them) that are going to greatly simplify our derivations later.

The juxtaposition of two matrices is used for the classic matrix multiplication.
We also use the + symbol to add two matrices element-wise as long as they have
the same shape.

The standard basis vector for a vector space of size \(n\) are going to be
denoted as \(\vec e^{[1]},\ldots,\vec e^{[n]}.\) \(\vec e^{[1]}\) has a 1 in
position \(i\) and 0 everywhere else.

We are going to denote the ``one vector'' as \(\vec 1 = \sum_{i=1}^n e^{[i]},\)
the ``zero vector'' as \(\vec 0\), the vector with all zeros, the identity
matrix \(I = [\begin{array}{ccc}e^{[1]} \cdots e^{[n]}\end{array}]\).

% The size of I could be done with a normal subscript, since we always index in
% matrix with two ``symbols'', therefore there is no ambiguity, but for consistency
% we don't.
If there is any ambiguity with the size of one of the above mentioned vectors or
matrix, we are going to use a left subscript, e.g. if a vector space has size
\(n\) then \(\leftidx{_n}{\vec e}{^{[i]}}\) is the \(i\)th\footnote{Trivially
\(\leftidx{_n^{\vphantom{^{[i]}}}}{\vec e}{^{[i]}_i} = 1\) for all \(i.\)}
standard basis vector of size \(n\) and \(\leftidx{_n}I{}\) has size \(n \times
n.\) 

Thanks to the matrix product we can specify a matrix of size \(n \times m\) of
all zeros, or ones as \(\cmat 0 n m\) and \(\cmat 1 n m.\)

For the rest of this section let \(A\) be \(n \times m\) and \(S\) be \(n \times
n.\)


The scalar-matrix product between \(a\) and A is denoted as \(a \cdot A\) and \[
a \cdot A = \left[\begin{array}{ccc}
    a A_{1,1} & \cdots & a A_{1,m} \\
    \vdots & \ddots & \vdots \\
    a A_{n,1} & \cdots & a A_{n,m}
\end{array}\right].
\] We will also write \(A \cdot a\) to mean the same thing.

If \(B\) has the same size as \(A\), then the Hadamard product between them, is
just their element-wise multiplication. It is denoted as \(A \hadam B\) and \[
A \hadam B = \left[\begin{array}{ccc}
    A_{1,1} B_{1,1} & \cdots & A_{1,m} B_{1,m} \\
    \vdots & \ddots & \vdots \\
    A_{n,1} B_{n,1} & \cdots & A_{n,m} B_{n,m}
\end{array}\right].\]

The matrix-scalar power raises each element of a matrix to a certain \(a\), the
scalar-matrix power instead makes it so that every element of a matrix is used as
the exponent of a certain base \(a\). We denote them as \(A^{{\cdot}a}\) and
\(a^{{\cdot}A}.\) Therefore \[
    A^{{\cdot}a} =
    \left[\begin{array}{ccc}
        A_{1,1}^a & \cdots & A_{1,m}^a \\
        \vdots & \ddots & \vdots \\
        A_{n,1}^a & \cdots & A_{n,m}^a
    \end{array}\right],
    a^{{\cdot}A} =
    \left[\begin{array}{ccc}
        a^{A_{1,1}} & \cdots & a^{A_{1,m}} \\
        \vdots & \ddots & \vdots \\
        a^{A_{n,1}} & \cdots & a^{A_{n,m}}
    \end{array}\right].
\]

We can now define the softmax of the columns of a matrix as \[S(X) =
e^{{\cdot}X} \hadam \left(\vec 1 \vec 1' e^{{\cdot}X}\right)^{{\cdot}-1}.\] In
the case of a vector we are going to write \(\vec s(\vec x) = e^{{\cdot}\vec x}
\cdot (\vec 1' e^{{\cdot}\vec x})^{{\cdot}-1}.\)

The Kronecker product is a sort of generalization of the matrix scalar product
but between every element of a matrix w.r.t. another one. If B has any size we
denote it as \(A \krone B\) and \[
    A \krone B = \left[\begin{array}{ccc}
        A_{1,1} \cdot B & \cdots & A_{1,m} \cdot B \\
        \vdots & \ddots & \vdots \\
        A_{n,1} \cdot B & \cdots & A_{n,m} \cdot B
    \end{array}\right].\]
Note that if \(A\) has size \(1 \times 1\) and \(A = a\) for some \(a\) than \(a
\krone A = a \cdot A.\)

Vectorizing a matrix means stacking its \(m\) columns on top of each other to
form another vector. Matricising a vector means taking its \(nm\) rows, \(n\) at
the time and concatenating them, from left to right, to form an \(n \times m\)
matrix. We denote these two operations as \(\vect(A) = \vec a\) and
\(\matr(\vec a) = A\) respectively and we define them formally as
\begin{eqnarray*}
    \vect(A) &=& \sum_{i=1}^{m} \vec e^{[i]} \krone A \vec e^{[i]}, \\
    % TODO: mettere un commento con l'URL a stackoverflow.
    \matr(\vec a) &=&
        (\vect(\leftidx{_n}I{})' \krone \leftidx{_m}I{})
        (\leftidx{_n}I{} \krone \vec a).
\end{eqnarray*}
It is interesting to note that \(\vect\) is idempotent i.e.  \(\vect(\vect(A)) =
\vect(A)\).

Diagonalizing a vector is the process of turning it into a squared matrix with
every element equal to zero except for the diagonal which is occupied by said
vector. Diagonalizing a square matrix is the process of extracting is diagonal
as a vector. If we \(\vec a\) be a vector of size \(n\) we define them formally
as
\begin{eqnarray*}
    % https://en.wikipedia.org/wiki/Diagonal_matrix
    \Diag(\vec a) &=& \vec a \vec 1' \hadam I, \\
    % https://math.stackexchange.com/a/3122442
    \diag(S) &=& (S \hadam I) \vec 1.
\end{eqnarray*}

If \(B\) has the same size as A and the Hadamard product is being vectorized
than it can be removed from an expression as such \[\vect(A \hadam B) =
\Diag(\vect(A)) \vect(B).\] For vectors we can do a similar thing, let \(\vec
x\) and \(\vec y\) have the same size, then \[\vec x \hadam \vec y = \Diag(\vec
x) \vec y.\]

It is interesting to note these connections between the Hadamard product and
matrix multiplication, if \(a\) has size \(m,\) then \(A \hadam \vec 1 \vec a' =
A \Diag(\vec a).\)


The trace of a square matrix is the sum of its diagonal elements. We define it
as \[\trace(S) = \vec 1' \diag(S) \vec 1.\]

This function is useful due to its many properties that make many index free
manipulation easier. In particular we have that \[\trace(A (B \hadam C)) =
\trace((A \hadam B')C)\] and that it is circulant shift invariant \[\trace(ABC)
= \trace(CAB) = \trace(BCA).\] Please note that you cannot generally commute
matrix products inside the trace function, you can just ``shift'' them.

The trace is a linear function, that is, if \(Q\) has the same size as \(S\)
then \(\trace(S + Q) = \trace(S) + \trace(Q)\) and if \(a\) is any scalar then
\(\trace(a \cdot S) = a \trace(S).\) We have similar results for the transpose
and element-wise functions e.g. if \(B\) has the same size as \(A\) \((A + B)' =
A' + B'\) and \((A \hadam B)' = A' \hadam B'.\)

% https://scicomp.stackexchange.com/a/23781

The Frobenius inner product is just a generalization of the familiar dot product
between two vectors. If \(B\) has a compatible dimension with \(A\) then we
denote it as \(\langle A,B\rangle\) and \[\langle A,B \rangle = \vect(A)'
\vect(B).\] A useful equality regarding the trace is the following \[\langle A,
B \rangle = \trace(A'B) = \vec 1' (A \hadam B) \vec 1.\]

The Frobenius norm is just a generalization of the familiar Euclidean norm. We
denote it as \(\Vert A\Vert\) and \[\Vert A \Vert = \sqrt{\langle A, A \rangle}.\]

If \(B\) has the same size as \(A\) and their columns represent a batch, we can
express neatly the sum of the mean squared errors of the batch as \(\mathrm{mse}
= \left(1/n \cdot \vec 1' (A - B)^{{\cdot}2}\right) \vec 1.\) We can note that
\[
    \mathrm{mse}
    = \frac 1 n \vec 1' (A - B)^{{\cdot}2} \vec 1
    = \frac 1 n \vect(A - B)' \vect(A - B)
    = \frac 1 n \Vert A - B\Vert^2.
\]

Finally we have that if \(A,\) \(B\) and \(C\) have compatible sizes
\[\vect(ABC) = (C' \krone A) \vect(B).\] In particular \[\vect(AB) = (I \krone
A) \vect(B) = (B' \krone A) \vect(I) = (B' \krone I) \vect(A) \mathrm,\] that
is with \(\vect\) we can transform a matrix multiplication in a matrix-vector
one.

\chapter{Back-propagation for Scalars}

% https://www.youtube.com/watch?v=mICbKwwHziI
% https://www.astrophys-neunhof.de/mtlg/se77211.pdf

The objective of this work is to remove the magic from the libraries used for
deep learning, in particular how the back-propagation algorithm is implemented
in the general case. Understanding this algorithm is crucial for implementing
one such library.

To do so we are going to start by explaining the trivial case of the
back-propagation for scalars. This is going to allow us to get some familiarity
with the general mechanism involved and to learn a bit about the differences with
other algorithmic differentiation processes.

Finally, we are going to see what are why is necessary to generalize
back-propagation and what problems with generalizing these algorithms to other
kind of ``numerical objects'' like vectors and matrices.

Back-propagation is a special case of automatic differentiation in reverse
accumulation mode. In particular, it is the case in which the function that we
are differentiating has only one (scalar) output. For neural network training
this scalar output is the loss.

\section{What back-propagation is not}

Depending on how you count, there are two to three main classes of automatic
method to calculate the gradient: numerical differentiation, symbolic
differentiation and automatic differentiation.

Numerical differentiation is done basically by applying the limit definition of
the derivative.\footnote{There are improved ways of doing this calculation but
they all rely on the same principle.} 
For example, let \(\vec f : \mathcal R^2 \to
\mathcal R\). If  we want to calculate \(\der\vec f(a,b) =
\left[\begin{array}{cc} \partialfrac{\vec f(a,b)}a &
\partialfrac{\vec f(a,b)}b\end{array}\right]\) we can approximate
it by doing
\begin{eqnarray*}
    \partialfrac{\vec f(a,b)}a
    &\approx& \frac{\vec f(a + h,b) - \vec f(a,b)}h \\
    \partialfrac{\vec f(a,b)}b
    &\approx& \frac{\vec  f(a,b + h) - \vec f(a,b)}h
\end{eqnarray*}
where \(h>0\) is a small number. This method is really easy to implement but it
has two great problems that cannot be solved. The first is that it has numerical
accuracy issues (the reasons for which this is the case are beyond the scope of
this work), the second one is that requires \(O(n)\) evaluations of \(f\) to
calculate the gradient, where \(n\) is the number of argument of the function,
in this case 2.

Symbolic differentiation is usually done when manipulating mathematical
expressions as a data structure. This method can solve the numerical stability
issues of the previous one, but if implemented in a naive way can lead to a
problem known as \emph{expression swell}. This basically amounts to an
exponential increase in the size of the formula (due to careless application of
differentiation rules) and in a lot of duplicated work for calculating again and
again parts of the formula that are identical but put in different places. To
give a trivial example, let \(f(a) = e^a g(a)\), now let's take its derivative
\[\der f(a) = e^a g(x) + e^a \der g(x),\] as you can see the careless
application of the derivative rule for multiplication made us increase the size
of the formula unnecessarily and made us recalculate the value of \(e^a\). A
more efficient formulation would have been \[\der f(a) = e^x (g(x) + \der
g(x)).\]

A better implementation of symbolical differentiation (i.e. one that involves
common sub-expression elimination) is functionally equivalent to automatic
differentiation, even with respect to control flow, as shown by S\"oren
Laue~\cite{laue2019}.\footnote{The difference in the name comes from the fact
that symbolic differentiation comes from a different approach to the problem and
it is usually used in computer algebra systems like Mathematica.}.

Automatic differentiation~\cite{baydin2017} has two main modes known as forward
and reverse. For the needs of this work, and most of machine learning, only the
reverse mode is used.\footnote{The forward mode is used for calculations
involving Hessians, so it definitely has its uses, but we are not going to see
them.} In particular, in the field of neural networks, the reverse-mode
automatic differentiation algorithm for the function of a scalar value (i.e. the
loss) has been independently rediscovered and given the name
back-propagation~\cite{rumelhart1986}. A Venn diagram showing where the
back-propagation algorithm is located in the automatic differentiation
literature is shown in figure~\ref{fig:ad-venn}.

\begin{figure}
\centering
\includegraphics{figures.1}
\caption{Venn diagram showing the various modes for accumulated gradients in the
automatic differentiation literature. FWD stands for forward-mode and BWD for
backward-mode. MIX is used to represent all the methods that interleave forward
and backward gradient accumulation, which in the general case are necessary since
gradient accumulation in NP-complete.}
\label{fig:ad-venn}
\end{figure}

In the next section, we are going to see what this algorithm is about and what
its advantages are.

\section{An illustrative example}

We are going to start with an example from~\cite{baydin2017}. Let \(y = f(a,b) =
\ln(a) + a b - \sin(b)\), say we compute \(y = f(2,5)\) and we are interested in
its gradient i.e. \(\partialfrac y a\) and \(\partialfrac y b.\) Let us start by
evaluating the function one elementary operation at a time in the following way:
\begin{eqnarray*}
a = x^{[-1]} &=& 2 \\
b = x^{[0]}  &=& 5 \\
    x^{[1]}  &=& \ln\left(x^{[-1]}\right) = \ln(2) \\
    x^{[2]}  &=& x^{[-1]} \times x^{[0]} = 2 \times 5 \\
    x^{[3]}  &=& \sin\left(x^{[0]}\right) = \sin(5) \\
    x^{[4]}  &=& x^{[1]} + x^{[2]} \approx 0.693 + 10 \\
y = x^{[5]}  &=& x^{[4]} + x^{[3]} \approx 10.693 + 0.959
\end{eqnarray*}

We know from basic calculus that \[
    \partialfrac y b = \partialfrac{x^{[5]}}{x^{[0]}}
    = \overbrace{
        \partialfrac{x^{[5]}}{x^{[4]}} \partialfrac{x^{[4]}}{x^{[2]}}
    }^{\partialfrac{x^{[5]}}{x^{[2]}}} \partialfrac{x^{[2]}}{x^{[0]}}
    + \partialfrac{x^{[5]}}{x^{[3]}} \partialfrac{x^{[3]}}{x^{[0]}},
\] and similarly for the other derivative we have that \[
    \partialfrac y a = \partialfrac{x^{[5]}}{x^{[-1]}}
    = \overbrace{
        \partialfrac{x^{[5]}}{x^{[4]}} \partialfrac{x^{[4]}}{x^{[1]}}
    }^{\partialfrac{x^{[5]}}{x^{[1]}}} \partialfrac{x^{[1]}}{x^{[-1]}}
    + \overbrace{\partialfrac{x^{[5]}}{x^{[4]}} \partialfrac{x^{[4]}}{x^{[2]}}
    }^{\partialfrac{x^{[5]}}{x^{[2]}}} \partialfrac{x^{[2]}}{x^{[-1]}}
\]

If we let \(\bar x^{[i]} = \partialfrac y {x^{[i]}} =
\partialfrac{x^{[5]}}{x^{[i]}},\) that is the sensitivity of the output w.r.t. a
change in that variable, we can incrementally calculate both \(\bar x^{[-1]}\)
and \(\bar x^{[0]}\) in the following way:
\begingroup
\renewcommand{\partialfrac}[2]{\textstyle\frac{\partial#1}{\partial#2}}
\begin{eqnarray*}
    \bar x^{[5]}    &=& \partialfrac{x^{[5]}}{x^{[5]}}
        = 1 \\
    \bar x^{[4]}    &=& \bar x^{[5]}\partialfrac{x^{[5]}}{x^{[4]}}
        = \bar x^{[5]} \times 1 = 1 \\
    \bar x^{[3]}    &=& \bar x^{[5]}\partialfrac{x^{[5]}}{x^{[3]}}
        = \bar x^{[5]} \times (-1) = -1 \\
    \bar x^{[1]}    &=& \bar x^{[4]}\partialfrac{x^{[4]}}{x^{[1]}}
        = \bar x^{[4]} \times 1 = 1 \\
    \bar x^{[2]}    &=& \bar x^{[4]}\partialfrac{x^{[4]}}{x^{[2]}}
        = \bar x^{[4]} \times 1 = 1 \\
    \bar x^{[0]}    &=& \bar x^{[3]}\partialfrac{x^{[3]}}{x^{[0]}}
        = \bar x^{[3]} \times \cos\left(x^{[0]}\right) \approx -0.282 \\
    \bar x^{[-1]} &=& \bar x^{[2]}\partialfrac{x^{[2]}}{x_{-1}}
        = \bar x^{[2]} \times x^{[0]} = 5 \\
    \bar x^{[0]}    &\pluseq\phantom+& \bar x^{[2]}\partialfrac{x^{[2]}}{x^{[0]}}
        = \bar x^{[2]} \times x_{-1} \approx 1.716 \\
    \bar x^{[-1]} &\pluseq\phantom+& \bar x^{[1]}\partialfrac{x^{[1]}}{x^{[-1]}}
        = \frac{\bar x^{[1]}}{x^{[-1]}} = 5.5 \\
\end{eqnarray*}
\endgroup

In a nutshell, this is the back-propagation algorithm. When performing the
calculations for our function, we record the dependency graph, as shown in
figure~\ref{fig:dag}, and we \emph{adjoint} \(\bar x^{[i]}\) to each variable
\(x^{[i]}\). When we have obtained our scalar output, we topologically sort our
graph and then start to propagate the gradient backward, starting from the
scalar output and ending at our input variables. In this way, we obtain the
gradient of our loss.

\begin{figure}
    \centering
    \includegraphics{figures.2}
    \caption{Dependency graph of \(f.\)}
    \label{fig:dag}
\end{figure}

This algorithm has no numerical stability issues, as opposed to numerical
differentiation, and has a cost that is proportional to evaluating the
function in any case, in contrast with naive symbolic differentiation.

This is also true for its generalization, reverse-mode automatic
differentiation, which is a numerically accurate and efficient algorithm for
calculating the derivatives of a function \(\vec f : \mathcal R^n \to \mathcal
R^m,\) where \(n>>m.\)

We can summarize what we have done until now using the C99-like pseudocode in
figure~\ref{fig:pseudo}.

% https://tex.stackexchange.com/questions/94699/absolutely-definitely-preventing-page-break
\newenvironment{absolutelynopagebreak}
  {\par\nobreak\vfil\penalty0\vfilneg
   \vtop\bgroup}
  {\par\xdef\tpd{\the\prevdepth}\egroup
   \prevdepth=\tpd}

\begin{figure}
\centering
\begin{minipage}{.7\textwidth}
% https://latexref.xyz/tabbing.html
\begin{tabbing}
void \= backprop(int l) \{\+\\
    toposort(tape);\\
    for (int i = 0; i \textless{} l; i++) grad[i] = 0;\\
    grad[l] = 1;\\
    /\textasteriskcentered{}
    Assuming only binary operations.
    \textasteriskcentered/\\
    for \=(int i = l; i \textgreater{} 0; i-{}-) \{\+\\
        int \=l = tape[i].left, r = tape[i].right, \\
            \>op = tape[i].op;\\
        grad[l] += vjp[op].l(grad[i], val[r], val[l]);\\
        grad[r] += vjp[op].r(grad[i], val[r], val[l]);\-\\
    \}\-\\
\}\\
\end{tabbing}
\end{minipage}
\caption{Pseudo-code for the back-propagation algorithm. We have three arrays:
tape, val and grad. The first one records the dependencies among the various
variables that make the computation, the second one stores the value of each of
the intermediate results for computing the function and the last one the
gradient of the output w.r.t that variable. The vjp function stands for
vector-Jacobian product, we are going to elaborate further on this in
section~\ref{sec:derivations}, right now you can think of it as the
multiplication between the gradient accumulated up to that point and the
derivative of the current function.}
\label{fig:pseudo}
\end{figure}

\section{The need for generalization}

The algorithm we have described before is fairly easy to implement and it can be
used, without any modification, to calculate losses of any kind of neural
networks.

There is a catch, let us consider an operation like matrix multiplication, which
is ubiquitous in neural networks, let us also assume that we can perform it in
\(O(n^2)\) operations. For each operation that we perform, we have to record some
amount of information, for the sake of the argument say that 1byte
is enough. This means that if we have to multiply two matrices of size \(1024
\times 1024\) we would have to keep in memory 1Mib just to remember the
operations that happen inside the matrix multiplication. If we use a realistic
algorithm for matrix multiplication and a realistic size for the number of bytes
that we need to record a single operation, we are very quickly in a much worse
situation.

The bad news is that it does not finish here, since keeping track of each scalar
makes it harder to organize the data in memory in such a way that these
calculations can be performed by BLAS routines, therefore even performance
suffers.

Therefore we need to group numbers in a way that alleviates our problem related
to tracking each number, we need matrices. The first thing that comes is to use
vectors, this is a fine choice if we plan to only use our network for inference
on a single datum. This alleviates both the problems that we have regarding the
memory needed for keeping track of many operations, since now a multiplication
between a matrix and a vector is a single operation, and of performance since it
is now much easier to use BLAS primitives like GEMV to implement our operations.

If we are also interested in training neural networks, we can do even better.
Training is often performed on batches (or mini-batches) of data by
summing the loss of each datum in the batch and this can be naturally expressed
using matrices.\footnote{This is also true if we are interested in performing
inference in batch.}

Let us give a practical example. Let \(\vec x\) be a datum and \(\vec y\) our
desired outcome for it, we can calculate the mean squared error loss of a
2-layer neural network, with a sigmoid activation function \(\vec s\), in the
following way \[\frac 1 n \Vert\left(W^{[2]} \vec s\left(W^{[1]} \vec x + \vec
b^{[1]}\right) + \vec b^{[2]}\right) - \vec y\Vert^2.\] Now if we let \(X\) be a
batch of data and \(Y\) a batch of desired outcomes for the elements in batch,
we can easily generalize this expression to an entire batch using matrices \[
    \frac 1 n \left\Vert
        \left(W^{[2]} S\!\left(W^{[1]} X + \vec b^{[1]} \vec 1'\right) + \vec b^{[2]} \vec 1'\right) - Y
    \right\Vert^2.
\]

In this way, we can always use single matrix operations to perform
calculations on the entire batch, independently of the number of elements in it,
we can achieve better throughput with BLAS primitives like GEMM.

We have now seen how by keeping the same semantics of the code that operates on
single scalars, we can gain performance benefits by simply switching the
mathematical objects at the base of our algorithm. This also makes operator
fusion easier and more effective.

\section{Problems with generalization}

\begingroup
\renewcommand\matrix{\left[\begin{array}{ccc}&&\\&&\\&&\end{array}\right]}
\renewcommand\vector{\left[\begin{array}{ccc}&&\end{array}\right]}

Let us now focus on the vector case, in particular of a function of vector
argument and scalar value that is internally composed of functions of vector
value and argument. If the input value is \(\vec x^{[0]}\) and the values
computed by the various functions are \(\vec x^{[1]}, \ldots, x^{[n]}\)
terminating in the scalar output of the last function, then we can write the
derivative of the output w.r.t. the input using the chain rule as
\[\partialfrac{x^{[n]}}{\vec x^{[n-1]}} \partialfrac{\vec x^{[n-1]}}{\vec
x^{[n-2]}} \cdots \partialfrac{\vec x^{[1]}}{\vec x^{[0]}}.\] If we focus only
on the shape of the objects along the chain (i.e. the vectors and matrices) we
can visualize the chain rule in this way\[\vector \matrix \dots \matrix.\]
Viewing it like this it is obvious that it is more convenient to perform our
calculation by multiplying the objects from left to right.
\endgroup

Even if we use the right order for the multiplications, which is the one
naturally imposed by the topological sort, we are still wasting a lot of
resources since usually the matrices \(\partialfrac{\vec x^{[j]}}{\vec
x^{[i]}}\) along the chain are very sparse and most of the memory we allocate
for them is going to be filled with zeros.

If we ignore this problem for now, implementing back-propagation for the loss of
vector argument is not much harder than the scalar version since taking
derivatives of a function of vector argument and value (that can happen in the
middle of the neural network) is common knowledge, just a standard vector
calculus.

When we start to take into account matrices this creates the possibility of
having functions of matrix argument and value in the middle of our computation,
and taking derivatives of this kind of function is not common knowledge.

The reader might be surprised by the fact that this notion for matrix calculus
is not only uncommon but it also has no standardized arrangement for the partial
derivatives. While the end result is always going to be some mathematical object
that contains all possible partial derivatives w.r.t. each single element in the
input and argument matrix, how they should be organized has not been settled.
This has important repercussions in how fundamental calculations (like the chain
rule~\cite{magnus2010}) have to be performed for these objects.

We may ask ourselves: can we sidestep this problem by performing this
calculation explicitly i.e.  decomposing the matrix operations in their
elementary operations and perform the derivation on that? This can be a
worthwhile exercise for small examples but proving anything in this way for the
general case rapidly becomes unwieldy, and we think that anyone can agree that
like for the vector case were developing a general symbolic method not only
helps the derivation but also the understanding and new discoveries.

Now that we have convinced ourselves that we need some robust method to go about
this matrix derivation business, as a humble computer scientist which options do
we have at our disposal?

The go-to reference for this kind of thing is The Matrix
Cookbook~\cite{petersen2012}. While this is a terrific reference for many things
and contains some of the answers we need, the way in which they are presented
does not make it obvious that it is what we are looking for. So without
additional information, it cannot help us.

Back-propagation is a specific case of the more general reverse-mode automatic
differentiation algorithm, which together with its ``dual'' algorithm
forward-mode automatic differentiation, has a good deal of literature on it. In
particular, there is a collection of results like~\cite{giles2008}, but as the
name may suggest they are just results without any satisfactory explanation
(again, unless pretty much you already know the answers and need reference).
This can certainly allow you to implement the back-propagation through the most
common operations, but this leaves you unable to derive your own results or
possible improvements for special cases that you care about. Most importantly,
this does not leave you with a good understanding of what is going on in the
algorithm you are implementing (without even talking about how one may approach
debugging this kind of numerical algorithm.)

If one starts looking around for answers, one may encounter the so-called
``adjoint trick'' for deriving these results or an even more general method
based on tensor calculus~\cite{laue2018, laue2020}, but they usually involve
more math than a humble computer scientist can stomach.

Together with the difficulty of derivation now the problem that we have cited
before is even more accentuated.  Let \(F\) be a function of size \(m \times n\)
with argument \(X\) of size \(p \times q\). However, we are going to define its
derivative at \(X\) it has to contain the \(mnpq\) partial derivatives. To be
more concrete, let us consider \(m=n=p=q=1024\), and say that we store all the
entries of our matrices in the IEEE-754 single precision floating point format.
This means that if we want to store this derivative we are going to need 4TiB of
memory, most of which is still going to be just zeros.

\chapter{Back-propagation for Matrices}

In this chapter, we are going to explain how to generalize the previously
introduced back-propagation for scalars to functions of matrix value and
argument. We start with an introduction of an elegant matrix calculus that is
going to greatly simplify our treatment of the subject.

\section{Matrix Calculus}\label{sec:matcalc}

This section is taken from the work of Magnus, Neudecker and Abadir~\cite{neudecker1969,
magnus1985, magnus2005, magnus2007, magnus2010, magnus2019}. 
Also~\cite{liu2022} gives a good perspective on the field of matrix
differential calculus and its development in general.

Matrix calculus is the less known, bigger cousin, of vector calculus. So let us
start with a quick refresher on the latter.

Now let \(\vec f\) be a differentiable function, and let \(\vec y = \vec f(\vec
x)\) for some \(\vec x\), with \[
    \der\vec f(\vec x)
    = \partialfrac{\vec y}{\vec x'}
    = \left[\begin{array}{c|c|c}
        \partialfrac{\vec y}{\vec x_1} & \cdots & \partialfrac{\vec y}{\vec x_n}
    \end{array}\right]
    = \left[\begin{array}{c|c|c}
        \partialfrac{\vec y_1}{\vec x_1} & \cdots & \partialfrac{\vec y_1}{\vec x_m} \\
        \vdots & \ddots & \vdots \\
        \partialfrac{\vec y_n}{\vec x_1} & \cdots & \partialfrac{\vec y_n}{\vec x_m}
    \end{array}\right]
\] we denote the derivative (or Jacobian) of \(\vec f\) at \(\vec x\).

We now want to emphasize a few facts. We explicitly use the numerator layout.
Numerator layout means that the Jacobian is going to have in each column each
element \(\vec y_i\) and in each row each \(\vec x_i\). We say explicitly it
because we are going to write \(\partialfrac{\vec y}{\vec x'}\), to emphasise
the fact that \(\vec x\) is laid out as a row and \(y\) as a column. In the
special case where our differentiable function is \(y\) and \(y = f(x)\) we have
that \[
    \der f(\vec x) = \partialfrac y {\vec x'} = \left[\begin{array}{c|c|c}
        \partialfrac y {\vec x_1} & \cdots & \partialfrac y {\vec x_n}
    \end{array}\right]
\] is a row vector, that is also known as the gradient of \(f\) at \(\vec x\),
written as \(\nabla f(x).\)

There is not a conventional definition for matrix valued functions of matrix
argument. The one we are going to use in this work is the one from Magnus and
Naudecker~\cite{magnus2019}. We have decided to adopt their definition because we have
found it is the most useful for our use case, and we hope that it becomes the standard
for machine learning related derivation. They go into great detail on why this
definition makes sense and why it is mathematically sound, we are not going to
elaborate on this matter. The interested reader is encouraged to read their
work for further details.

To make our lives easier we are going to approach matrix calculus using
differentials rather than derivatives.

% https://en.wikipedia.org/wiki/Differential_of_a_function
Let us start by defining the differential for the scalar case. Remember that
\[\lim_{h \to 0} \frac{f(x + h) - f(x)}h = \der f(x)\] which can be rewritten as
\[f(x + h) - f(x) = (\der f(x))h + r(h,x),\] \[f(x + h) = f(x) + (\der f(x))h +
r(h,x),\] were \(\lim_{h \to 0} r(h,x) = 0\) and it can be thought of as
a reminder or error term. We can now define the differential of \(f\) at \(x\)
with increment \(h\) as \(\dif f(x;h) = (\der f(x))h\), but usually written as
\((\der f(x))\dif x.\)

We can now do a similar reasoning for vector functions and get \[\vec f(\vec x +
\vec h) = \vec f(\vec x) + (\der\vec f(\vec x))\vec h + \vec r(\vec h,\vec x)\]
were \(\dif \vec f(\vec x;\vec h) = (\der \vec f(\vec x))\vec h,\) and it is
usually written as \((\der\vec f(\vec x))\dif\vec x.\)

Again with a similar reasoning we can generalize our definition to the matrix
case using the \(\vect\) function as such \[\vect(F(X+U)) = \vect(F(X)) +
(\der\vect(F(X))\vect(U) + \vect(R(U,X)).\]

Therefore now if \(Y=F(X)\) we now have that \[\der F(X) =
\partialfrac{\vect(Y)}{\vect(X)'}\] i.e.  the derivative (or Jacobian) of \(F\)
at \(X.\)

We now need a way to find the derivative from an expression involving
differentials and some way to perform the chain rule but with differentials. The
ways to do this thing are called identification results and Cauchy's rule of
invariance. The identification rules are \[\dif\vec f(\vec x) = A(\vec x)
\dif\vec x \iff \der \vec f(\vec x) = A(\vec x)\] and \[\dif \vect(F(X)) = A(X)
\dif\vect(X) \iff \der F(X) = A(X).\] These two rules, together with their
special cases, are reported in table~\ref{tab:identifications}.

\begin{table}
\centering
\begin{tabular}{cccc}
 & \(x\) & \(\vec x\) & \(X\) \\[2pt]
         \cline{2-4}
\multicolumn{1}{r|}{\(\dif f =\)}
    & \(a(x) \dif x\) & \(\vec a(\vec x)' \dif\vec x\) & \(\vect(A(X))' \dif\vect(X)\) \\
\multicolumn{1}{r|}{\(\dif\vec f =\)}
    & \(\vec a(x) \dif x\) & \(A(\vec x) \dif\vec x\) & \(A(X) \dif\vect(X)\) \\
\multicolumn{1}{r|}{\(\dif F =\)}
    & \(\vect(A(x)) \dif x\) & \(A(\vec x) \dif\vec x\) & \(A(X) \dif\vect(X)\)
\end{tabular}
\caption{Identification theorems~\cite{kinghorn1996}.}
\label{tab:identifications}
\end{table}

These two rules mean that if we can manipulate an expression involving
differentials so that it has the form on the right side of the equal sign in the
left part of the logical equivalence, then we can ``harvest'' the derivative.

Cauchy's rule of invariance states that \[\dif C(X; H) = \dif B(A(X); \dif
A(X;H))\] which, if you squint hard enough, you notice that is it the same as the
chain rule \[\der C(X) = \der B(A(X)) \der A(X)\] but for differential.

We are now ready to state some useful properties of these differentials, let
\(A\) an \(n \times m\) be a constant, \(X\) and \(Y\) variables of compatible
sizes and \(S\) a square variable, then
\begin{eqnarray*}
    \dif A            &=& \leftidx{_n}{\vec 0}{}\,\leftidx{_m}{\vec 0}{'}, \\
    \dif (a \cdot X)  &=& a \cdot \dif X, \\
    \dif (X + Y)      &=& \dif X + \dif Y, \\
    \dif (x \cdot X)  &=& (\dif x) \cdot X + x \cdot \dif X, \\
    \dif (XY)         &=& (\dif X)Y + X \dif Y, \\
    \dif (X \krone Y) &=& (\dif X) \krone Y + X \krone \dif Y, \\
    \dif (X \hadam Y) &=& (\dif X) \hadam Y + X \hadam \dif Y, \\
    \dif X^{{\cdot}a} &=& a \cdot X^{{\cdot}a-1} \hadam \dif X, \\
    \dif a^{{\cdot}X} &=& \ln(a) \cdot a^{{\cdot}X} \hadam \dif X \\
    \dif X'          &=& (\dif X)', \\
    \dif\vect(X)      &=& \vect(\dif X), \\
    \dif\trace(S)     &=& \trace(\dif S).
\end{eqnarray*}

The rules we have stated are in the general form and can appropriately be
specialized for scalar and vector cases.

After all this, we are finally ready for an example! Consider the softmax
function for vectors \(\vec s(\vec x)\), we now proceed to find the value of
\(\der\vec s(\vec x)\). First, let \(\vec y = e^{{\cdot}\vec x}\)for brevity,
then we have that
\begin{eqnarray}
\dif\vec s(\vec x)
&=& \dif\left((\vec 1' \vec y)^{-1} \cdot \vec y\right)\nonumber\\
&=& \dif\left((\vec 1' \vec y)^{-1}\right) \cdot \vec y
    + (\vec 1' \vec y)^{-1} \cdot \dif\vec y\nonumber\\
&=& (\vec 1' \vec y)^{-1} \cdot \dif\vec y
    - (\vec 1' \vec y)^{-2} \cdot \dif(\vec 1' \vec y) \cdot \vec y\nonumber\\
&=& (\vec 1' \vec y)^{-1} \cdot \dif\vec y
    - (\vec 1' \vec y)^{-2} \cdot (\vec 1' \dif\vec y) \cdot \vec y\nonumber\\
&=& (\vec 1' \vec y)^{-1} \cdot \dif\vec y
    - (\vec 1' \vec y)^{-2} \cdot \vec y \cdot (\vec 1' \dif\vec y)\nonumber\\
&=& (\vec 1' \vec y)^{-1} \cdot \dif\vec y
    - (\vec 1' \vec y)^{-2} \cdot \vec y \vec 1' \dif\vec y\label{eq:switcheroo}\\
&=& \left((\vec 1' \vec y)^{-1} \cdot I
    - (\vec 1' \vec y)^{-2} \cdot \vec y \vec 1'\right) \dif\vec y\nonumber\\
&=& \left((\vec 1' e^{{\cdot}\vec x})^{-1} \cdot I
    - (\vec 1' e^{{\cdot}\vec x})^{-2} \cdot e^{{\cdot}\vec x} \vec 1'\right)
    \dif e^{{\cdot}\vec x}\nonumber\\
&=& \left((\vec 1' e^{{\cdot}\vec x})^{-1} \cdot I
    - (\vec 1' e^{{\cdot}\vec x})^{-2} \cdot e^{{\cdot}\vec x} \vec 1'\right)
    e^{{\cdot}\vec x} \hadam \dif \vec x\nonumber\\
&=& \left((\vec 1' e^{{\cdot}\vec x})^{-1} \cdot I
    - (\vec 1' e^{{\cdot}\vec x})^{-2} \cdot e^{{\cdot}\vec x} \vec 1'\right)
    \Diag(e^{{\cdot}\vec x}) \dif \vec x\nonumber\\
&=& \left((\vec 1' e^{{\cdot}\vec x})^{-1} \cdot \Diag(e^{{\cdot}\vec x})
    - (\vec 1' e^{{\cdot}\vec x})^{-2} \cdot e^{{\cdot}\vec x} \vec 1' \Diag(e^{{\cdot}\vec x})\right)
    \dif \vec x\nonumber\\
&=& \left((\vec 1' e^{{\cdot}\vec x})^{-1} \cdot \Diag(e^{{\cdot}\vec x})
    - (\vec 1' e^{{\cdot}\vec x})^{-2} \cdot e^{{\cdot}\vec x} (e^{{\cdot}\vec x})'\right)
    \dif \vec x\nonumber\\
&=& (\Diag(\vec s(\vec x)) - \vec s(\vec x) \vec s(\vec x)') \dif\vec x\nonumber
\end{eqnarray}
Now thanks to the identification theorem, we have found our derivative.  In the
equation~\ref{eq:switcheroo} a step has been performed that requires a bit of
justification, i.e. the fact that \(\vec y \cdot (\vec 1' \dif\vec y) = \vec y
\vec 1' \dif\vec y.\) This is simply explained if we think about the fact that
\(n \cdot \vec y = \vec y \vec n',\) where \(n\) is a vector of size 1, and how
matrix multiplication is defined.

It is interesting to note that this result can be rewritten like this
\[\Diag(\vec s(\vec x)) - \vec s(\vec x)^{{\cdot}2} \vec 1'.\] We will in
section~\ref{sec:broadcasting} that formulations involving outer products with
\(\vec 1\) can usually be implemented efficiently with broadcasting.

We have found our first derivative, of a non-trivial function, using
differentials and the identification theorem. The process was a bit long to make
sure that things were as clear as possible, but with a bit of experience, the
process can be streamlined by skipping some (or a lot) of steps.

\section{Derivations}\label{sec:derivations}

We have now almost all the instruments we need to derive the back-propagation
formulas.

Let us revamp the example from before using matrices, therefore let \(A\) and
\(B\) be two square matrices and \(y = f(A,B) = \Vert\ln(A) + A B +
\sin(B)\Vert^2.\) Again say we are interested in \(\partialfrac y A\) and
\(\partialfrac y B.\) We can again evaluate the function one elementary
operation at the time
\begin{eqnarray*}
A = X^{[-1]} &=& 2 \\
B = X^{[0]}  &=& 5 \\
    X^{[1]}  &=& \ln\left(X^{[-1]}\right) \\
    X^{[2]}  &=& X^{[-1]} \times X^{[0]} \\
    X^{[3]}  &=& \sin\left(X^{[0]}\right) \\
    X^{[4]}  &=& X^{[1]} + X^{[2]} \\
    X^{[5]}  &=& X^{[4]} + X^{[3]} \\
y = x^{[6]}  &=& \Vert X^{[5]}\Vert^2.
\end{eqnarray*}

Thanks to how we have defined our matrix calculus we have that
\begin{eqnarray*}
    \partialfrac y {\vect(B)'}
    &=& \partialfrac{x^{[6]}}{\vect\left(X^{[0]}\right)'} \\
    &=& \underbrace{
        \overbrace{
            \partialfrac{x^{[6]}}{X^{[5]}} \partialfrac{X^{[5]}}{X^{[4]}}
        }^{\partialfrac{x^{[6]}}{\vect\left(X^{[4]}\right)'}}
        \partialfrac{X^{[4]}}{X^{[2]}}
    }_{\partialfrac{x^{[6]}}{\vect\left(X^{[2]}\right)'}}
    \partialfrac{X^{[2]}}{X^{[0]}}
    + \overbrace{
        \partialfrac{x^{[6]}}{X^{[5]}}
        \partialfrac{X^{[5]}}{X^{[3]}}
    }^{\partialfrac{x^{[6]}}{\vect\left(X^{[3]}\right)'}
    } \partialfrac{X^{[3]}}{X^{[0]}}
\end{eqnarray*}
\begin{eqnarray*}
    \partialfrac y {\vect(A)'}
    &=& \partialfrac{x^{[6]}}{\vect\left(X^{[-1]}\right)'} \\
    &=& \underbrace{
        \overbrace{
            \partialfrac{x^{[6]}}{X^{[5]}} \partialfrac{X^{[5]}}{X^{[4]}}
        }^{\partialfrac{x^{[6]}}{\vect\left(X^{[4]}\right)'}}
        \partialfrac{X^{[4]}}{X^{[2]}}
    }_{\partialfrac{x^{[6]}}{\vect\left(X^{[2]}\right)'}} \partialfrac{X^{[2]}}{X^{[{-1}]}}
    + \underbrace{
        \overbrace{
            \partialfrac{x^{[6]}}{X^{[5]}} \partialfrac{X^{[5]}}{X^{[4]}}
        }^{\partialfrac{x^{[6]}}{\vect\left(X^{[4]}\right)'}}
        \partialfrac{X^{[4]}}{X^{[1]}}
    }_{\partialfrac{x^{[6]}}{\vect\left(X^{[1]}\right)'}}
    \partialfrac{X^{[1]}}{X^{[-1]}}
\end{eqnarray*}
where some \(\vect\) and transpose are omitted for simplicity.

Now let us proceed with the back-propagation algorithm step by step. Let
\(\vect\left(\bar X^{[i]}\right)' = \partialfrac y {\vect\left(X^{[i]}\right)'}
= \partialfrac{x^{[6]}}{\vect\left(X^{[i]}\right)'}\) then

\begingroup
\renewcommand{\partialfrac}[2]{\textstyle\frac{\partial#1}{\partial#2}}
\let\oldvect\vect
\renewcommand\vect[1]{\oldvect\mathchoice{\!}{}{}{}\left(#1\right)}
\begin{eqnarray*}
\bar x^{[6]}          &=& \partialfrac{x^{[6]}}{x^{[6]}} = 1 \\
\left(\bar{\vec x}^{[5]}\right)' &=& \bar x^{[6]} \partialfrac{x^{[6]}}{\vect{X^{[5]}}'} \\
\vect{\bar X^{[4]}}'  &=& \left(\bar{\vec x}^{[5]}\right)' \partialfrac{\vect{X^{[5]}}}{\vect{X^{[4]}}'} \\
\vect{\bar X^{[3]}}'  &=& \left(\bar{\vec x}^{[5]}\right)' \partialfrac{\vect{X^{[5]}}}{\vect{X^{[3]}}'} \\
\vect{\bar X^{[1]}}'  &=& \vect{\bar X^{[4]}}' \partialfrac{\vect{X^{[4]}}}{\vect{X^{[1]}}'} \\
\vect{\bar X^{[2]}}'  &=& \vect{\bar X^{[4]}}' \partialfrac{\vect{X^{[4]}}}{\vect{X^{[2]}}'} \\
\vect{\bar X^{[0]}}'  &=& \vect{\bar X^{[3]}}' \partialfrac{\vect{X^{[3]}}}{\vect{X^{[0]}}'} \\
\vect{\bar X^{[-1]}}' &=& \vect{\bar X^{[2]}}' \partialfrac{\vect{X^{[2]}}}{\vect{X^{[-1]}}'} \\
\vect{\bar X^{[0]}}'  &\pluseq\phantom+& \vect{\bar X^{[2]}}' \partialfrac{\vect{X^{[2]}}}{\vect{X^{[0]}}'} \\
\vect{\bar X^{[-1]}}' &\pluseq\phantom+& \vect{\bar X^{[1]}}' \partialfrac{\vect{X^{[1]}}}{\vect{X^{[-1]}}'}
\end{eqnarray*}
\endgroup

In general, what we are doing is always, while calculating the derivative of  a
scalar value function \(X^{[j]} = F\!\left(X^{[i]}\right),\) where \(j > i,\)
when performing back-propagation we have \[\vect\!\left(\bar X^{[i]}\right)' =
\vect\!\left(\bar X^{[j]}\right)' \partialfrac{X^{[j]}}{X^{[i]}} =
\partialfrac{x^{[n]}}{X^{[j]}} \partialfrac{X^{[j]}}{X^{[i]}} =
\partialfrac{x^{[n]}}{X^{[i]}}.\]

This is equivalent to doing the derivative in this way
\[\vect\!\left(\bar X^{[i]}\right)'
= \mathrm{vJp}_F\!\left(X^{[i]},X^{[j]}\right)
= \partialfrac{\left\langle\bar X^{[j]},X^{[j]}\right\rangle}{\vect\!\left(X^{[i]}\right)'}
= \partialfrac{\left\langle\bar X^{[j]},F\left(X^{[i]}\right)\right\rangle}{\vect\left(X{[i]}\right)'}.\]
As we are going to see in the next sections
this formulation, involving the vector-Jacobian product is much more useful.

\subsection{Matrix Multiplication}

Now let us set \(F(X) = X B,\) dropping the index for simplicity, let us also
call \(\bar X^{[j]} = G.\) We now have that if we want to find the vJp for
matrix multiplication we  have to find the derivative of \[\partialfrac{\langle
G, X B \rangle}{\vect(X)'}.\]

We can proceed using the differential calculus we have developed and then
applying the appropriate identification theorem:
\begin{eqnarray*}
\dif\langle G, X B \rangle
&=& \dif\trace(G' (X B)) \\
&=& \trace(\dif(G'(X B))) \\
&=& \trace(\dif G' X B + G' \dif(X B)) \\
&=& \trace(G' \dif( XB)) \\
&=& \trace(G'(\dif X\,B + X\dif B)) \\
&=& \trace(G' \dif X\,B) \\
&=& \trace(B G' \dif X) \\
&=& \trace((G B')' \dif X)) \\
&=& \vect(G B')' \vect(\dif X).
\end{eqnarray*}
Now with the identification theorem we get that the derivative is \(\vect(G B')'.\)

Let us note that this expression is in function of two ``small'' dense matrices
\(G\) and \(B.\) In this way we completely avoid the need to instantiate the
sparse Jacobian of \(X B.\)

We can do a similar thing for the function that left multiplies a matrix \(F(X) = A X.\)
\begin{eqnarray*}
\dif\langle G,A X\rangle
&=& \dif\trace(G' A X) \\
&=& \trace(G' A \dif X) \\
&=& \trace((A' G)' \dif X).
\end{eqnarray*}
With the identification theorem we get that the derivative is  \(\vect(A' G)'.\)

\subsection{Element-wise functions}

One of the most used family of functions in neural networks is the one of
element-wise functions i.e. functions that are applied to each element of a
matrix. Many activation functions belong to this family, like ReLU and Sigmoid.
It is therefore essential to learn how to back-propagate the gradient through
this family of functions.

First let us note that for an element-wise function \(F\) we have that its
differential is \(\dif F(X) = \tilde F(X) \hadam \dif X,\) where \(\tilde F(X)\)
is the derivative of the function applied to each element.\footnote{E.g the
function applied to each element is \(\sin\) then the derivative is \(\cos\)
applied to each element.} The reason is quickly explained if we think about what
the differential does to each element of the matrix, that is, if we restrict
ourselves to the scalar case we have \(\dif f(x) = \der f(x) \dif x.\)

With the same setup as before we now have
\begin{eqnarray*}
\dif\langle G, F(X) \rangle
&=& \dif\trace(G' F(X)) \\
&=& \trace(\dif(G' F(X))) \\
&=& \trace(\dif G' F(X) + G' \dif F(X)) \\
&=& \trace(G' \dif F(X)) \\
&=& \trace(G' (\tilde F(X) \hadam \dif X)) \\
&=& \vect(G)' \vect(\tilde F(X) \hadam \dif X) \\
&=& \vect(G)' \diag(\vect(\tilde F(X))) \vect(\dif X) \\
&=& \vect(G \hadam \tilde F(X))' \vect(\dif X).
\end{eqnarray*}
Using the usual identification theorem we get that the vJp for element-wise
functions is \(\vect(G \hadam \tilde F(X))'\)

\subsection{Broadcasting}\label{sec:broadcasting}

% http://coldattic.info/post/116/+
% https://numpy.org/doc/stable/user/basics.broadcasting.html#:~:text=outer%20product+
% https://lesia.obspm.fr/perso/thibaut-paumard/yorick-doc/manual/yorick_50.html+
% https://stackoverflow.com/a/26950256+

Broadcasting is a convenient and efficient way to implement outer products. It
first appeared in Yorick~\cite{munro1995} and was then made popular by
NumPy~\cite{harris2020}.

When performing training on batches it is common, in the most common deep
learning frameworks to write operations expression of this kind \(S(AX + \vec
b)\) or \(S(AX + b)\). These two expressions have no well defined meaning in the
standard mathematical language but in these frameworks usually mean add \(\vec
b\) to every row of \(AX\), for the first case, and add \(b\) to every element
of \(AX\) in the second. This behavior has the name of broadcasting and it is
done to improve performance and reduce memory usage at the same time (since both
kind of operations can be done without instantiating the matrices explicitly),
therefore are worth implementing.

This raises a question, how do we back-propagate through this kind of
operations? Luckily the answer is easy. We can mathematically fix this
expressions by adding a few multiplications with constants that keep the result
identical and allow us to use the formalism that we have developed until now.
We can therefore rewrite the two expressions in the following way
\begin{eqnarray*}
    S(AX + \vec b) &\Rightarrow& S(AX + \vec b \vec 1'), \\
    S(AX + b)      &\Rightarrow& S(AX + b \cdot \vec 1 \vec 1').
\end{eqnarray*}

We are now going to derive the back-propagation formula for the broadcasting
with a scalar. To simplify a bit our calculations later let  \(F(b) = b \cdot \vec
1 \vec 1'.\), then
\begin{eqnarray*}
    \dif F(b)
    &=& \dif(b \cdot \vec 1 \vec 1') \\
    &=& \dif(b \cdot \vec 1) \vec 1' + b \vec 1 \dif \vec 1' \\
    &=& \dif(b \cdot \vec 1) \vec 1' \\
    &=& (\dif(b) \cdot \vec 1 + b \dif \vec 1) \vec 1' \\
    &=& \dif b \cdot \vec 1  \vec 1' \\
    &=& \vec 1 \vec 1' \cdot \dif b.
\end{eqnarray*}

Now we can proceed with the back-propagation through the broadcast of the scalar
\begin{eqnarray*}
    \dif\langle G,F(b)\rangle
    &=& \dif\trace(G' F(b)) \\
    &=& \trace(\dif(G' F(b))) \\
    &=& \trace(\dif G' F(b) + G' \dif F(b)) \\
    &=& \trace(G' \dif F(b))) \\
    &=& \trace(G' \vec 1 \vec 1' \cdot \dif b)) \\
    &=& \trace(G' \vec 1 \vec 1') \cdot \dif b \\
    &=& \trace(\vec 1' G' \vec 1) \cdot \dif b \\
    &=& \vec 1' G' \vec 1 \cdot \dif b.
\end{eqnarray*}
That is we just have to sum all the elements of \(G\). When we broadcast \(b\)
the result is similar. In general we just have to sum the contributions along
the broadcast dimensions.

\subsection{Dropout}

The dropout layer~\cite{srivastava2014} is commonly used for regularization
technique to avoid overfitting. It is usually described as ``randomly drop units
(along with their connections) from the neural network during training'' and it
is usually implemented as an Hadamard multiplication with a random binary vector
\(m\) that acts as a mask for the dropped units.

Therefore for the vector case we can write \[\vec x \hadam \vec m,\] while for
the matrix case we can either promote \(\vec m\) to a random binary matrix \(M\)
\[X \hadam M\] or rely on broadcasting \[X \hadam \vec m \vec 1'.\]

In either case is easy to derive expressions to back-propagate through this
operations \(\dif\trace(G' (M \hadam X)) = \trace(G' (M \hadam \dif X)) =
\trace((G' \hadam M') \dif X) = \trace((G \hadam M)' \dif X)\).

\subsection{Attention}

The attention is the core operation behind the transformer
architecture~\cite{vaswani2017}. It operates on ``text'' represented as a
sequence of vector embeddings in a matrix, say \(X\). It has three parameters to
``learn'', \(W^{[q]},\) \(W^{[k]},\) and \(W^{[v]}.\)

% https://en.wikipedia.org/wiki/Attention_(machine_learning)#Core_calculations
Let \(Q = X W^{[q]},\) \(K = X W^{[k]},\) \(V = X W^{[v]}\). We can define the
attention as \[V' S\left(\frac1{\sqrt{d}} \cdot K Q'\right),\] where \(d\) is
the number of rows in \(X.\)

If we break it down into its elementary operations we obtain
\begin{eqnarray*}
    X^{[1]} &= & (X W^{[q]})'\\
    X^{[2]} &= & X W^{[k]}\\
    X^{[3]} &= & X^{[2]} X^{[1]}\\
    X^{[4]} &= & 1/\sqrt d \cdot X^{[3]}\\
    X^{[5]} &= & S(X^{[4]})\\
    X^{[6]} &= & (X W^{[v]})'\\
    X^{[7]} &= & X^{[6]} X^{[5]}.
\end{eqnarray*}
In order to backpropagate through it we need to know how to propagate through
scalar multiplication, transposed matrix multiplication and the softmax
function. We are going to derive the latter, the other two are left as an easy
exercise for the reader.

Let \(A = \vec 1 \vec 1' e^{{\cdot}X}\)
\begin{eqnarray*}
\dif\langle G,S(X)\rangle
&=& \trace\left(G' \dif \left(e^{{\cdot}X} \hadam A^{{\cdot}-1}\right)\right)\\
&=& \trace\left(G' \left(
    \dif e^{{\cdot}X} \hadam A^{{\cdot}-1}
    + e^{{\cdot}X} \hadam \dif A^{{\cdot}-1}\right)\right)\\
&=& \trace\left(G' \left(
    e^{{\cdot}X} \hadam \dif X \hadam A^{{\cdot}-1}
    + e^{{\cdot}X} \hadam -A^{{\cdot}-2} \hadam \dif A\right)\right)\\
&=& \trace\left(G' \left(e^{{\cdot}X} \hadam A^{{\cdot}-1}
    \hadam \left(\dif X - A^{{\cdot}-1} \hadam \dif A\right)\right)\right)\\
&=& \trace\left(G'
    \left(S(X) \hadam \left(\dif X - A^{{\cdot}-1} \hadam \dif A\right)\right)\right)\\
&=& \trace\left((G' \hadam S(X)')
    \left(\dif X - A^{{\cdot}-1} \hadam \dif A\right)\right)\\
&=& \trace((G \hadam S(X))' \dif X)
    - \trace\left((G \hadam S(X))' \left(A^{{\cdot}-1} \hadam \dif A\right)\right)
\end{eqnarray*}
Now let \(B = G \hadam S(X),\) so that \(\dif\langle G,S(X)\rangle = \trace(B' \dif X) - \trace\left(B' \left(A^{{\cdot}-1} \hadam \dif A\right)\right),\) then
\begin{eqnarray*}
\dif\langle G,S(X)\rangle
&=& \trace(B' \dif X) - \trace\left(\left(B' \hadam \left(A^{{\cdot}-1}\right)'\right) \dif A\right)\\
&=& \trace(B' \dif X) - \trace\left(
    \left(B \hadam A^{{\cdot}-1}\right)' \vec 1 \vec 1'
    \left(e^{{\cdot}X} \hadam \dif X\right)
    \right)\\
&=& \trace(B' \dif X)
    - \trace\left(\left(\vec 1 \vec 1' \left(B \hadam A^{{\cdot}-1}\right)\right)'
    \left(e^{{\cdot}X} \hadam \dif X\right)\right)\\
&=& \trace(B' \dif X)
    - \trace\left(\left(\left(\vec 1 \vec 1' \left(B \hadam A^{{\cdot}-1}\right)\right)' \hadam \left(e^{{\cdot}X}\right)'\right)
    \dif X\right)\\
&=& \trace(B' \dif X)
    - \trace\left(\left(\vec 1 \vec 1' \left(B \hadam A^{{\cdot}-1}\right) \hadam e^{{\cdot}X}\right)'
    \dif X\right)\\
&=& \trace\left(
    \left(
    B' - \left(\vec 1 \vec 1' \left(B \hadam A^{{\cdot}-1}\right) \hadam e^{{\cdot}X}\right)'
    \right)\dif X\right)\\
&=& \trace\left(
    \left(
    B - e^{{\cdot}X} \hadam \vec 1 \vec 1' \left(B \hadam A^{{\cdot}-1}\right)
    \right)' \dif X\right)\\
&=& \trace\left(
    \left(
    B - e^{{\cdot}X} \Diag\left(\left(\vec 1' \left(B \hadam A^{{\cdot}-1}\right)\right)'\right)
    \right)' \dif X\right).
\end{eqnarray*}

Now thanks to the identification theorem we get the formula that we need to
propagate the gradient through the softmax function.

\part{Practice}

\chapter{Metal}

There are three main ways to program modern GPUs. One is via some high-level
library like MAGMA~\cite{tomov2010}, where you call functions that internally deal
with GPU interaction and kernel implementation. Another possibility is to use a
language like CUDA~\cite{nickolls2008} where you are in charge of the kernels and you
have to interact more with the GPU. Finally, you can use APIs like
OpenCL~\cite{stone2010} where, together with being in charge of the kernels, you
have to control all interactions with the GPU.

Since our main machine is a MacBook, to make use of our GPU we had to go the
Apple's way.

Modern Apple systems have a custom API/language called Metal gives you low
level control much like OpenCL (or the CUDA Driver API).

Apple's Metal has two components. The first is the ``Metal Shading Language''
(MSL), which is the high-level language that is then compiled to the GPU's
ISA\footnote{Often it is compiled to an intermediate binary format that is
shipped with the application and at run time that binary is compiled to the ISA
of the GPU available at that time on the machine.}, the second is the ``Metal
Framework''\footnote{Framework is a kind of library on macOS.} which is an
Objective-C library that allows you to communicate with the GPU.

We are now going to describe both using an example, shown in
figure~\ref{fig:metal1} and~\ref{fig:metal2}. You can compile the code
with the following commands
\begin{verbatim}
$ clang traceme.m -framework Metal -framework CoreGraphics \
    -framework Foundation -g -o traceme
$ xcrun -sdk macosx metal simple.metal
\end{verbatim}

The first command is just a normal compiler invocation even if strangely to make
the program work you also have to link with the CoreGraphics Framework (the
reason may be that Metal is a unified compute and graphics API.). The second
compiles the metal code to an intermediate representation and puts it in a file
called \texttt{default.metallib}. It is important for this file to be kept in
the same directory that contains the \texttt{traceme} binary, this is a quirk of
how \texttt{[NSBundle mainBundle]} behaves with command line application.

The binary can then run with \texttt{sudo dtruss -s traceme} to get a peek of
what happens behind the curtains. You will see that it is quite a lot.

\begin{figure}
\centering
\begin{minipage}{\textwidth}
\footnotesize
\begin{verbatim}
#import <Metal/Metal.h>

int main(void) {
    uint len = 1023;
    id<MTLDevice> dev = MTLCreateSystemDefaultDevice();
    // Code loading.
    id<MTLLibrary> lib = [dev newDefaultLibrary];
    id<MTLFunction> fun = [lib newFunctionWithName:@"simple_function"];
    id<MTLComputePipelineState> cps =
        [dev newComputePipelineStateWithFunction:fun
                                           error:nil];
    // Data loading.
    id<MTLBuffer> buf = [dev newBufferWithLength:len
                                         options:MTLResourceStorageModeShared];
    for (uint i = 0; i < len;  i++) ((float *)buf.contents)[i] = 1.f;
    // Work submission.
    id<MTLCommandQueue> cmdQueue = [dev newCommandQueue];
    id<MTLCommandBuffer> cmdBuf = [cmdQueue commandBuffer];
        id<MTLComputeCommandEncoder> cCmdEnc = [cmdBuf computeCommandEncoder];
        [cCmdEnc setComputePipelineState:cps];
        [cCmdEnc setBuffer:buf offset:0 atIndex:0];
        [cCmdEnc setBytes:&(float){1.0f} length:sizeof 1.0f atIndex:1];
        [cCmdEnc dispatchThreads:MTLSizeMake(len,1,1)
           threadsPerThreadgroup:MTLSizeMake(cps.threadExecutionWidth,1,1)];
        [cCmdEnc endEncoding];
    [cmdBuf commit];
    [cmdBuf waitUntilCompleted];
    // Result.
    float *res = buf.contents;
    return 0;
}
\end{verbatim}
\end{minipage}
\caption{Content of the file \texttt{traceme.m}}
\label{fig:metal1}
\end{figure}

\begin{figure}
\centering
\begin{minipage}{\textwidth}
\footnotesize
\begin{verbatim}
[[kernel]] void
simple_function(
        device   float *x [[buffer(0)]],
        constant float *a [[buffer(1)]],
                 uint   i [[thread_position_in_grid]]
    ) {
    x[i] += 1.0f;
    x[i] *= *a;
}
\end{verbatim}
\end{minipage}
\caption{Content of the file \texttt{simple.metal}}
\label{fig:metal2}
\end{figure}

\section{Metal.Framework}

Let us start by describing how to use Metal.Framework to interact with the GPU.

The first thing that we have to do is to obtain a handle to a GPU device. Since
this is an Objective-C Framework this ``handle'' and pretty much everything else
in this API is an object. We can obtain the default GPU by calling the function
\texttt{MTLCreateSystemDefaultDevice()}.

In order to perform calculations on the GPU we need to load code and data on it
and tell it to perform the actual calculations. This is done via Library, Buffer
and CommandQueue objects.

The Library object is used to load the code from a file and compilation, then from
it you can obtain a Function object that gives you the possibility to set some
constants in the function. From the Function object we can obtain a
PipelaneState object, which as far as we understand the creation of this objects
load the compiled kernel code on the GPU memory.

To allocate the data that we need to process we use Buffer objects, which are a
simple object that when created performs a \texttt{malloc} like function on the
GPU.

Finally, to submit work to the GPU we use a CommandQueue object in which we
enqueue CommandBuffer in which multiple commands are encoded using command
CommandEncoder objects.

As you may have noticed we do not use explicit methods to move data to and from
the GPU memory. While there are ways to force this behavior the library prefers
to perform these actions inside other methods in an effort to simplify the usage
of the API.

When encoding commands we specify, the function to call via the compute pipeline
state, the arguments by passing references to buffers and copying miscellaneous
arguments directly to GPU memory, finally we specify the number of threads
needed and the preferred group size. This allows to run the threads of the GPU
in groups of non-uniform size avoiding the need to guard in the kernel for out
of bound memory accesses.

In figure~\ref{fig:metal.framework} we show a diagram that depicts how the
various types in this API interact with each other.

\begin{figure}
\centering
\includegraphics{figures.3}
\caption{A graph showing the main types needed for sending work to an Apple GPU,
with the \textsf{MTL} prefix removed for brevity.}
\label{fig:metal.framework}
\end{figure}

The API is huge it does not only contains features related to GPGPU but also to
graphics, here we have shown the minimal amount of information needed to
understand its general usage pattern.

\section{Metal Shading Language}

The MSL is based on C++14, with some features missing like \texttt{new} and
\texttt{delete} since on GPU you have to coordinate with the CPU to manage
memory, it also adds some features and some restrictions (like \emph{address
spaces}). These changes are needed to take advantage of all the peculiarities of
GPU architectures.

GPUs execute programs called either shaders or kernels (depending if we are
talking about graphics or compute). Compute programs on GPUs are written as MSL
functions (appropriately marked as \texttt{[[kernel]]}) and are executed in a
parallel fashion. If you look at figure~\ref{fig:metal2} the index \texttt{i} is
like the one used in loops. In fact GPU kernels are executed on a GPU like if
they were in a big parallel for.

The function takes two arguments, we are going to see later that the third is
not really a function argument. The first is a pointer to a \texttt{float} with
the \texttt{device} address space. This means that it will be placed in the
GPU's global memory. The optional attribute \texttt{[[buffer(0)]]} is used to
make explicit that the first call to \texttt{setBuffer} or \texttt{setBytes} is
going to be assigned to that argument. The second argument has the
\texttt{constant} address space that is always used for data located in the
GPU's global memory but that can be cached in special GPU caches. Finally, the
variable \texttt i is calculated automatically by the MSL (this is why it is not
really an argument) and, as the name implies, contains the thread position in
the grid. In our case each element in the input vector \texttt x is updated by a
thread, therefore the position of the thread in the grid is just an index inside
the the vector that said thread has to update.

In table~\ref{tab:metal2cuda} how to translate between CUDA and Metal. By
looking at that table we note that \verb+[[thread_position_in_grid]]+ is just
the familiar expression \verb|blockDim.x*blockIdx.x + threadIdx.x| used in CUDA
(for the one-dimensional case).

\begin{table}
\centering
\begin{tabular}{ll}
    Metal & CUDA \\
    \multicolumn{2}{c}{Terminology} \\
    thread       & thread \\
    SIMD group   & thread warp \\
    thread group & thread block \\
    thread grid  & thread grid \\
    % \hline
    \multicolumn{2}{c}{Indexing} \\
    % \hline
    \verb+thread_position_in_threadgroup+ & \texttt{threadIdx} \\
    \verb+threads_per_threadgroup+        & \texttt{blockDim} \\
    \verb+threadgroup_position_in_grid+   & \texttt{blockIdx} \\
    \verb+threads_per_grid+               & \texttt{gridDim} \\
    \verb+threads_per_simdgroup+          & \texttt{warpSize} \\
    % \hline
    \multicolumn{2}{c}{Address Spaces} \\
    % \hline
    \texttt{device}      & \verb+__device__+ \\
    \texttt{threadgroup} & \verb+__shared__+ \\
    \texttt{constant}    & \verb+__constant__+ \\
    \multicolumn{2}{c}{Synchronization} \\
    \verb+threadgroup_barrier(metal::mem_flags)+ & \verb+__syncthreads()+ \\
    \verb+simdgroup_barrier(metal::mem_flags)+   & \verb+__syncwarp()+
\end{tabular}
\caption{The terminology section shows the terms used to represent the same
concepts. The indexing section shows MSL function input attributes and the
equivalent global variables for CUDA. The address spaces section shows which
keywords have to be used in MSL and CUDA to place object in certain portions of
memory/caches. The synchronization section shows the functions to call to
synchronize threads at the right granularity.}
\label{tab:metal2cuda}
\end{table}

\iffalse
\subsection{Peculiarities and Shortcomings}

With our great surprise we have discovered that in the MSL the function
\texttt{pow(x, y)}, when the compiler option \texttt{-ffast-math} is present
(which it is by default), is rewritten as \texttt{exp2(y * log2(x))}. This is
done because GPUs implement this two functions as single instruction, hence
making the implementation more efficient than a classic one one involving the
Taylor approximation of $e^x$, sadly this gives a \texttt{NaN} when \texttt x is
negative. This problem was solved by disabling \texttt{-ffast-math} with
\texttt{-fno-fast-math}. When don't understand why this is considered an
acceptable transformation, maybe in computer graphics is really uncommon take
the exponential of negative value.

Another questionable thing happened with the Metal Performance Shaders (MPS)
library and the their implementation of GEMM. To make a long story short we have
implemented a specialized\footnote{In the sense that it does not support leading
dimensions or implicit transpositions.} GEMM kernel that for certain dimensions
is faster than the one provided by Apple. This can happen with performance
focused implementations, the fact that makes this ``problematic'' is that we
just tiled and coarsened the kernel (which are standard textbook techniques for
improving the performance of GPU kernels) without going into any more advanced
shenanigans or even doing any optimization specific to the architecture. The
last two are things that one would expect Apple's engineers to do, in particular
because they should have access to more advanced internal tools for GPU
profiling etc\dots

TODO: display the kernel that I'm talking about and show how the profiling was done.
\fi

\chapter{Implementation}

Once we have derived all the formulas we need and we have understood how to
communicate with the GPU we have all the instruments we need to implement our
simple back-propagation library.

The idea behind the implementation is simple. For each elementary operation we
support we implement a GPU kernel for it. Then we provide a function (in C) that
records the arguments passed to it as dependencies and allocates a buffer on the
GPU of the appropriate size to hold the result and another one for the gradient.

After having recorded the graph of the expression that calculates the loss for a
neural network, we can initialize its parameters and inputs. Then we can proceed
with the training. We can enqueue on the GPU commands to calculate the loss,
propagate the gradient backward and finally update the parameters. These
steps (together with updating the training data) can be performed in a loop to
train the neural network.

In figure~\ref{fig:sample-code} we show how to implement a single linear layer
with sigmoid activation with a mean squared error loss. and overfit a single
batch of data.

\begin{figure}
\begin{footnotesize}
\begin{verbatim}
// linear layer with sigmoid activation.
Mat W = Mat_new((MShape){4,3}, true);
Mat X = Mat_new((MShape){3,4}, false);
Mat B = Mat_new((MShape){4,4}, true);
Mat T = Mat_new((MShape){4,4}, false);
Mat E = Mat_new(scalar_shape, false);
Mat Y = Mat_sig(Mat_add(Mat_dot(W, X), B)); 

// mean squared error loss
Mat nT = Mat_neg(T);
Mat YpnT = Mat_add(Y, nT);
Mat YpnTe2 = Mat_pow(YpnT, E);
Mat l = Mat_sum(YpnTe2);
assert(l != 0);
make_shared(l);

for (int i = 0; i < 4*3; i++) {
    ((float *)tape[W].fwd.contents)[i] = 1;
    ((float *)tape[X].fwd.contents)[i] = 1;
}
for (int i = 0; i < 4*4; i++) {
    ((float *)tape[B].fwd.contents)[i] = .5;
    ((float *)tape[T].fwd.contents)[i] = 1;
}
*(float *)tape[E].fwd.contents = 2;

// our objective for now is to overfit a batch (i.e. X, T).
for (int i = 0; i < 10; i++) {
    enqueue_forward(cmdQueue);
    float loss = *(float *)tape[l].fwd.contents;
    NSLog(@"%f", loss);
    if (isnan(loss)) {
        printMat(cmdQueue, E);
        NSLog(@"");
        printMat(cmdQueue, YpnT);
        NSLog(@"");
        printMat(cmdQueue, YpnTe2);
        break;
    }
    enqueue_backward(cmdQueue);
    enqueue_update_params(cmdQueue, -0.005);
    enqueue_zero_grad(cmdQueue);
}
\end{verbatim}
\end{footnotesize}
\caption{Sample usage of the back-propagation library.}
\label{fig:sample-code}
\end{figure}

\chapter*{Conclusions}

The purpose of this work is to present the back-propagation algorithm in matrix
form, providing the necessary tools to derive gradient propagation formulas and
show how they can be used to implement a simple library to train neural
networks.

We have shown how to derive all the necessary results (i.e., matrix
multiplication and element-based functions) to implement a back-propagation
library, along with some auxiliary results (broadcasting and dropout) that are
commonly used in deep neural networks.  Finally, we also showed how one can
derive a vJp for the non-elementary operation \emph{softmax}. In this way, we
have shown that with the right mathematical tools available, it is relatively
easy to obtain these results.

This is the first self-contained resource that allows any computer scientist,
with a standard mathematical background, to derive back-propagation formulas for
his or her operations. Until now, this was a major obstacle and anyone who had
to implement such code needed either the help of a mathematician or laboriously
and painfully to learn on its own.  With this work, we believe we have overcome
this difficulty due to the organization and vastness of the literature, allowing
anyone to obtain their own results more easily.

We also have implemented a na\"ive back-propagation library that runs all of its
operations on the GPU interfacing directly with the Metal API from Apple, and
writing our own kernels. This shows how easy it can be to implement your own
algorithm to train neural networks if one can derive the back-propagation
formulas.

Possible future work includes making derivations even easier by using direct
sums and block matrices to represent operations involving batch matrix
multiplications, for example multi-head attention (look at
appendix~\ref{sec:direct-sums} for more context).

Also on the implementation side, possible future directions include the
possibility of generating GPU kernels on-demand via kernel fusion approaches. This
allows not only to reduce training times by generating more efficient
operations, but also to spare precious GPU RAM. The reason is quickly explained
by an example. Let \(F\) and \(G\) be two element-wise functions and let \(H = G
\circ F\) be their composition, then when performing the back-propagation
through the two functions we have to store \(\bar F\) first to then calculate
\(\bar G\), if we implement operator fusion instead, we only have to store
\(\bar H.\)

In the general deep learning literature one can find all sorts of architectures,
we have explicitly focused on static DAGs, that is to say, graphs without
control flow and fixed at the start, this is for simplicity but also due to the
fact that the most common and successful architectures (multi-layer perceptrons,
convolutional neural networks and transformers) are all static. We have also
limited ourselves to the exclusive treatment of differentiable functions, always
for the sake of simplicity, but the use of non-differentiable functions (think
ReLU or max-pool) is ubiquitous in modern architectures. Luckily it is possible to
generalize the concept of gradient to subgradient and include this kind of
functions too in our framework.

\iffalse
Finally, since in deep learning it is common to use non-differentiable functions
such as ReLU or max-pool, taking advantage of the approach used in this work,
obtaining these functions will not be a problem since it is possible to
generalize the concept of gradient to subgradient.
\fi

\appendix

\iffalse
\chapter{Why We Need Gradients in Machine Learning}

% This infrocution is mostly taken from `Machine Learning: The Basics' book.
% There is to correct/expicitate what is a random variable and what is a
% realization of it. In particular I should use $p(x,y)$ to indicate the
% distribution, not $\mathcal D$ which shuld be used to indicate the dataset.

\cite{jung2022, goodfellow2016, shalev-shwartz2014}.

As we have seen in section~\ref{sec:notable_results} machine learning algorithms
can be used form many kind of tasks. We are now going to give a formal
introduction to the field by restricting ourselves to the classification
case\footnote{Since this work concerns itself with the derivation and
implementation of the back-propagation this restriction does not really matter.
It is just used as a simplification to introduce the key concepts that we are
going to need.}.

To formalize the notion of machine learning we are going to treat our data as a
collection of data points $\mathbf x \in \mathcal X$. To each of those data
points we are going to associate a label $y \in \mathcal Y$. Our objective,
through the training algorithm, is to find a certain hypothesis $h : \mathcal X
\to \mathcal Y$, in the hypothesis space $\mathcal H$, that dives the best
prediction, as measured by $\ell : \mathcal X \times \mathcal Y \times \mathcal
H \to \mathbf R$, the loss function. We are going to assume that all of our data
point are the realization of the two i.i.d. random variables $X$ and $Y$ that
come from the (unknown) joint probability distribution $\mathcal
D$\footnote{This assumption would not be good for time series data.}.

We can now define the risk of a hypothesis $h \in \mathcal H$ as
\begin{displaymath}
	\expected_{\mathbf x, y}[\ell(\mathbf x, y, h)] \mathrm .
\end{displaymath}
Therefore the hypothesis that gives the best prediction is
\begin{displaymath}
	h^* = \argmin_{h \in \mathcal H} \expected_{\mathbf x, y}[\ell(\mathbf x, y, h)] \mathrm .
\end{displaymath}
Sadly, as we have said before, the true distribution is unknown. Therefore even
with huge amount of computational power ant time at our disposal we can't find
the true $h^*$, we cannot minimize the risk.

We have at our disposal the data $\mathcal D = \{(\mathbf x^{(1)}, y^{(1)}),
\ldots, (\mathbf x^{(m)}, y^{(m)})\}$. Since we have assumed that our random
variables are i.i.d. we can use the law of large numbers\footnote{In brief it
states that the expected value (or average) of random variable, if it exists, it
can be obtained from a large enough independent and identical samples.} to
estimate the expected value of the loss. We call
\begin{displaymath}
	\hat h \in \argmin_{h \in \mathcal H} \frac 1 m
		\sum_{i=1}^m \ell(\mathbf x^{(i)}, y^{(i)}, h)
\end{displaymath}
empirical risk minimization\footnote{We use set ownership notation because there
might be more than one hypothesis that minimize the loss over the data.}. Now
since we are interested in deep neural networks our parameter space is going to
be parametrized by $\mathbf \theta$, the weights of the network\footnote{as with
many things in math this is an abuse of notation. $\mathbf \theta$ is not a
single vector, matrix or tensor but the entire set of weights. One can imagine
this being a single vector that is sliced and reshaped accordingly to form each
parameter of the network.}. Therefore if $f$ is the function that evaluates the
neural network (our current hypothesis) and the loss, we can rewrite our
empirical risk minimization as follows:
\begin{equation}
	\hat h \in \argmin_{\mathbf \theta \in \mathbf R^n} \frac 1 m
		\sum_{i=1}^m f(\mathbf x^{(i)}, y^{(i)}; \theta) \mathrm .
		\label{eq:erm}
\end{equation}

Now equation~\ref{eq:erm} is a classical \emph{unconstrained} optimization
problem. Also since $f$ is a neural network, we have access to it's
gradient\footnote{In practice for a function being differentiable is not enough
to make sure that optimization algorithms that use the gradient works well.
There are certain properties like being Lipschitz continuous which are also
necessary. But in general neural networks are functions whit have such
properties and it is beyond the scope of this work to talk about this
conditions.

Also due to the common use of non differentiable function like ReLU we need the
more general concept of a Clarke subdifferential\dots But we since we wanted a
degree we are going to ignore this kinds of functions and restrict ourselves to
only differentiable ones.}.

% This section is taken from `An Introduction To Optimization'. A clenanup of
% the notation is also needed.

Given a point $\mathbf x$, $\nabla f(\mathbf x)$\footnote{We drop the parameter
notation for simplicity.}, if different from the zero vector, is a vector
orthogonal to the tangent hyper-plane passing per $\mathbf x$. $-\nabla f
(\mathbf x)$ gives us the direction with the maximum rate of decrease of $f$ at
$x$, which is a good direction to follow if we want to minimize $f$.

Let us now consider the first order Taylor expansion of a function $f:\mathbf
R^n \to \mathbf R$ ($\mathbf a$ and $\mathbf b$ need to be near each other)

\begin{displaymath}
	f(\mathbf b) \approx f(\mathbf a) + \nabla f(\mathbf a)(\mathbf b - \mathbf a) \mathrm .
\end{displaymath}

Let us now set $\mathbf a = \mathbf x^{(0)}$ and $\mathbf b = \mathbf x^{(0)} -
\alpha\nabla f(x^{(0)})'$ for some $\alpha > 0$, therefore $\mathbf b - \mathbf
a = -\alpha\nabla f(x^{(0)})'$. We now get
\begin{eqnarray*}
	f(\mathbf x^{(0)} - \alpha\nabla f(x^{(0)})')
	&\approx& f(\mathbf x^{(0)}) + \nabla f(\mathbf x^{(0)})(-\alpha\nabla f(\mathbf x^{(0)})') \\
	&=& f(\mathbf x^{(0)}) - \alpha\nabla f(\mathbf x^{(0)})^2 \mathrm .
\end{eqnarray*}

Now since $\alpha\nabla f(\mathbf x^{(0)})^2 > 0$, we get that for a
sufficiently small $\alpha$
\begin{displaymath}
	f(\mathbf x^{(0)} - \alpha\nabla f(x^{(0)})') < f(\mathbf x^{(0)}) \mathrm ,
\end{displaymath}

This means that, if we are trying to minimize $f$, $\mathbf x^{(0)} -
\alpha\nabla f(x^{(0)})'$ is an improvement over the point $\mathbf x^{(0)}$.

Our assumptions require $\mathbf a$ to be near $\mathbf b$ and $\alpha$ to be
small. Therefore if we want to minimize $f$ we can iterate this procedure as
such:
\begin{displaymath}
	\mathbf x^{(k+1)} \leftarrow x^{(k)} - \alpha\nabla f(\mathbf x^{(k)})' \mathrm .
\end{displaymath}

This is a so called \emph{gradient descent} algorithm because of the continuous
decrease of the value of $f$.

Finally if we go back to equation~\ref{eq:erm}, lets now call the mean inside
$\argmin$ $F$. If we plug $F$ in the gradient descent formula we get by the
linearity of the gradient operator that we have to sum the evaluate the gradient
of $f$ over all the data.

The algorithm by itself is trivial if not for piece that sums the gradient over
the data. This piece of the algorithm is going to be what the rest of this
work is about.

\iffalse %{
We could simplify the introduction by assuming single vector arguments and
convex optimization to then discharge those assumption later. This would allow
us to talk about this challenges in the generalization of this concepts.

We have to explicit that we don't care about over-fitting or regularization and
that stochastic approximation via randomly sampled minibatch are actually used.

Right now we ignore the local minima problem since we are not in the convex
case.

I have to use Cauchy-Schwarz to prove that negative gradient is always the
steepest descent direction.
\fi %}
\fi

\chapter{Slicing}\label{sec:slicing}

\newcommand\slice{\mathrm{Slice}}
\newcommand\sss[2]{#2{s}{#1}#2{t}{#1}#2{p}}

It is common in programming languages to access the elements of an array in
various ways. The most common are by index and by \emph{slicing}. The latter is
a primitive form of down-sampling in which starting at and element \(s\) we take
elements \(t\) step apart until we reach the stop element at index \(p.\) The
notation used is usually \(x_{\sss:{}}.\)\footnote{\(s\)tart, s\(t\)op,
ste\(p.\)} In matrices this notations can be combined between each other as
shown in table~\ref{tab:slice}.

\begin{table}
    \newcommand{\p}{\tilde}
    \centering
    \begin{tabular}{cccc}
                                   & \(i\)       & :           & \(\sss:{}\) \\[2pt]
                                   \cline{2-4}
        \multicolumn{1}{r|}{\(j\)} & \(A_{i,j}\) & \(A_{:,j}\) & \(A_{\sss:{},j}\) \\
        \multicolumn{1}{r|}{:}     & \(A_{i,:}\) & \(A_{:,:}\) & \(A_{\sss:{},:}\) \\
        \multicolumn{1}{r|}{\(\sss:\p\)} & \(A_{i,\sss:{\p}}\) & \(A_{:,\sss:\p}\) & \(A_{\sss:{},\sss:\p}\)
    \end{tabular}
    \caption{All possible combinations of indexing notations for a matrix.}
    \label{tab:slice}
\end{table}

Slicing of a matrix \(A\) can be implemented as the multiplication with a
binary matrix, generated by a \(\slice(\sss,{})\) function. Therefore we can say
that \(A_{\sss:{},\sss:\tilde}\) is just a shorthand for \[\slice(\sss,{}) \, A
\, \slice(\sss,\tilde)'.\]

To give a more concrete example let \(\vec x\) be a vector of arbitrary length, if we
want to take every 3rd element of it starting at index 2 and less than 11 we
have \[\slice(2,3,8) \, \vec x
= \left[\begin{array}{cccccccccc}
    0&1&0&0&0&0&0&0&0&\cdots\\
    0&0&0&0&1&0&0&0&0&\cdots\\
    0&0&0&0&0&0&0&1&0&\cdots
\end{array}\right] \vec x.\]

We note that for any ``slice matrix'' we have that \(\slice(\sss,{}) \leftidx{_m}{\vec 1}{} =
\leftidx{_n}{\vec 1}{},\) that is to say that each row has a single 1 in it.

The other indexing notation in table~\ref{tab:slice} can also be implemented via
matrix multiplication. Let \(A\) be an \(n \times m\) matrix and \(x\) be of
lenght \(n\), than using the standard basis vectors we have that
\(\leftidx{_n}{\vec e}{^{[i]}}\)
\begin{eqnarray*}
    A_{i,j} &=& \left(\vec e^{[j]}\right)' A \vec e^{[i]}, \\
    A_{:,j} &=& A \vec e^{[i]}, \\
    A_{i,:} &=& \left(\vec e^{[j]}\right)' A, \\
    x_i     &=& x' \vec e^{[i]}.
\end{eqnarray*}

\chapter{Direct Sums}\label{sec:direct-sums}

\newcommand{\Z}{\phantom{{}'}\vec 0 \vec 0'}

Let the direct sum operation be\footnote{The Kronecker product is sometimes
called the direct product.} \[ A \oplus B =
\left[\begin{array}{cc}A&\Z\\\Z&B\end{array}\right].\] If we represent a batch
of matrices as \(\bigoplus_{i=1}^n A^{[i]},\) (sometimes it is also written as
\(\Diag\!\left(A^{[1]},\ldots,A^{[n]}\right),\) then the batched matrix
multiplication with a matrix W is \[
\left(\bigoplus_{i=1}^n A^{[i]}\right) (\vec 1 \krone W)
= \left[\begin{array}{ccc}
    A^{[1]} & \cdots & \Z \\
    \vdots & \ddots & \vdots \\
    \Z & \cdots & A^{[n]}
\end{array}\right]
\left[\begin{array}{c}W \\ \vdots \\ W \end{array}\right]
= \left[\begin{array}{c}A^{[1]} W \\ \vdots \\ A^{[n]} W\end{array}\right].\]
At this point we can apply a normal element-wise activation function to the
entire batch of matrices.

If we need to apply bias we can similarly do \[
\left[\begin{array}{c}A^{[1]} W\\ \vdots \\ A^{[n]} W\end{array}\right]
+ (\vec 1 \krone \vec b \vec 1')
= \left[\begin{array}{c}
    A^{[1]} W + \vec b \vec 1' \\ \vdots \\ A^{[n]} W + \vec b \vec 1'
\end{array}\right].
\]

We notice that \(\vec 1 \krone W\) is just another form of broadcasting since it
involves a product with a vector of ones.

% https://math.hecker.org/2011/06/25/multiplying-block-diagonal-matrices/
Alternatively we could have broadcasted \(W\) in a block diagonal form and using
blocked diagonal matrix multiplication we could have obtained
\begin{eqnarray*}
\left(\bigoplus_{i=1}^n A^{[i]}\right) (I \krone W)
&=& \left[\begin{array}{ccc}
    A^{[1]} & \cdots & \Z \\
    \vdots & \ddots & \vdots \\
    \Z & \cdots & A^{[n]}
\end{array}\right]
\left[\begin{array}{ccc}
    W & \cdots & \Z \\
    \vdots & \ddots & \vdots \\
    \Z & \cdots & W
\end{array}\right] \\
&=& \left[\begin{array}{ccc}
    A^{[1]} W & \cdots & \Z \\
    \vdots & \ddots & \vdots \\
    \Z & \cdots & A^{[n]} W
\end{array}\right] \\
&=& \left(\bigoplus_{i=1}^n A^{[i]} W\right).
\end{eqnarray*}

% https://stackoverflow.com/questions/596216/formula-to-determine-perceived-brightness-of-rgb-color
We can also use direct sum to represent colored images. Let \(R,\) \(G\) and
\(B\) be the red, green and blue channel of an image, and let  \(\vec l =
[\begin{array}{ccc}r&g&b\end{array}]'\) be the scalars used for
the linear combination to convert the channels into luminance. Then we can
convert an image in grayscale with the following formula
\begin{eqnarray*}
(\vec 1' \krone I) (R \oplus G \oplus B) (\vec l \krone I)
&=& (\vec 1' \krone I)
\left[\begin{array}{ccc}
    R & 0 & 0 \\
    0 & G & 0 \\
    0 & 0 & B
\end{array}\right]
\left[\begin{array}{c}r \cdot I \\ g \cdot I \\ b \cdot I\end{array}\right] \\
&=& [\begin{array}{ccc}I & I & I\end{array}]
\left[\begin{array}{c}r \cdot R \\ g \cdot G \\ b \cdot B\end{array}\right] \\
&=& r \cdot R + g \cdot G + b \cdot B.
\end{eqnarray*}

Now say that you are given a block ``vector''  \(\mathbf A =
\left[\begin{array}{ccc}\left(A^{[1]}\right)' & \cdots &
\left(A^{[n]}\right)'\end{array}\right]'\) and that you want to turn it into a
block diagonal matrix, that is \(\bigoplus_{i=1}^n A^{[i]}.\) We can do this in the
following way \[\mathbf A (\vec 1' \krone I) \hadam (I \krone \vec 1 \vec 1').\]

It may be more helpful to look at its expanded form \[
\left[\begin{array}{c}A^{[1]} \\ \vdots \\ A^{[n]}\end{array}\right]
\left[\begin{array}{ccc}I & \cdots & I\end{array}\right]
\hadam
\left[\begin{array}{ccc}
    \phantom{{}'}\vec 1 \vec 1' & \cdots & \Z \\
    \vdots & \ddots & \vdots \\
    \Z & \cdots & \phantom{{}'}\vec 1 \vec 1'
\end{array}\right].
\] This is basically an outer product followed by a mask, that is very similar
to how we have defined the \(\Diag\) operator. Hence we are going to call this
the \(\mathrm{DIAG}\) operator, \(\mathrm{DIAG}(\mathbf{A}) = \bigoplus_{i=1}^n
A^{[i]}.\)

We can now express the multi-head attention, let
\(\mathbf W^{[q]} = \bigoplus_{i=1}^n \left(W^{[q]}\right)^{[i]},\)
\(\mathbf W^{[k]} = \bigoplus_{i=1}^n \left(W^{[k]}\right)^{[i]},\)
\(\mathbf W^{[v]} = \bigoplus_{i=1}^n \left(W^{[v]}\right)^{[i]},\)
\(\mathbf K = \mathbf W^{[k]} (I \krone X),\)
\(\mathbf Q = \mathbf W^{[q]} (I \krone X)\) and
\(\mathbf V = \mathbf W^{[v]} (I \krone X)\)
then \[
\mathbf V'
S\left(
    \frac1{\sqrt{d}}
    \cdot \mathbf K \mathbf Q' (\vec 1 \krone I)
\right)
W^{[o]}.
\]

\iffalse %{
\chapter{Batch Normalization}

The mean of a vector \(x\) of size \(n\) is \(\mu(x) = \frac 1 n 1_n' x\) while
its variance is \(\mu(x^{{\cdot}2}) - \mu(x)^2.\) Let \(X\) be an \(n \times m\)
then the mean of its rows is \[\tilde m(X) = \frac 1 m \cdot X 1_m,\] then, if
we let \(Y = X - \title m(X) 1_n',\) the variance along each row is \[v(X) =
\tilde m\left(X^{{\cdot}2}\right) - \tilde m(X)^{{\cdot}2} = \frac 1 n \cdot
\diag(Y Y'),\] where \(\frac 1 n \cdot Y Y'\) is the covariance matrix.

Now with the help of broadcasting we can write the batch normalization
layer~\cite{ioffe2015} as \[(X - \tilde m(X)) \hadam v(X)^{{\cdot}-\frac 1 2}
\hadam g + b,\] which can be rewritten as \[\left(X - \tilde m(X) 1_m'\right)
\hadam \left(v(X) 1_m'\right)^{{\cdot}-\frac 1 2} \hadam g 1_m' + b 1_m',\]
where \(g\) and \(b\) are the trainable parameters.

We are now ready to derive the formulas to propagate the gradient through the
layer. Since the expression is big we will proceed piece by piece.

\begin{eqnarray*}
    X_1 &=& \tilde m(X) 1_m' \\
    X_2 &=& X - X_1 \\
    X_3 &=& v(X) 1_m' = \frac 1 n \cdot \diag(X_2 X_2') 1_m'\\
    X_4 &=& X_2 \hadam X_3^{{\cdot}-\frac 1 2} \\
    X_5 &=& X_4 \hadam g 1_m' + b 1_m'
\end{eqnarray*}

\begin{eqnarray*}
\dif\left\langle G,\tilde m(X) 1_m'\right\rangle
&=& \dif\trace(G' (1/m \cdot X 1_m) 1_m') \\
&=& \dif\trace(G' (1/m \cdot X) 1_m 1_m') \\
&=& \dif\trace(1_m 1_m' G' (1/m \cdot X)) \\
&=& \trace(1_m 1_m' G' (1/m \cdot \dif X)) \\
&=& \trace((1_m 1_m')' G' (1/m \cdot \dif X)) \\
&=& \trace((G (1_m 1_m'))' (1/m \cdot \dif X)) \\
&=& \vect((G (1_m 1_m'))') 1/m \cdot \dif\vect(X) \\
&=& \vect(1/m \cdot G (1_m 1_m'))' \dif\vect(X) \\
&=& \vect(\tilde m(G) 1_m')' \dif\vect(X)
\end{eqnarray*}

The back-propagation formula for \(x_2\) is just \(G\) for the left operand and
\(-G\) for the right one.

\comment{disegnare il grafo delle dipendenze per la batch norm...}

\begin{eqnarray*}
\dif\langle G,v(X)1_m'\rangle
&=& 
\end{eqnarray*}
\comment{Provare a derivare l'espressione pezzo pezzo}

\begin{eqnarray*}
\dif\left\langle G,N(X) \hadam g 1_m' + b 1_m'\right\rangle
&=& \dif\trace(G'(N(X) \hadam g 1_m' + b 1_m')) \\
&=& \trace(G' \dif(N(X) \hadam g 1_m' + b 1_m')) \\
&=& \trace(G' \dif(N(X) \hadam g 1_m')) \\
&=& \trace(G' (N(X) \hadam \dif (g 1_m'))) \\
&=& \trace(G' (N(X) \hadam ((\dif g) 1_m'))) \\ % Usiamo la ``nuova proprietà'' qua
&=& \trace((G' \hadam N(X)') (\dif g) 1_m') \\
&=& \trace((N(X) \hadam G)' (\dif g) 1_m') \\
&=& \trace(1_m' (N(X) \hadam G)' \dif g) \\
&=& 1_m' (N(X) \hadam G)' \dif g \\
&=& ((N(X) \hadam G) 1_m)' \dif g
\end{eqnarray*}

\chapter{Misc.}

\comment{Derivare la derivata della moltiplicazione tra matrici ma ricordare il
problema che è troppo grossa\dots}

The derivative of the right matrix multiplication is
\[\partialfrac{\vect(AB)}{\vect(A)'} = B' \krone I,\] this matrix is very big
and sparse\dots

Before we can calculate the vJp for the right matrix multiplication we have to
introduce a new concept\footnote{Later we are going to see that we can make it
so that we can avoid this complication.}


If \(G\) has size \(n \times m\) then the binary matrix \(K^{(n,m)}\) is the one
such that \(K^{(n,m)} \vect(G) = \vect(G').\) For any commutation matrix \(K\)
of any size we have \(K' K = K K' = I,\) hence \(K' = K^{-1}.\)

Finally for every \(n \times m\) matrix \(A\) and every \(r \times q\) matrix
\(B\), \(K^{(r,n)}(A \krone B) K^{(m,q)} = B \krone A.\)

The identity matrix ha size \(m \times m.\) \(B\) has size \(q \times m\)

\begin{eqnarray*}
\vect(G)' (B' \krone I)
    &=& \vect(G)' K' (I \krone B') K \\
    &=& \vect(G)' K' (B \krone I)' K \\
    &=& ((B \krone I) K \vect(G) )' K \\
    &=& ((B \krone I) K K' \vect(G') )' K \\
    &=& ((B \krone I) \vect(G') )' K \\
    &=& \vect(G' B)' K \\
    &=& (K \vect(B' G))' K \\
    &=& \vect(B' G)' K' K \\
    &=& \vect(B' G)' \\
    &=&  \\
    &=& \vect(G)' (B \krone I)'  \\
    &=& \vect(G)' (B \krone I)' I \\
    &=& \vect(G)' (B \krone I)' K' K \\
    &=& (K (B \krone I) \vect(G))' K \\
    &=& (K \vect(G B'))' K \\
    &=& \vect(G B')' K' K \\
    &=& \vect(G B')' \\
\end{eqnarray*}

First, let us notice that (by looking at table~\ref{tab:identifications}), the
total derivative of a scalar function of vector argument \(\dif \phi(x) = \sum_i
\partialfrac\phi{x_i} \dif x_i = \langle a(x),\dif x\rangle,\) and for matrix
argument we have \(\dif\phi(X) = \langle A(X),\dif X\rangle.\)

\comment{Devo esplicitare il bene la precedenza e come uso le parentesi con
\(\dif\) e \(\der\)}

If we let \(g' = \nabla\phi(x)\) we have \[\der\phi(x) = (\nabla\phi(x)) x =
\langle g', x \rangle.\] Similarly for matrices we have that if \(G' =
\vect^{-1}(\nabla\phi(X))\) then we have \[\der\phi(X) = \trace(G X) = \langle
G', X \rangle.\] \comment{Does Magnus agree with this transpose?}

Important things, calculating the directional derivative is a linear operation
that is performed by applying the Jacobian matrix \(J\) to a directions vector \(v\) as
such \(Jv\). This is the case because the Jacobian contains in its columns all
the partial derivatives i.e. the directional derivatives w.r.t. every canonical
direction \(e_i^n\). For a scalar function \(\phi\) we have that the directional
derivative is \(\langle v, \nabla\phi(x)' \rangle\). This can be generalized to
matrices with the Frobenius inner product i.e. \(\langle V, \nabla\phi(X)'
\rangle.\) Also the total derivative of a scalar function \(\phi(X)\) can be
expressed in terms of differentials as \(\langle G, \dif X \rangle,\) where
\(G\) is the gradient.

There is another way to think about this. First let us consider that with
broadcasting we can define a generalization of the matrix-scalar power that we
are going to call the Hadamard power operation, i.e. given two matrices \(A\)
and \(E\), both of size \(n \times m\), when we write \(A^{{\hadam}E}\) it
calculates \(\alpha_{i,j}^{\epsilon_{i,j}}\) for each \(i,j.\)

We can use it with the Napier's constant \(e\) to neatly express the
element-wise exponential function applied to a matrix \(X\) of size \(n \times
m\) as \(e^{{\hadam}X} \Rightarrow (e 1_n 1_m')^{{\hadam}X}.\)

Let us say that we are interested in back-propagating through \(\beta\) in
the following expression \(C = A^{{\hadam}\beta}.\)\footnote{Mainly because
every other binary operation supports back-propagation though the left and the
right argument.} We have \[
C = A^{{\hadam}\beta} \Rightarrow A^{{\hadam}\beta \cdot 1_n 1_m'} 
= A^{{\hadam}\left[\begin{array}{cc}\beta&\beta\\\beta&\beta\end{array}\right]}
= A^{{\cdot}\beta}.
\]

Which since scalar and matrix back propagation are semantically equivalent, this
is the same as doing
\begin{eqnarray*}
    \chi_1 &=& \alpha_{1,1}^\beta = \alpha_1^\beta \\
    \chi_2 &=& \alpha_{1,2}^\beta = \alpha_2^\beta \\
    \chi_3 &=& \alpha_{2,1}^\beta = \alpha_3^\beta \\
    \chi_4 &=& \alpha_{2,2}^\beta = \alpha_4^\beta
\end{eqnarray*}
with scalar back-propagation. If we think at the graph representation of this
operation we have that \(\beta\) has an out-degree of 4 and when we
back-propagate through it we have to sum the contribution of 4 nodes to obtain
its gradient, that is if we let \(c = [\begin{array}{cccc}\chi_1& \chi_2&
\chi_3& \chi_4 \end{array}]'\) and \(g =
[\begin{array}{cccc}\bar\chi_1& \bar\chi_2& \bar\chi_3& \bar\chi_4
\end{array}]'\) then \[
\bar\beta
    = g' \partialfrac c \beta
    = \sum_{i=1}^{4} \bar\chi_i \times \partialfrac{\chi_i}{\beta}
    = \sum_{i=1}^{4} \bar\chi_i \times \chi_i \times \ln(\alpha_i),
\] which is summing all the contributions from the gradients accumulated during
the back-propagation in a single variable. which is exactly what we have done
before in the case of scalar broadcasting for the addition.

The areas of automatic differentiation and deep learning are really wide, there
are therefore many things that we are not going to cover in this work.

For example we are only going to concern ourselves with differentiation of
static DAGs, while it is possible to differentiate control flow
constructs\footnote{Both using an execution tracing approach and representing
this operations directly in our static graph.} (like \texttt{if}, \texttt{for},
\dots) and through data structures.

There are also other operations that needed by some important deep learning
architectures (convolution is probably the most important one missing) but we
are not going to talk about them. Also our discussion is going to naively assume
that all our functions are differentiable, but the really common ReLU activation
function is not differentiable at all points. Our derivations can work also for
some non differentiable function if we generalize the concept of gradient to
sub-gradient.
\fi %}

\bibliographystyle{plain}
\bibliography{bibliography}

\end{document}