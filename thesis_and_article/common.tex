\newcommand{\TODO}[1]{TODO: #1}
% https://tex.stackexchange.com/a/284054
\newcommand{\argmin}{\mathop{\mathrm{arg\,min}}}
\newcommand{\trace}{\mathrm{tr}}
\newcommand{\expected}{\mathop\mathsf E}
\newcommand{\partialfrac}[2]{\frac{\partial #1}{\partial #2}}
\newcommand{\Diag}{\mathrm{Diag}} % diagonalize vector to matrix
\newcommand{\diag}{\mathbf{diag}} % diagonalize matrix to vector
\newcommand{\vect}{\mathbf{vec}} % vectorize
\newcommand{\krone}{\otimes} % Kronecker product
\newcommand{\hadam}{\odot} % Hadamard product
% https://tex.stackexchange.com/a/163041
\newcommand{\dif}{\mathop{}\!\mathsf d} % differential
\newcommand{\der}{\mathop{}\!\mathsf D} % derivative
\newcommand{\pluseq}{\mathrel{{+}{=}}}
\renewcommand\vec[1]{\mathbf{#1}}
% https://ctan.org/pkg/leftidx
\newcommand\leftidx[3]{{\vphantom{#2}}#1#2#3}
\newcommand\cmat[3]{% constant scalar matrix
    \leftidx{_{#2}}{\vec{#1}}{}%
    \,% small space to aid readability
    \leftidx{_{#3}}{\vec{#1}}{'}%
}
\newcommand\matr{\mathrm{Mat}}

% https://tex.stackexchange.com/questions/133147/how-to-use-pdfcolorstack
% 0 is the main color stack
\chardef\Color=0
\newcommand{\comment}[1]{\pdfcolorstack \Color push {0 1 0 rg 0 1 0 RG} #1 \pdfcolorstack \Color pop {}}
%\newcommand{\article}[1]{\pdfcolorstack \Color push {0 0 1 rg 0 0 1 RG} #1 \pdfcolorstack \Color pop {}}

% https://tex.stackexchange.com/a/638348
% http://latexref.xyz/_005cDeclareGraphicsRule.html
\DeclareGraphicsRule{*}{mps}{*}{}