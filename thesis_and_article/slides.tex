\documentclass[noamsthm]{beamer}
\usetheme{sapienza}
% \logo{\includegraphics[width=0.09\paperwidth]{assets/logo_RGB}}
\titlebackground*{assets/background}

\setbeamertemplate{footline}{}

\newcommand{\TODO}[1]{TODO: #1}
% https://tex.stackexchange.com/a/284054
\newcommand{\argmin}{\mathop{\mathrm{arg\,min}}}
\newcommand{\trace}{\mathrm{tr}}
\newcommand{\expected}{\mathop\mathsf E}
\newcommand{\partialfrac}[2]{\frac{\partial #1}{\partial #2}}
\newcommand{\Diag}{\mathrm{Diag}} % diagonalize vector to matrix
\newcommand{\diag}{\mathbf{diag}} % diagonalize matrix to vector
\newcommand{\vect}{\mathbf{vec}} % vectorize
\newcommand{\krone}{\otimes} % Kronecker product
\newcommand{\hadam}{\odot} % Hadamard product
% https://tex.stackexchange.com/a/163041
\newcommand{\dif}{\mathop{}\!\mathsf d} % differential
\newcommand{\der}{\mathop{}\!\mathsf D} % derivative
\newcommand{\pluseq}{\mathrel{{+}{=}}}
\renewcommand\vec[1]{\mathbf{#1}}
% https://ctan.org/pkg/leftidx
\newcommand\leftidx[3]{{\vphantom{#2}}#1#2#3}
\newcommand\cmat[3]{% constant scalar matrix
    \leftidx{_{#2}}{\vec{#1}}{}%
    \,% small space to aid readability
    \leftidx{_{#3}}{\vec{#1}}{'}%
}
\newcommand\matr{\mathrm{Mat}}

% https://tex.stackexchange.com/questions/133147/how-to-use-pdfcolorstack
% 0 is the main color stack
\chardef\Color=0
\newcommand{\comment}[1]{\pdfcolorstack \Color push {0 1 0 rg 0 1 0 RG} #1 \pdfcolorstack \Color pop {}}
%\newcommand{\article}[1]{\pdfcolorstack \Color push {0 0 1 rg 0 0 1 RG} #1 \pdfcolorstack \Color pop {}}

% https://tex.stackexchange.com/a/638348
% http://latexref.xyz/_005cDeclareGraphicsRule.html
\DeclareGraphicsRule{*}{mps}{*}{}

\title{Demystifying Back-Propagation}
\subtitle{Matrix calculus for the working computer scientist}
\author{Diego Bellani}
\IDnumber{1838645}
\course{Computer Science}
%\AccademicYear
\date{Academic Year 2023/2024}

\setlength{\parskip}{1em plus2pt}

% https://tex.stackexchange.com/questions/146529/design-a-custom-beamer-theme-from-scratch
\begin{document}
\maketitle

\section{Introduction}

\begin{frame}{Training neural networks}
Neural networks are trained via the gradient descent algorithm
\[\theta^{[n+1]} = \theta^{[n]} - \epsilon \nabla j\!\left(\theta^{[n]}\right)\!.\]

The most important step in this algorithm is the calculation of the gradient.
\end{frame}

\begin{frame}{Gradient calculation}
\emph{Back-propagation} is the algorithm used to calculate the gradient.

At its core it is a conceptually simple and efficient algorithm.

If we deal only with scalars its implementation is also trivial.
\end{frame}

\begin{frame}{The topic of this thesis}
\framesubtitle{Filling a gap in the literature}
There is surprisingly little (accessible) information about back-propagation
with matrices as primitives.

We want to allow any computer scientist\footnote{With a standard mathematical background.}
to derive formulas for back-propagating through matrix expressions.

This was done by using a \emph{matrix calculus} from the econometric literature.
\end{frame}

\begin{frame}{Why care?}
\begin{itemize}
\item Implementing the algorithm in another language.
\item Optimize it (e.g. implementing custom operations on exotic hardware.)
\item Debug current implementations (e.g. there may be subtle numerical issues.)
\end{itemize}
\end{frame}

\section{Back-propagation}

\begin{frame}{An illustrative example}
Let \[y = f(a,b) = \ln(a) + a b - \sin(b),\] we can evaluate it piece by piece
as such
\begin{eqnarray*}
a = x^{[-1]} &=& 2 \\
b = x^{[0]}  &=& 5 \\
    x^{[1]}  &=& \ln\left(x^{[-1]}\right) = \ln(2) \\
    x^{[2]}  &=& x^{[-1]} \times x^{[0]} = 2 \times 5 \\
    x^{[3]}  &=& \sin\left(x^{[0]}\right) = \sin(5) \\
    x^{[4]}  &=& x^{[1]} + x^{[2]} \approx 0.693 + 10 \\
y = x^{[5]}  &=& x^{[4]} + x^{[3]} \approx 10.693 + 0.959
\end{eqnarray*}    
\end{frame}

\begin{frame}{Dependency graph}
Then we can look at \(y\)'s dependency graph
\begin{center}
\includegraphics{figures.2}
\end{center}
\end{frame}

\begin{frame}{Total derivatives}
We know by looking at the graph and with basic calculus that \[
    \partialfrac y b = \partialfrac{x^{[5]}}{x^{[0]}}
    = \overbrace{
        \partialfrac{x^{[5]}}{x^{[4]}} \partialfrac{x^{[4]}}{x^{[2]}}
    }^{\partialfrac{x^{[5]}}{x^{[2]}}} \partialfrac{x^{[2]}}{x^{[0]}}
    + \partialfrac{x^{[5]}}{x^{[3]}} \partialfrac{x^{[3]}}{x^{[0]}},
\] and similarly for the other derivative we have that \[
    \partialfrac y a = \partialfrac{x^{[5]}}{x^{[-1]}}
    = \overbrace{
        \partialfrac{x^{[5]}}{x^{[4]}} \partialfrac{x^{[4]}}{x^{[1]}}
    }^{\partialfrac{x^{[5]}}{x^{[1]}}} \partialfrac{x^{[1]}}{x^{[-1]}}
    + \overbrace{\partialfrac{x^{[5]}}{x^{[4]}} \partialfrac{x^{[4]}}{x^{[2]}}
    }^{\partialfrac{x^{[5]}}{x^{[2]}}} \partialfrac{x^{[2]}}{x^{[-1]}}.
\]
\end{frame}

\begingroup
\renewcommand{\partialfrac}[2]{\textstyle\frac{\partial#1}{\partial#2}}
\begin{frame}{Evaluating the derivatives}
\framesubtitle{Avoiding redundant calculations}
Let \(\bar x^{[i]} = \partialfrac y {x^{[i]}},\) then we propagate the gradient in topological order
\begin{eqnarray*}
\bar x^{[5]}    &=& \partialfrac{x^{[5]}}{x^{[5]}} = 1 \\
\bar x^{[4]}    &=& \bar x^{[5]}\partialfrac{x^{[5]}}{x^{[4]}}
    = \bar x^{[5]} \times 1 = 1 \\
\bar x^{[3]}    &=& \bar x^{[5]}\partialfrac{x^{[5]}}{x^{[3]}}
    = \bar x^{[5]} \times (-1) = -1 \\
\bar x^{[1]}    &=& \bar x^{[4]}\partialfrac{x^{[4]}}{x^{[1]}}
    = \bar x^{[4]} \times 1 = 1 \\
\bar x^{[2]}    &=& \bar x^{[4]}\partialfrac{x^{[4]}}{x^{[2]}}
    = \bar x^{[4]} \times 1 = 1 \\
\bar x^{[0]}    &=& \bar x^{[3]}\partialfrac{x^{[3]}}{x^{[0]}}
    = \bar x^{[3]} \times \cos\left(x^{[0]}\right) \approx -0.282 \\
\bar x^{[-1]} &=& \bar x^{[2]}\partialfrac{x^{[2]}}{x^{[-1]}}
    = \bar x^{[2]} \times x^{[0]} = 5 \\
\bar x^{[0]}    &\pluseq\phantom+& \bar x^{[2]}\partialfrac{x^{[2]}}{x^{[0]}}
    = \bar x^{[2]} \times x^{[-1]} \approx 1.716 \\
\bar x^{[-1]} &\pluseq\phantom+& \bar x^{[1]}\partialfrac{x^{[1]}}{x^{[-1]}}
    = \frac{\bar x^{[1]}}{x^{[-1]}} = 5.5
\end{eqnarray*}
\end{frame}
\endgroup

\begin{frame}{The vector case}
Let us now focus on the vector case, where \[y = f(\vec a,\vec b) = \Vert\!\ln(\vec a) + \vec a \hadam \vec b - \sin(\vec b)\Vert^2,\]
the procedure is much the same, we just have to be careful about a few things\dots
\end{frame}

\begingroup
\renewcommand\matrix{\left[\begin{array}{ccc}&&\\&&\\&&\end{array}\right]}
\renewcommand\vector{\left[\begin{array}{ccc}&&\end{array}\right]}
\begin{frame}{Why we go backward}
Now associativity matters,
\[\partialfrac{x^{[n]}}{\vec x^{[n-1]}} \partialfrac{\vec x^{[n-1]}}{\vec x^{[n-2]}}
\cdots \partialfrac{\vec x^{[1]}}{\vec x^{[0]}},\]
if we look at the shapes
\[\vector \matrix \cdots \matrix\!\!.\]
This is why it is called \emph{back-propagation}. We propagate the gradient
backward in the dependency graph.
\end{frame}
\endgroup

\begin{frame}{vector-Jacobian product (vJp)}
\framesubtitle{Taking advantage of Jacobians' sparsity}
Let us focus on \(\vec g' \partialfrac{\sin(\vec b)}{\vec b'}\) (or any
element-wise function,) where \(\vec g'\) is the gradient we propagated so far.
We have that \[
\vec g' \partialfrac{\sin(\vec b)}{\vec b'}
= \vec g' \left[\begin{array}{ccc}
    \cos(b_1) & \cdots & 0 \\
    \vdots & \ddots & \vdots \\
    0 & \cdots & \cos(b_n)
\end{array}\right]
= \vec g' \hadam cos(\vec b)'.
\]
Hence (almost always) we can perform the vector-matrix multiplication without
explicit instantiation of the Jacobian.
\end{frame}

\begin{frame}{So far so good\dots}
But what about gradient propagation through functions of matrix argument and value?
Lets look at
\[\partialfrac{x^{[n]}}{\vec X^{[n-1]}} \partialfrac{\vec X^{[n-1]}}{\vec X^{[n-2]}}
\cdots \partialfrac{\vec X^{[1]}}{\vec X^{[0]}}.\]
\begin{itemize}
\item What shape do they even have (i.e. how do we arrange the partial derivatives)?
\item How do we multiply this mathematical objects?
\item Does the chain rule apply?
\end{itemize}
\end{frame}

\begin{frame}{Current approaches}
Given the fact that we have already have back-propagation implementation for matrices how did they do it?
\begin{itemize}
\item Indices and summations,
\item tensors~\cite{laue2018} (Ricci calculus) or
\item operator theory.
\end{itemize}
\end{frame}

\begin{frame}{Indices and summations}
Very accessible but it just does not scale, some derivations can be done, but it
gets out of hand quickly.

Therefore except for some examples nobody really uses it.
\end{frame}

\begin{frame}{Ricci calculus}
The tensor approach can be used but the math is\dots Hard.

Legend has it that even Einstein was helped on the topic by Marcel Grossmann.
\end{frame}

\begin{frame}{Operators theory}
The approaches based on operator theory usually go something like
\begin{quote}
Something, something inner product spaces, something, something adjoint
trick\dots This is the result. Okay moving on\dots
\end{quote}

Again not very accessible\dots
\end{frame}

\begin{frame}{Matrix calculus is the answer}
\framesubtitle{Right?}
There is no standard ``Matrix Calculus''.

Most of them just ignore the concept of high order Jacobian.

All of them do not care about back-propagation.
\end{frame}

\iffalse
\begin{frame}{We need another approach}
Whatever we are going to do our generalized Jacobians needs
\begin{itemize}
\item to contain all possible partial derivatives;
\item a definition of multiplication to propagate the gradient;
\item a reasonable chain rule.
\end{itemize}
\end{frame}
\fi

\begin{frame}{The problem}
We need to derive vJps for operations involving matrices, so that we can propagate the gradient.

If we upgrade our running example to
\[y = f(A,B) = \Vert\!\ln(A) + A B - \sin(B)\Vert^2,\]
we see that we just need vJps for all the \emph{elementary} operations, the
back-propagation algorithm is otherwise unchanged.
\end{frame}

\begin{frame}{Me for a few months}
\framesubtitle{Trying to put together the pieces of the puzzle}
\centering
\includegraphics[height=\textheight]{res/pepesilvia.jpg}
\end{frame}

\section{Matrix Calculus}

\begin{frame}{Enter matrix calculus}
\framesubtitle{By Magnus and Naudecker~\cite{magnus2019}}
Let \(\vec y=\vec f(\vec x)\) than
\[
    \der\vec f(\vec x)
    = \partialfrac{\vec y}{\vec x'}
    = \left[\begin{array}{ccc}
        \partialfrac{\vec y_1}{\vec x_1} & \cdots & \partialfrac{\vec y_1}{\vec x_m} \\
        \vdots & \ddots & \vdots \\
        \partialfrac{\vec y_n}{\vec x_1} & \cdots & \partialfrac{\vec y_n}{\vec x_m}
    \end{array}\right]\!\!.
\] 
\end{frame}

\begin{frame}{Differentials}
\framesubtitle{For scalars}
The definition of the derivative
\[\lim_{h \to 0} \frac{f(x + h) - f(x)}h = \der f(x)\]
can be rewritten as
\[f(x + h) = f(x) + (\der f(x))h + r(h,x),\]
where \(\lim_{h \to 0} r(h,x) = 0.\)

Here \(\dif(x;h) = (\der f(x))h\) is the differential.
\end{frame}

\begin{frame}{Differentials}
\framesubtitle{Generalizations}
We can do similar reasoning for vector functions
\[\vec f(\vec x + \vec h) = \vec f(\vec x) + (\der\vec f(\vec x))\vec h + \vec r(\vec h,\vec x)\]
we can generalize our definition to the matrix
\[\vect(F(X+H)) = \vect(F(X)) + (\der\vect(F(X))\vect(H) + \vect(R(H,X)).\]

Hence the \(\dif(X;H)=(\der\vect(F(X))\vect(H),\) which is just a big vector.
\end{frame}

\begin{frame}{Derivative of a matrix}
\framesubtitle{Just a big matrix}
Let \(Y=F(X)\) than
\[\der F(X) = \partialfrac{\vect(Y)}{\vect(X)'}.\]
This is a generalization of the classic definition, we can interpret the rank
and determinant as usual and the chain rule is the same~\cite{magnus2007}.
\end{frame}

%Cauchy's rule of invariance

\begin{frame}{Matrix differential calculus}
\framesubtitle{With a simpler notation}
\begin{eqnarray*}
\dif A            &=& \vec 0 \vec 0', \\
\dif (a \cdot X)  &=& a \cdot \dif X, \\
\dif (X + Y)      &=& \dif X + \dif Y, \\
\dif (x \cdot X)  &=& (\dif x) \cdot X + x \cdot \dif X, \\
\dif (XY)         &=& (\dif X)Y + X \dif Y, \\
\dif (X \hadam Y) &=& (\dif X) \hadam Y + X \hadam \dif Y, \\
\dif X^{{\cdot}a} &=& a \cdot X^{{\cdot}a-1} \hadam \dif X, \\
\dif a^{{\cdot}X} &=& \ln(a) \cdot a^{{\cdot}X} \hadam \dif X, \\
\dif X'           &=& (\dif X)', \\
\dif\vect(X)      &=& \vect(\dif X), \\
\dif\trace(S)     &=& \trace(\dif S).
\end{eqnarray*}
\end{frame}

\begin{frame}{Identification theorems}
The identification rules are \[\dif\vec f(\vec x) = A(\vec x)
\dif\vec x \iff \der \vec f(\vec x) = A(\vec x)\] and \[\dif \vect(F(X)) = A(X)
\dif\vect(X) \iff \der F(X) = A(X).\]
\begin{tabular}{cccc}
 & \(x\) & \(\vec x\) & \(X\) \\[2pt]
         \cline{2-4}
\multicolumn{1}{r|}{\(\dif f =\)}
    & \(a(x) \dif x\) & \(\vec a(\vec x)' \dif\vec x\) & \(\vect(A(X))' \dif\vect(X)\) \\
\multicolumn{1}{r|}{\(\dif\vec f =\)}
    & \(\vec a(x) \dif x\) & \(A(\vec x) \dif\vec x\) & \(A(X) \dif\vect(X)\) \\
\multicolumn{1}{r|}{\(\dif F =\)}
    & \(\vect(A(x)) \dif x\) & \(A(\vec x) \dif\vec x\) & \(A(X) \dif\vect(X)\)
\end{tabular}
\end{frame}

\begin{frame}{The last piece of the puzzle}
While calculating the derivative of  a scalar value function \(X^{[j]} =
F\!\left(X^{[i]}\right),\) where \(j > i,\) when performing back-propagation we
have \[\vect\!\left(\bar X^{[i]}\right)' =
\vect\!\left(\bar X^{[j]}\right)' \partialfrac{X^{[j]}}{X^{[i]}} =
\partialfrac{x^{[n]}}{X^{[j]}} \partialfrac{X^{[j]}}{X^{[i]}} =
\partialfrac{x^{[n]}}{X^{[i]}}.\]

This is equivalent to
\[\vect\!\left(\bar X^{[i]}\right)'
= \mathrm{vJp}_F\!\left(X^{[i]},X^{[j]}\right)
= \partialfrac{\left\langle\bar X^{[j]},X^{[j]}\right\rangle}{\vect\!\left(X^{[i]}\right)'}
= \partialfrac{\left\langle\bar X^{[j]},F\left(X^{[i]}\right)\right\rangle}{\vect\left(X{[i]}\right)'}.\]
\end{frame}

\section{Derivations}

\begin{frame}{The contribution}

Given a function \(F(X)\) that you want to back propagate through, write
{\LARGE\[\dif\langle G, F(X)\rangle\]}
then apply differentials until you can use the identification theorem.
\end{frame}

\begin{frame}{Matrix multiplication}
\framesubtitle{Now is easy!}
\begin{minipage}{.5\textwidth}
\begin{eqnarray*}
\lefteqn{\dif\langle G, X B \rangle} \\
&=& \dif\trace(G' (X B)) \\
&=& \trace(G' (\dif X) B) \\
&=& \trace(B G' \dif X) \\
&=& \trace((G B')' \dif X)) \\
&=& \vect(G B')' \vect(\dif X) \\
\end{eqnarray*}
\end{minipage}%
\begin{minipage}{.5\textwidth}
\begin{eqnarray*}
\lefteqn{\dif\langle G,A X\rangle} \\
&=& \dif\trace(G' A X) \\
&=& \trace(G' A \dif X) \\
&=& \trace((A' G)' \dif X) \\
&=& \vect(A' G)' \vect(\dif X). \\
&& \\
\end{eqnarray*}
\end{minipage}
\end{frame}

\begin{frame}{Element-wise functions}
For an element-wise function \(F\) we have its differential is \(\dif F(X) =
\tilde F(X) \hadam \dif X,\) where \(\tilde F(X)\) is the derivative of the
function applied to each element.
\begin{eqnarray*}
\dif\langle G, F(X) \rangle
&=& \dif\trace(G' F(X)) \\
&=& \trace(\dif(G' F(X))) \\
&=& \trace(\dif G' F(X) + G' \dif F(X)) \\
&=& \trace(G' \dif F(X)) \\
&=& \trace(G' (\tilde F(X) \hadam \dif X)) \\
&=& \vect(G)' \vect(\tilde F(X) \hadam \dif X) \\
&=& \vect(G)' \diag(\vect(\tilde F(X))) \vect(\dif X) \\
&=& \vect(G \hadam \tilde F(X))' \vect(\dif X).
\end{eqnarray*}
\end{frame}

\begin{frame}{Broadcasting}
It is common to write \(AX + \vec b\) to mean
\(AX + \vec b \vec 1',\) or \(AX + b\) to mean \(AX + b \cdot
\vec 1 \vec 1'.\) For example
let  \(F(b) = b \cdot \vec 1 \vec 1',\) then
\begin{minipage}{.45\textwidth}
\begin{eqnarray*}
\lefteqn{\dif F(b)} \\
&=& \dif(b \cdot \vec 1 \vec 1') \\
&=& \dif(b \cdot \vec 1) \vec 1' + b \vec 1 \dif \vec 1' \\
&=& \dif(b \cdot \vec 1) \vec 1' \\
&=& (\dif(b) \cdot \vec 1 + b \dif \vec 1) \vec 1' \\
&=& \dif b \cdot \vec 1  \vec 1' \\
&=& \vec 1 \vec 1' \cdot \dif b. \\
&& \\
&& \\
\end{eqnarray*}
\end{minipage}%
\begin{minipage}{.5\textwidth}
\begin{eqnarray*}
\lefteqn{\dif\langle G,F(b)\rangle} \\
&=& \dif\trace(G' F(b)) \\
&=& \trace(\dif(G' F(b))) \\
&=& \trace(\dif G' F(b) + G' \dif F(b)) \\
&=& \trace(G' \dif F(b))) \\
&=& \trace(G' \vec 1 \vec 1' \cdot \dif b)) \\
&=& \trace(G' \vec 1 \vec 1') \cdot \dif b \\
&=& \trace(\vec 1' G' \vec 1) \cdot \dif b \\
&=& \vec 1' G' \vec 1 \cdot \dif b. \\
\end{eqnarray*}
\end{minipage}
\end{frame}

\begin{frame}{Softmax}
\framesubtitle{Trust me on this one}
Let
\(S(X) = e^{{\cdot}X} \hadam \left(\vec 1 \vec 1' e^{{\cdot}X}\right)^{{\cdot}-1}\) then
\[\dif\langle G,S(X)\rangle = \trace\!\left(
    \left(
    B - e^{{\cdot}X} \Diag\left(\left(\vec 1' \left(B \hadam A^{{\cdot}-1}\right)\right)'\right)
    \right)' \dif X\right)\!,\]
where \(A = \vec 1 \vec 1' e^{{\cdot}X}\) and \(B = G \hadam S(X).\)\footnote{Numerical issues aside.}
\end{frame}

\section{Implementation}

\begin{frame}{Implementation}
The results of our derivations have been used to implement a simple deep
learning library using Apple's Metal.\footnote{Because it is ``what the convent
delivers''.}

And the resulting gradient confronted, with the state of the art PyTorch
library, by implementing the same neural network in both and comparing the
results up to a certain numerical error.
\end{frame}

\begin{frame}[fragile]{A taste of Metal.Framework}
\framesubtitle{A fairly low level API}
\footnotesize
\begin{verbatim}
id<MTLCommandQueue> cmdQueue = [dev newCommandQueue];
id<MTLCommandBuffer> cmdBuf = [cmdQueue commandBuffer];
id<MTLComputeCommandEncoder> cCmdEnc = [cmdBuf computeCommandEncoder];
[cCmdEnc setComputePipelineState:cps];
[cCmdEnc setBuffer:buf offset:0 atIndex:0];
[cCmdEnc setBytes:&(float){1.0f} length:sizeof 1.0f atIndex:1];
[cCmdEnc dispatchThreads:MTLSizeMake(len,1,1)
   threadsPerThreadgroup:MTLSizeMake(cps.threadExecutionWidth,1,1)];
[cCmdEnc endEncoding];
[cmdBuf commit];
\end{verbatim}
\end{frame}

\begin{frame}[fragile]{A taste of Metal Shading Language}
\framesubtitle{An high level language}
\footnotesize
\begin{verbatim}
[[kernel]] void
simple_function(
        device   float *x [[buffer(0)]],
        constant float *a [[buffer(1)]],
                 uint   i [[thread_position_in_grid]]
    ) {
    x[i] += 1.0f;
    x[i] *= *a;
}
\end{verbatim}
\end{frame}


\section{Conclusion}

\begin{frame}{Conclusion}
\framesubtitle{And future work}
This is the first self-contained resource that allows any computer scientist,
with a standard mathematical background, to derive back-propagation formulas for
his or her operations.
\end{frame}

\begin{frame}{Future work}
\framesubtitle{Direct sums and products}
\newcommand{\Z}{\phantom{{}'}\vec 0 \vec 0'}

Batched matrix multiplication with a matrix \(W\) is \[
\left(\bigoplus_{i=1}^n A^{[i]}\right) (\vec 1 \krone W)
= \left[\begin{array}{ccc}
    A^{[1]} & \cdots & \Z \\
    \vdots & \ddots & \vdots \\
    \Z & \cdots & A^{[n]}
\end{array}\right]
\left[\begin{array}{c}W \\ \vdots \\ W \end{array}\right]
= \left[\begin{array}{c}A^{[1]} W \\ \vdots \\ A^{[n]} W\end{array}\right]\!\!.\]

If we need to apply bias we can similarly do \[
\left(\bigoplus_{i=1}^n A^{[i]}\right) (\vec 1 \krone W)
+ (\vec 1 \krone \vec b \vec 1')
= \left[\begin{array}{c}
    A^{[1]} W + \vec b \vec 1' \\ \vdots \\ A^{[n]} W + \vec b \vec 1'
\end{array}\right]\!\!.
\]
\end{frame}

% TODO: rimuovere numeri lucidi
% TODO Citare un po' di cose...

\begin{frame}
\centering
\LARGE Fin.

\footnotesize Questions?
\end{frame}

\begin{frame}{Bibliography}
\bibliographystyle{alpha}
\bibliography{bibliography}
\end{frame}

\end{document}