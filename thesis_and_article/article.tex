\let\oldcline\cline
\documentclass[pdflatex,sn-mathphys-num]{sn-jnl}%

%\usepackage{graphicx}%
%\usepackage{multirow}%
\usepackage{amsmath}% creates lots of problems...
\newcommand{\cline}[1]{\oldcline{#1}}
%\usepackage{amssymb,amsfonts}
%\usepackage{amsthm}%
%\usepackage{mathrsfs}%
%\usepackage[title]{appendix}%
%\usepackage{xcolor}%
%\usepackage{textcomp}%
%\usepackage{manyfoot}%
%\usepackage{booktabs}%
%\usepackage{algorithm}%
%\usepackage{algorithmicx}%
%\usepackage{algpseudocode}%
%\usepackage{listings}%
%%%%

\iffalse %{
\theoremstyle{thmstyleone}%
\newtheorem{theorem}{Theorem}%  meant for continuous numbers
%%\newtheorem{theorem}{Theorem}[section]% meant for sectionwise numbers
%% optional argument [theorem] produces theorem numbering sequence instead of independent numbers for Proposition
\newtheorem{proposition}[theorem]{Proposition}% 
%%\newtheorem{proposition}{Proposition}% to get separate numbers for theorem and proposition etc.

\theoremstyle{thmstyletwo}%
\newtheorem{example}{Example}%
\newtheorem{remark}{Remark}%

\theoremstyle{thmstylethree}%
\newtheorem{definition}{Definition}%
\fi %}

\raggedbottom
%%\unnumbered% uncomment this for unnumbered level heads

\newcommand{\TODO}[1]{TODO: #1}
% https://tex.stackexchange.com/a/284054
\newcommand{\argmin}{\mathop{\mathrm{arg\,min}}}
\newcommand{\trace}{\mathrm{tr}}
\newcommand{\expected}{\mathop\mathsf E}
\newcommand{\partialfrac}[2]{\frac{\partial #1}{\partial #2}}
\newcommand{\Diag}{\mathrm{Diag}} % diagonalize vector to matrix
\newcommand{\diag}{\mathbf{diag}} % diagonalize matrix to vector
\newcommand{\vect}{\mathbf{vec}} % vectorize
\newcommand{\krone}{\otimes} % Kronecker product
\newcommand{\hadam}{\odot} % Hadamard product
% https://tex.stackexchange.com/a/163041
\newcommand{\dif}{\mathop{}\!\mathsf d} % differential
\newcommand{\der}{\mathop{}\!\mathsf D} % derivative
\newcommand{\pluseq}{\mathrel{{+}{=}}}
\renewcommand\vec[1]{\mathbf{#1}}
% https://ctan.org/pkg/leftidx
\newcommand\leftidx[3]{{\vphantom{#2}}#1#2#3}
\newcommand\cmat[3]{% constant scalar matrix
    \leftidx{_{#2}}{\vec{#1}}{}%
    \,% small space to aid readability
    \leftidx{_{#3}}{\vec{#1}}{'}%
}
\newcommand\matr{\mathrm{Mat}}

% https://tex.stackexchange.com/questions/133147/how-to-use-pdfcolorstack
% 0 is the main color stack
\chardef\Color=0
\newcommand{\comment}[1]{\pdfcolorstack \Color push {0 1 0 rg 0 1 0 RG} #1 \pdfcolorstack \Color pop {}}
%\newcommand{\article}[1]{\pdfcolorstack \Color push {0 0 1 rg 0 0 1 RG} #1 \pdfcolorstack \Color pop {}}

% https://tex.stackexchange.com/a/638348
% http://latexref.xyz/_005cDeclareGraphicsRule.html
\DeclareGraphicsRule{*}{mps}{*}{}

\begin{document}

\title{Demystifying Back-Propagation}
\subtitle{Matrix calculus for the working computer scientist}
\author{\fnm{Diego}\sur{Bellani}}\email{bellani.1838645@studenti.uniroma1.it}
\author{\fnm{Annalisa}\sur{Massini}}\email{massini@di.uniroma1.it}
\affil{%
    \orgdiv{Computer Science Department},
    \orgname{Sapienza University of Rome},
    \orgaddress{
        \street{viale Regina Elena 295},
        \city{Rome},
        \postcode{00161},
        \country{Italy}%
    }%
}

\abstract{% The setting
In deep learning libraries, the back-propagation algorithm is typically
implemented to automatically calculate gradients, relieving users of the burden
of understanding the underlying mechanics. However, when implementing a
optimized GPU kernel for a custom operation, understanding how gradients are
propagated becomes necessary.

% The problem
To propagate gradients through batched (or matrix) operations, one needs to grasp
enough automatic differentiation and matrix calculus, which can be challenging
for various reasons. Matrix results in the automatic differentiation literature
are usually roughly explained and rely on advanced mathematical concepts. The
matrix calculus literature typically is not concerned with derivations for
back-propagation. Finally, the lack of a standard definition for matrix
derivative makes it hard to translate between different resources.

% Our solution/contribution
In this work, we present the back-propagation algorithm in matrix form together
with a simple matrix calculus, providing the readers with the necessary tools to
derive gradient propagation formulas. Our work constitutes the first short and
self-contained explanation of both of these topics, addressing a gap in existing
literature.}

\keywords{Back-propagation, Automatic Differentiation, Matrix Calculus}

\maketitle

\section{Introduction}\label{intro}

Machine learning is a subset of artificial intelligence (AI) in which
statistical models and training algorithms are developed, such that, given a set
of data, the training algorithm can specialize the statistical model to solve a
certain task.

Machine learning, in some form or another, has been around for a long time; think for example of inferential statistics. The reason it only recently became a viable
method, and ultimately the most effective one in many fields, is due to the
abundance of easily available data in digital form (in the order of gibibyte and
tebibyte) and the huge computing power available.

The machine learning sub-field that has contributed most to the success of the field, i.e. the late winter and early spring of artificial intelligence, is the so-called \emph{deep
learning}.  This sub-field concerns a specific kind of model: deep artificial
neural networks. These networks are composed of layers and are deep in the sense
that they consist of many layers. Having many layers allows these models to perform
their tasks with minimal pre-processing of the data and without manual feature
extraction. This is opposed to more classical machine learning models which expect input of data as a vector of features.

\iftrue
The simplest example of a task commonly performed with machine learning is the one of image classification. In this task, you have a set of labels for pictures (e.g. animal, car, book, etc.) and, given an input image, you have to assign the correct label to it.  
When thinking about image classification, it is not obvious what features to extract from images for a classification algorithm to be able to perform its task. With the deep learning approach, we provide the images to the algorithm, almost as-is, and the layers of the model learn to perform the various feature extraction sub-tasks that are ultimately used by the network itself to perform classification.
\fi

The family of training algorithms that have made much of this progress possible with deep neural network models are those based on mini-batch stochastic gradient descent. This algorithm turns the process of learning into an unconstrained optimization problem with a certain loss function.

Optimizing functions requires determining their gradient and this can be done efficiently with the back-propagation algorithm.
To implement this algorithm, we need to deal with the gradient of functions with a matrix argument.
This involves the use of matrix calculus notions, which are not commonly taught in standard computer science curricula and which are not easily accessible in the vast and mixed literature on the subject.

The aim of this work is to provide an accessible introduction to matrix calculus, focused on the implementation of the back-propagation algorithm, that gives the reader the necessary tools to derive formulas for gradient propagation for her or his customized procedures.

We provide a self-contained explanation of the back-propagation
algorithm for scalar functions containing matrix expressions and the matrix calculus necessary to derive the formulas needed for an efficient implementation
of the gradient accumulation. To keep the explanation short and focused on the
deep learning use case, we avoid introducing unnecessary concepts from the
automatic differentiation literature (i.e. forward-mode gradient accumulation).

The paper is organized as follows. Section~\ref{Relwork} provides some brief
considerations on related work. Section~\ref{background} gives mathematical
notation and definitions useful in the rest of the paper. In
Section~\ref{sec:generalization}, we discuss the problems arising with the
generalization of the back-propagation algorithm to the matrix case and define a
matrix calculus.  In Section~\ref{sec:contribution}, we describe how to derive
back-propagation formulas for common operations in neural networks.  In
Section~\ref{sec:conclusion}, we give some final considerations.

\section{Related Work}\label{Relwork}

Modern deep learning architectures involve loss functions of matrix argument.
This implies the use of matrix algebra for calculating the loss and the use of
matrix calculus for the gradient estimation.  For learning matrix algebra, there
are many resources, see e.g. \cite{strang2019,searle2017,magnus2005}.
Unfortunately, the same cannot be said for matrix calculus resources focused on
deep learning.  This is exactly the gap in the literature that this work tries
to cover.

The problems that plague many resources are that they stop at the vector case or
treat the matrix case too vaguely or too
complexly~\cite{giering1998,buithanh2023} or just giving results with little to
no explanation~\cite{giles2008}.  Therefore, we are left either with resources
focused on matrix calculus~\cite{magnus2019,turkington2001}, but with no regard
for back-propagation or resources focused on automatic differentiation (of which
back-propagation is a specific case) that do not talk about the matrix
case~\cite{griewank2008, naumann2011}.

\section{Notation and Preliminaries}\label{background}

In this section, we introduce all the notation and conventions that we are going
to use in the rest of this work.

\subsection{Notation}

First of all, our functions and variables are defined in terms of
real numbers.

Lowercase Latin and Greek letters are used for scalars e.g. \(a,\) \(b,\)
\(c,\). This allows us to use this notation consistently, even for common
constants like, e.g., \(e = 2.71\ldots\)  Boldface Latin and Greek letters are
used for vectors, which are always considered as column matrices, e.g. \(\vec
a,\) \(\vec b,\) \(\vec c,\) etc. Finally, for matrices, we will use capital
Latin and Greek letters, e.g. \(A,\) \(B,\) \(C,\) etc.  The same convention is
used for functions of a certain value or argument e.g. \(F(x)\) is a function of
matrix value and vector argument.

When we have a sequence of scalars, vectors or matrices, we will index elements
with an upper script number in square brackets, so as to avoid confusion with
the exponentiation operator, e.g. \(a^{[1]},\ldots,a^{[n]}\) is a sequence of
\(n\) scalars.

% For ease of typesetting i.e. avoiding \left( and \right) and to avoid any
% confusion with exponentiation.
The transpose of a matrix \(A\) is going to be denoted with a prime superscript
\(A'\).

When we specify the dimensions of a matrix, we will write \(n \times m\),
meaning that it has \(n\) rows and \(m\) columns.  When we say that a vector has
size \(n\) we mean that it is a matrix of size \(n \times 1\), this is handy
also because the first subscript always indicates a row in both matrices and
vectors.

When we refer to an element in a vector, we will use a scalar subscript, e.g. if
\(\vec a\) is a vector of size \(n\), then we have \(\vec a =
[\begin{array}{ccc} \vec a_1 & \cdots & \vec a_n \end{array}]'.\)

For matrices, we will use a pair of scalar subscripts, separated by a comma. One
of the scalar subscripts can be substituted by a colon to indicate an entire row
or column, e.g.  if \(A\) is \(n \times m\) then \[
    A
    = \left[\begin{array}{c}A_{1,:} \\ \vdots \\ A_{n,:}\end{array}\right]
    = [\begin{array}{ccc}A_{:,1} & \cdots & A_{:,m}\end{array}]
    = \left[\begin{array}{ccc}
        A_{1,1} & \cdots & A_{1,m} \\
        \vdots & \ddots & \vdots \\
        A_{n,1} & \cdots & A_{n,m}
    \end{array}\right]\!\!.
\]

% https://en.wikipedia.org/wiki/Notation_for_differentiation
% https://en.m.wikipedia.org/wiki/Matrix_calculus#Layout_conventions
Finally, when referring to derivatives, we will use \(\der\) for the total
derivative, \(\partialfrac\relax\relax\) for the partial derivative (which will
be written using the numerator convention) and \(\dif\) for the differential.
We will elaborate further on this part in Section~\ref{sec:matcalc}.

\subsection{Common Functions and Constants}

Below, we define some operations and some useful constants, and also state some
useful properties of the operations (without proving them) that will greatly
simplify our derivations later.

The standard basis vector for a vector space of size \(n\) is going to be
denoted as \(\vec e^{[1]},\ldots,\vec e^{[n]}.\) \(\vec e^{[i]}\) has a 1 in
position \(i\) and 0 everywhere else.

We will denote the \emph{vector of ones} as \(\vec 1 = \sum_{i=1}^n e^{[i]},\)
the \emph{vector of zeros} as \(\vec 0\), the vector with all zeros, the identity
matrix \(I = [\begin{array}{ccc}e^{[1]}; \cdots; e^{[n]}\end{array}]\).

% The size of I could be done with a normal subscript, since we always index in
% matrix with two ``symbols'', therefore there is no ambiguity, but for consistency
% we don't.
To avoid ambiguity with the size of a vector or a matrix, we will use a left
subscript, e.g. if a vector space has size \(n\) then \(\leftidx{_n}{\vec
e}{^{[i]}}\) is the \(i\)th standard basis vector of size \(n\) (and trivially
\(\leftidx{_n^{\vphantom{^{[i]}}}}{\vec e}{^{[i]}_i} = 1\) for all \(i\)),
whereas \(\leftidx{_n}I{}\) is the identity matrix of size \(n \times n.\)

Thanks to the matrix product, we can specify a matrix of size \(n \times m\) of
all zeros, or ones as \(\cmat 0 n m\) and \(\cmat 1 n m.\)

The scalar-matrix product between \(a\) and A is denoted as \(a \cdot A.\)
We will also write \(A \cdot a\) to mean the same thing.

In the rest of this section let \(A\) be \(n \times m\) and \(S\) be \(n \times
n.\)

If \(B\) has the same size as \(A\), then the Hadamard product between $A$ and $B$, is
just their element-wise multiplication. It is denoted as \(A \hadam B.\)

The matrix-scalar power raises each element of a matrix to a certain \(a\), the
scalar-matrix power instead makes it so that every element of a matrix is used as
the exponent of a certain base \(a\). We will denote them as \(A^{{\cdot}a}\) and
\(a^{{\cdot}A}\), respectively. 
Therefore: \[
    A^{{\cdot}a} =
    \left[\begin{array}{ccc}
        A_{1,1}^a & \cdots & A_{1,m}^a \\
        \vdots & \ddots & \vdots \\
        A_{n,1}^a & \cdots & A_{n,m}^a
    \end{array}\right]
    \quad\mathrm{and}\quad
    a^{{\cdot}A} =
    \left[\begin{array}{ccc}
        a^{A_{1,1}} & \cdots & a^{A_{1,m}} \\
        \vdots & \ddots & \vdots \\
        a^{A_{n,1}} & \cdots & a^{A_{n,m}}
    \end{array}\right]\!\!.
\]

We can now define the \emph{softmax} of the columns contained in a matrix as \(S(X) =
e^{{\cdot}X} \hadam \left(\vec 1 \vec 1' e^{{\cdot}X}\right)^{{\cdot}-1}.\) In
the case of a vector we will write \(\vec s(\vec x) = e^{{\cdot}\vec x}
\cdot (\vec 1' e^{{\cdot}\vec x})^{{\cdot}-1}.\)

The Kronecker product, denoted $\krone$, is a sort of generalization of the
matrix scalar product but between every element of a matrix w.r.t. another
matrix. The Kronecker product \(A \krone B\), where B is any size matrix, is: \[
    A \krone B = \left[\begin{array}{ccc}
        A_{1,1} \cdot B & \cdots & A_{1,m} \cdot B \\
        \vdots & \ddots & \vdots \\
        A_{n,1} \cdot B & \cdots & A_{n,m} \cdot B
    \end{array}\right]\!\!.\]
Note that if \(A\) has size \(1 \times 1\) and \(A = a\) for some \(a\) than \(a
\krone A = a \cdot A.\)

Vectorizing a matrix means stacking its \(m\) columns on top of each other to
form a vector. Matricising a vector consisting of \(nm\) elements means taking
\(n\) elements at a time and using them as rows to form an \(n \times m\)
matrix.  We denote these two operations as \(\vect(A) = \vec a\) and
\(\matr(\vec a) = A\), respectively, and we define them formally as
\begin{eqnarray*}
    \vect(A) &=& \sum_{i=1}^{m} \vec e^{[i]} \krone A \vec e^{[i]}, \\
    % TODO: mettere un commento con l'URL a stackoverflow.
    \matr(\vec a) &=&
        (\vect(\leftidx{_n}I{})' \krone \leftidx{_m}I{})
        (\leftidx{_n}I{} \krone \vec a).
\end{eqnarray*}
It is interesting to note that \(\vect\) is idempotent i.e.  \(\vect(\vect(A)) =
\vect(A)\).

The \emph{diag} operator can be used on both vectors, denoted $\Diag$, and matrices, denoted $\diag$. In the first
case, it turns the vector into a square matrix where every element is equal to
zero except for the diagonal, which is occupied by the vector. In the second
case, it turns a square matrix into a vector containing the elements in its
diagonal.  If \(\vec a\) is a vector of size \(n\) and $S$ is matrix $n\times n$
we express the above-defined operations as:
\begin{eqnarray*}
    % https://en.wikipedia.org/wiki/Diagonal_matrix
    \Diag(\vec a) &=& \vec a \vec 1' \hadam I, \\
    % https://math.stackexchange.com/a/3122442
    \diag(S) &=& (S \hadam I) \vec 1.
\end{eqnarray*}

If \(B\) has the same size as $A$ and the Hadamard product is vectorized, then
it can be removed from an expression \(\vect(A \hadam B) = \Diag(\vect(A))
\vect(B).\) For vectors, we can do a similar thing: let \(\vec x\) and \(\vec
y\) have the same size, then \(\vec x \hadam \vec y = \Diag(\vec x) \vec y.\)

An interesting connection between the Hadamard product and matrix multiplication
is the following: if \(\vec a\) has size \(m,\) then \(A \hadam \vec 1 \vec a' =
A \Diag(\vec a).\)

The trace of a square matrix is the sum of its diagonal elements. We define it
as \[\trace(S) = \vec 1' \diag(S) \vec 1.\]

This function is useful due to its many properties that make many index-free
manipulation easier. In particular we have that \(\trace(A (B \hadam C)) =
\trace((A \hadam B')C)\) and that it is circulant shift invariant \(\trace(ABC)
= \trace(CAB) = \trace(BCA).\) Please note that you cannot generally commute
matrix products inside the trace function, you can just \emph{shift} them.

The trace is a linear function, that is, if \(Q\) has the same size as \(S\)
then \(\trace(S + Q) = \trace(S) + \trace(Q)\) and if \(a\) is any scalar then
\(\trace(a \cdot S) = a \trace(S).\) We have similar results for the transpose
and element-wise functions. For example, if \(B\) has the same size as \(A\)
then \((A + B)' = A' + B'\) and \((A \hadam B)' = A' \hadam B'.\)


% https://scicomp.stackexchange.com/a/23781

The Frobenius inner product is just a generalization of the familiar dot product
between two vectors. If \(B\) has a compatible dimension with \(A\), then we
denote it as \(\langle A,B\rangle\) and $\langle A,B \rangle = \vect(A)'
\vect(B).$ 

A useful equality regarding the trace is: \(\langle A,
B \rangle = \trace(A'B) = \vec 1' (A \hadam B) \vec 1.\)

The Frobenius norm is just a generalization of the familiar Euclidean norm. We
denote it as \(\Vert A\Vert\) and \[\Vert A \Vert = \sqrt{\langle A, A \rangle}.\]

If \(B\) has the same size as \(A\) and their columns represent a batch, we can
express neatly the sum of the mean squared errors of the batch as \(\mathnormal{mse}
= \left(1/n \cdot \vec 1' (A - B)^{{\cdot}2}\right) \vec 1.\) We can note that
\[
    \mathnormal{mse}
    = \frac 1 n \vec 1' (A - B)^{{\cdot}2} \vec 1
    = \frac 1 n \vect(A - B)' \vect(A - B)
    = \frac 1 n \Vert A - B\Vert^2.
\]

\subsection{Back-propagation for Scalars}

% https://www.youtube.com/watch?v=mICbKwwHziI
% https://www.astrophys-neunhof.de/mtlg/se77211.pdf

The goal of this work is to remove the magic from the libraries used for
deep learning, in particular how the back-propagation algorithm is implemented
in the general case. Understanding this algorithm is crucial for implementing
one such library.

To do so we start by explaining the trivial case of the back-propagation for
scalars. This will allow us to get some familiarity with the general mechanism
involved and to learn about the differences with other algorithmic
differentiation processes.

Finally, we will see what are the reasons why is necessary to generalize
back-propagation and what problems arise in generalizing these algorithms to other
kind of \emph{objects} like vectors and matrices.

Back-propagation is a special case of automatic differentiation in reverse
accumulation mode. In particular, it is the case in which the function that we
are differentiating has only one (scalar) output. For neural network training
this scalar output is the \emph{loss}.

\subsubsection{What back-propagation is not}

As in~\cite{baydin2017} it is useful to start by giving some non-examples of
back-propagation to avoid any possible confusion with, related, but different
algorithm for derivatives calculation.

Depending on how you count, there are two to three main classes of automatic
method to calculate the gradient: numerical differentiation, symbolic
differentiation and automatic differentiation.

Numerical differentiation is done basically by applying the limit definition of
the derivative.  For example, let us consider a function \(\vec f : \mathcal R^2 \to \mathcal R.\)
If  we want to calculate \(\der\vec f(a,b) =
\left[\begin{array}{cc} \partialfrac{\vec f(a,b)}a &
\partialfrac{\vec f(a,b)}b\end{array}\right]\) we can approximate it by doing:
\begin{eqnarray*}
    \partialfrac{\vec f(a,b)}a
    &\approx& \frac{\vec f(a + h,b) - \vec f(a,b)}h \\
    \partialfrac{\vec f(a,b)}b
    &\approx& \frac{\vec  f(a,b + h) - \vec f(a,b)}h
\end{eqnarray*}
where \(h>0\) is a small number. This method is really easy to implement, but it
has two great problems that cannot be solved. The first is that it has numerical
accuracy issues.  the second one is that it requires \(O(n)\) evaluations of
\(f\) to calculate the gradient, where \(n\) is the number of arguments of the
function, in this case 2.

Symbolic differentiation is usually done when manipulating mathematical
expressions as a data structure (and not necessarily caring about their
evaluation.) This method can solve the numerical stability issues of the
previous one, but if implemented in a naive way can lead to a problem known as
\emph{expression swell}. This basically amounts to an exponential increase in
the size of the formula (due to careless application of differentiation rules)
and a lot of duplicated work for calculating again and again parts of the
formula that are identical but put in different places. To give a trivial
example, let \(f(a) = e^a g(a)\), now let us take its derivative \(\der f(a) =
e^a g(x) + e^a \der g(x).\) As you can see, the plain application of the
derivative rule for multiplication made us increase the size of the formula
unnecessarily and made us calculate twice the value of \(e^a\).  A more
efficient formulation would have been \(\der f(a) = e^x (g(x) + \der g(x)).\)

A better implementation of symbolical differentiation (i.e. one that involves
common sub-expression elimination) is functionally equivalent to automatic
differentiation, even with respect to control flow, as shown by S\"oren
Laue~\cite{laue2019}.

Automatic differentiation~\cite{baydin2017} has two main modes known as forward
and reverse. For the needs of this work, and most machine learning, only the
reverse mode is used.  In particular, in the field of neural networks, the
reverse-mode automatic differentiation algorithm for the function of a scalar
value (i.e. the \emph{loss}) has been independently rediscovered and given the name back-propagation~\cite{rumelhart1986}.

\subsubsection{An illustrative example}\label{sec:backprop-example}

Now, we are ready to see what the back-propagation algorithm is using an example
from~\cite{baydin2017}. Let us consider \(y = f(a,b) = \ln(a) + a b - \sin(b)\).  We
compute \(y = f(2,5)\) and we are interested in its gradient, i.e.
\(\partialfrac y a\) and \(\partialfrac y b.\) Let us start by evaluating the
function one elementary operation at a time in the following way:

\begin{minipage}{.4\textwidth}
\begin{eqnarray*}
a = x^{[-1]} &=& 2 \\
b = x^{[0]}  &=& 5 \\
    x^{[1]}  &=& \ln\left(x^{[-1]}\right) = \ln(2) \\
    x^{[2]}  &=& x^{[-1]} \times x^{[0]} = 2 \times 5 \\
             &\cdots& \\
\end{eqnarray*}
\end{minipage}%
\begin{minipage}{.55\textwidth}
\begin{eqnarray*}
             &\cdots&\\
    x^{[3]}  &=& \sin\left(x^{[0]}\right) = \sin(5) \\
    x^{[4]}  &=& x^{[1]} + x^{[2]} \approx 0.693 + 10 \\
y = x^{[5]}  &=& x^{[4]} + x^{[3]} \approx 10.693 + 0.959 \\
&\phantom-&\\
\end{eqnarray*}
\end{minipage}

We know from basic calculus that: \[
    \partialfrac y b = \partialfrac{x^{[5]}}{x^{[0]}}
    = \overbrace{
        \partialfrac{x^{[5]}}{x^{[4]}} \partialfrac{x^{[4]}}{x^{[2]}}
    }^{\partialfrac{x^{[5]}}{x^{[2]}}} \partialfrac{x^{[2]}}{x^{[0]}}
    + \partialfrac{x^{[5]}}{x^{[3]}} \partialfrac{x^{[3]}}{x^{[0]}},
\] and similarly for the other derivative we have that: \[
    \partialfrac y a = \partialfrac{x^{[5]}}{x^{[-1]}}
    = \overbrace{
        \partialfrac{x^{[5]}}{x^{[4]}} \partialfrac{x^{[4]}}{x^{[1]}}
    }^{\partialfrac{x^{[5]}}{x^{[1]}}} \partialfrac{x^{[1]}}{x^{[-1]}}
    + \overbrace{\partialfrac{x^{[5]}}{x^{[4]}} \partialfrac{x^{[4]}}{x^{[2]}}
    }^{\partialfrac{x^{[5]}}{x^{[2]}}} \partialfrac{x^{[2]}}{x^{[-1]}}.
\]

If we assume \(\bar x^{[i]} = \partialfrac y {x^{[i]}} =
\partialfrac{x^{[5]}}{x^{[i]}},\) that is the sensitivity of the output w.r.t. a
change in that variable, we can incrementally calculate both \(\bar x^{[-1]}\)
and \(\bar x^{[0]}\) in the following way:

\begingroup
\renewcommand{\partialfrac}[2]{\textstyle\frac{\partial#1}{\partial#2}}
\begin{minipage}{.4\textwidth}
\begin{eqnarray*}
    \bar x^{[5]}    &=& \partialfrac{x^{[5]}}{x^{[5]}}
        = 1 \\
    \bar x^{[4]}    &=& \bar x^{[5]}\partialfrac{x^{[5]}}{x^{[4]}}
        = \bar x^{[5]} \times 1 = 1 \\
    \bar x^{[3]}    &=& \bar x^{[5]}\partialfrac{x^{[5]}}{x^{[3]}}
        = \bar x^{[5]} \times (-1) \\ %= -1 \\
    \bar x^{[1]}    &=& \bar x^{[4]}\partialfrac{x^{[4]}}{x^{[1]}}
        = \bar x^{[4]} \times 1 = 1 \\
    \bar x^{[2]}    &=& \bar x^{[4]}\partialfrac{x^{[4]}}{x^{[2]}}
        = \bar x^{[4]} \times 1 = 1 \\
                    &\cdots& \\
\end{eqnarray*}
\end{minipage}%
\hspace{.7cm}
\begin{minipage}{.5\textwidth}
\begin{eqnarray*}
                    &\cdots& \\
    \bar x^{[0]}    &=& \bar x^{[3]}\partialfrac{x^{[3]}}{x^{[0]}}
        = \bar x^{[3]} \times \cos\left(x^{[0]}\right) \\ %\approx -0.282 \\
    \bar x^{[-1]} &=& \bar x^{[2]}\partialfrac{x^{[2]}}{x_{-1}}
        = \bar x^{[2]} \times x^{[0]} = 5 \\
    \bar x^{[0]}    &\pluseq& \bar x^{[2]}\partialfrac{x^{[2]}}{x^{[0]}}
        = \bar x^{[2]} \times x_{-1} \approx 1.716 \\
    \bar x^{[-1]} &\pluseq& \bar x^{[1]}\partialfrac{x^{[1]}}{x^{[-1]}}
        = \frac{\bar x^{[1]}}{x^{[-1]}} = 5.5 \\
                  && \\
\end{eqnarray*}
\end{minipage}
\endgroup

What we saw in the example above is the scheme of the back-propagation
algorithm.  When performing the calculations for our function, we record the
dependency graph, as shown in Figure~\ref{fig:dag}, and we \emph{adjoint} \(\bar
x^{[i]}\) to each variable \(x^{[i]}\). When we have obtained our scalar output,
we topologically sort our graph and then start to propagate the gradient
backwards, starting from the scalar output and ending at our input variables. In
this way, we obtain the gradient of our \emph{loss}.

\begin{figure}
    \centering
    \includegraphics[width=0.58\textwidth]{figures.2}
    \caption{Dependency graph of \(f.\)}
    \label{fig:dag}
\end{figure}

This algorithm has no numerical stability issues, as opposed to numerical
differentiation, and has a cost that is proportional to evaluating the
function in any case, in contrast with naive symbolic differentiation.

This is also true for its generalization, reverse-mode automatic
differentiation, which is a numerically accurate and efficient algorithm for
calculating the derivatives of a function \(\vec f : \mathcal R^n \to \mathcal
R^m,\) where \(n>>m.\)

We can summarize what we have done until now using the C99-like pseudocode shown
in Figure~\ref{fig:pseudo}.  We consider three arrays: tape, val and grad. The
first one records the dependencies among the variables used during the
computation, the second one stores the value of each intermediate result for
computing the function, and the last one stores the gradient of the output w.r.t
that variable. The vJp function stands for vector-Jacobian product, which will
be described in Section~\ref{sec:derivations}. At the moment, you can think of
it as the multiplication between the gradient accumulated up to that point and
the derivative of the current function.

% https://tex.stackexchange.com/questions/94699/absolutely-definitely-preventing-page-break
\newenvironment{absolutelynopagebreak}
  {\par\nobreak\vfil\penalty0\vfilneg
   \vtop\bgroup}
  {\par\xdef\tpd{\the\prevdepth}\egroup
   \prevdepth=\tpd}

\begin{figure}[h]
\centering
\begin{minipage}{.7\textwidth}
% https://latexref.xyz/tabbing.html
\begin{tabbing}
void \= backprop(int l) \{\+\\
    toposort(tape);\\
    for (int i = 0; i \textless{} l; i++) grad[i] = 0;\\
    grad[l] = 1;\\
    /\textasteriskcentered{}
    Assuming only binary operations.
    \textasteriskcentered/\\
    for \=(int i = l; i \textgreater{} 0; i-{}-) \{\+\\
        int \=l = tape[i].left, r = tape[i].right, \\
            \>op = tape[i].op;\\
        grad[l] += vJp[op].l(grad[i], val[r], val[l]);\\
        grad[r] += vJp[op].r(grad[i], val[r], val[l]);\-\\
    \}\-\\
\}\\
\end{tabbing}
\end{minipage}
\caption{Pseudo-code for the back-propagation algorithm. }
\label{fig:pseudo}
\end{figure}

\section{Problems with generalization}\label{sec:generalization}

In this section, we discuss the problems with generalizing the back-propagation
algorithm to matrices. In particular, in Subsection~\ref{sec:matcalc} we
summarize results for matrix calculus which are difficult to find in literature
since they are non standardized yet~\cite{magnus2010}.

\begingroup
\newcommand{\bb}{\phantom{12}}
\renewcommand\matrix{\left[\begin{array}{ccc}\bb&\bb&\bb\\\bb&\bb&\bb\\\bb&\bb&\bb\end{array}\right]}
\renewcommand\vector{\left[\begin{array}{ccc}\bb&\bb&\bb\end{array}\right]}

Let us start by focusing on the vector case, in particular on a function of
vector argument and scalar value that is internally composed of functions of
vector value and argument.  If the input value is \(\vec x^{[0]}\) and the
values computed by the various functions are \(\vec x^{[1]}, \ldots, x^{[n]}\)
terminating in the scalar output of the last function, then we can write the
derivative of the output w.r.t. the input using the chain rule as:
\[\partialfrac{x^{[n]}}{\vec x^{[n-1]}} \partialfrac{\vec x^{[n-1]}}{\vec
x^{[n-2]}} \cdots \partialfrac{\vec x^{[1]}}{\vec x^{[0]}}.\] If we focus only
on the shape of the objects along the chain (i.e. the vectors and matrices), we
can visualize the chain rule in this way: \[\vector \matrix \dots \matrix\!\!.\]
Viewing it like this it is obvious that it is more convenient to perform our
calculation by multiplying the objects from left to right.
\endgroup

Even if we use the right order for the multiplications, which is the one
naturally imposed by the topological sort, we are still wasting a lot of
resources since usually the matrices \(\partialfrac{\vec x^{[j]}}{\vec
x^{[i]}}\) along the chain are very sparse and most of the memory we allocate
for them is going to be filled with zeros.

If we ignore this problem for now, implementing back-propagation for the
\emph{loss} of vector argument is not much harder than the scalar version since
taking derivatives of a function of vector argument and value (that can happen
in the middle of the neural network) is common knowledge, just a standard vector
calculus.

When we start to take into account matrices this creates the possibility of
having functions of matrix argument and value in the middle of our computation,
and taking derivatives of this kind of function is not common knowledge.

The reader might be surprised by the fact that this notion for matrix calculus
is not only uncommon, but it also has no standardized arrangement for the partial
derivatives. While the end result is always going to be some mathematical object
that contains all possible partial derivatives w.r.t. each single element in the
input and argument matrix, how they should be organized has not been settled.
This has important repercussions in how fundamental calculations (like the chain
rule~\cite{magnus2010}) have to be performed for these objects.

We may ask ourselves: can we sidestep this problem by performing this
calculation explicitly i.e. decomposing the matrix operations in their
elementary operations and performing the derivation on that? This can be a
helpful exercise for small examples, but proving something this way for the
general case rapidly becomes unwieldy. We think that anyone can agree that, as
with the vector case, developing a general symbolic method not only helps the
derivation but also contributes to understanding and new discoveries.

\iffalse
Now that we have convinced ourselves that we need some robust method to tackle
this matrix derivation task, as humble computer scientists what options do we
have available to us?
\fi

The go-to reference for this kind of thing is The Matrix
Cookbook~\cite{petersen2012}. While this is a terrific reference for many things
and contains some of the answers we need, the way in which they are presented
does not make it obvious that it is what we are looking for. So, without
additional information, it cannot help us.

Back-propagation is a specific case of the more general reverse-mode automatic
differentiation algorithm, which together with its \emph{dual} algorithm
forward-mode automatic differentiation, has a good deal of literature on it. In
particular, there is a collection of results like~\cite{giles2008}, but as the
title may suggest they are just results without any satisfactory explanation
(again, unless pretty much you already know the answers and just need the
references).  This can certainly allow you to implement the back-propagation
through the most common operations, but this leaves you unable to derive your
own results or possible improvements for special cases that you care about. Most
importantly, this does not leave you with a good understanding of what is going
on in the algorithm you are implementing (without even talking about how one may
approach debugging this kind of numerical algorithm.)

If one starts looking around for answers, one may encounter the so-called
\emph{adjoint trick} for deriving these results or an even more general method
based on tensor calculus~\cite{laue2018, laue2020}, but they usually involve
more math than a humble computer scientist can stomach.

Together with the difficulty of derivation, now the problem we have cited before
is even more accentuated.  Let \(F\) be a function of size \(m \times n\) with
argument \(X\) of size \(p \times q\).  When we define its derivative at \(X\)
we have that it contains the \(mnpq\) partial derivatives. To be more concrete,
let us consider \(m=n=p=q=1024\), and say that we store all the entries of our
matrices in the IEEE-754 single precision floating point format.  This means
that if we want to store this derivative we are going to need 4TiB of memory,
most of which is still going to be just zeros.

We therefore not only need to to perform the gradient accumulation in the right
order but also find a way to multiply the gradient \emph{vector} with the sparse
Jacobian in such a way that we do not need to instantiate it fully.

\subsection{Matrix Calculus}\label{sec:matcalc}

This section is based on the work of Magnus, Neudecker and Abadir, see
e.g.~\cite{neudecker1969,magnus1985, magnus2005, magnus2007, magnus2010,
magnus2019}.  Also~\cite{liu2022} gives a good perspective on the field of
matrix differential calculus and its development in general.  Matrix calculus is
the less-known, bigger cousin, of vector calculus. So, let us start with a quick
review of it.  We start with an introduction to an elegant matrix
calculus that is going to greatly simplify our treatment of the subject.

Now let \(\vec f\) be a differentiable function, and let \(\vec y = \vec f(\vec
x)\) for some \(\vec x\). We denote the derivative (or Jacobian) of \(\vec f\) at \(\vec x\) with: \[
    \der\vec f(\vec x)
    = \partialfrac{\vec y}{\vec x'}
    = \left[\begin{array}{ccc}
        \partialfrac{\vec y}{\vec x_1} & \cdots & \partialfrac{\vec y}{\vec x_n}
    \end{array}\right]
    = \left[\begin{array}{ccc}
        \partialfrac{\vec y_1}{\vec x_1} & \cdots & \partialfrac{\vec y_1}{\vec x_m} \\
        \vdots & \ddots & \vdots \\
        \partialfrac{\vec y_n}{\vec x_1} & \cdots & \partialfrac{\vec y_n}{\vec x_m}
    \end{array}\right]\!\!.
\] 

We now want to emphasise a few facts. We explicitly use the numerator layout.
Numerator layout means that the Jacobian has an element \(\vec y_i\) in each column and a \(\vec x_i\) in each row. We say it explicitly
because we will write \(\partialfrac{\vec y}{\vec x'}\), and we want to emphasise
that \(\vec x\) is laid out as a row and \(\vec y\) as a column. In the
special case where our differentiable function is \(y = f(x)\) we have
that: \[
    \der f(\vec x) = \partialfrac y {\vec x'} = \left[\begin{array}{ccc}
        \partialfrac y {\vec x_1} & \cdots & \partialfrac y {\vec x_n}
    \end{array}\right]
\] is a row vector, that is also known as the gradient of \(f\) at \(\vec x\),
written as \(\nabla f(x).\)

There is no conventional definition for matrix-valued functions of matrix
argument. The one we use in this work follows Magnus and
Naudecker~\cite{magnus2019}. We adopt their definition because it is the most
useful for our use case, and it would be useful if it becomes the standard for
machine learning-related derivation.

To make our lives easier we approach matrix calculus using
differentials rather than derivatives.

% https://en.wikipedia.org/wiki/Differential_of_a_function
In the scalar case, we have that:
\[\lim_{h \to 0} \frac{f(x + h) - f(x)}h = \der f(x)\] which can be rewritten as
\(f(x + h) = f(x) + (\der f(x))h + r(h,x)\), where \(\lim_{h \to 0} r(h,x) = 0\)
and it can be thought of as a reminder or error term.

We denote the differential of \(f\) at \(x\) with increment \(h\) as
\(\dif f(x;h) = (\der f(x))h\), which is usually written as \((\der f(x))\dif
x.\)

We can do similar reasoning for vector functions and get: \[\vec f(\vec x + \vec
h) = \vec f(\vec x) + (\der\vec f(\vec x))\vec h + \vec r(\vec h,\vec x)\] where
\(\dif \vec f(\vec x;\vec h) = (\der \vec f(\vec x))\vec h,\) and it is usually
written as \((\der\vec f(\vec x))\dif\vec x.\)

% FIXME: here I use U instead of H...
Again, with similar reasoning, we can generalize our definition to the matrix
case using the \(\vect\) function as such: \[\vect(F(X+U)) = \vect(F(X)) +
(\der\vect(F(X))\vect(U) + \vect(R(U,X)).\]

Therefore now if \(Y=F(X)\) we have that: \[\der F(X) =
\partialfrac{\vect(Y)}{\vect(X)'}\] i.e.  the derivative (or Jacobian) of \(F\)
at \(X.\)

We now need a way to find the derivative from an expression involving
differentials and to perform the chain rule, but with differentials. To this end
we can exploit identification results and Cauchy's rule of invariance.

The identification rules are: \[\dif\vec f(\vec x) = A(\vec x) \dif\vec x \iff
\der \vec f(\vec x) = A(\vec x)\] and \[\dif \vect(F(X)) = A(X) \dif\vect(X)
\iff \der F(X) = A(X).\] These two rules mean that if we can manipulate an
expression involving differentials so that it has the form on the right side of
the equal sign in the left part of the logical equivalence, then we can
\emph{harvest} the derivative. These two rules, together with their special
cases, are reported in Table~\ref{tab:identifications}.

\begin{table}
\centering
\begin{tabular}{cccc}
 & \(x\) & \(\vec x\) & \(X\) \\[2pt]
         \cline{2-4}
\multicolumn{1}{r|}{\(\dif f =\)}
    & \(a(x) \dif x\) & \(\vec a(\vec x)' \dif\vec x\) & \(\vect(A(X))' \dif\vect(X)\) \\
\multicolumn{1}{r|}{\(\dif\vec f =\)}
    & \(\vec a(x) \dif x\) & \(A(\vec x) \dif\vec x\) & \(A(X) \dif\vect(X)\) \\
\multicolumn{1}{r|}{\(\dif F =\)}
    & \(\vect(A(x)) \dif x\) & \(A(\vec x) \dif\vec x\) & \(A(X) \dif\vect(X)\)
\end{tabular}
\caption{Identification theorems~\cite{kinghorn1996}.}
\label{tab:identifications}
\end{table}

Cauchy's rule of invariance states that \(\dif C(X; H) = \dif B(A(X); \dif
A(X;H))\). If you squint hard enough, you notice that it is the same as the
chain rule for differential: \[\der C(X) = \der B(A(X)) \der A(X).\]

We are now ready to state some useful properties of these differentials. Let
\(A\) be an \(n \times m\) matrix, $a$ a scalar, and let both be constant.
Let then \(X\) and \(Y\) be matrices of compatible sizes and \(S\) a
square matrix, and let them all be variables. Then:

\begin{minipage}{.45\textwidth}
\begin{eqnarray*}
    \dif A            &=& \leftidx{_n}{\vec 0}{}\,\leftidx{_m}{\vec 0}{'}, \\
    \dif (a \cdot X)  &=& a \cdot \dif X, \\
    \dif (X + Y)      &=& \dif X + \dif Y, \\
    \dif (x \cdot X)  &=& (\dif x) \cdot X + x \cdot \dif X, \\
    \dif (XY)         &=& (\dif X)Y + X \dif Y, \\
    \dif (X \krone Y) &=& (\dif X) \krone Y + X \krone \dif Y, \\
\end{eqnarray*}
\end{minipage}%
\begin{minipage}{.5\textwidth}
\begin{eqnarray*}
    \dif (X \hadam Y) &=& (\dif X) \hadam Y + X \hadam \dif Y, \\
    \dif X^{{\cdot}a} &=& a \cdot X^{{\cdot}a-1} \hadam \dif X, \\
    \dif a^{{\cdot}X} &=& \ln(a) \cdot a^{{\cdot}X} \hadam \dif X \\
    \dif X'          &=& (\dif X)', \\
    \dif\vect(X)      &=& \vect(\dif X), \\
    \dif\trace(S)     &=& \trace(\dif S). \\
\end{eqnarray*}
\end{minipage}

The rules we have stated are in general form and can appropriately be
specialized for scalar and vector cases.

\iffalse
\subsubsection{Example}
After all this, we are finally ready for an example! Consider the softmax
function for vectors \(\vec s(\vec x)\), we now proceed to find the value of
\(\der\vec s(\vec x)\). First, let \(\vec y = e^{{\cdot}\vec x}\) for brevity,
then we have that
\begin{eqnarray}
\dif\vec s(\vec x)
&=& \dif\left((\vec 1' \vec y)^{-1} \cdot \vec y\right)\nonumber\\
&=& \dif\left((\vec 1' \vec y)^{-1}\right) \cdot \vec y
    + (\vec 1' \vec y)^{-1} \cdot \dif\vec y\nonumber\\
&=& (\vec 1' \vec y)^{-1} \cdot \dif\vec y
    - (\vec 1' \vec y)^{-2} \cdot \dif(\vec 1' \vec y) \cdot \vec y\nonumber\\
&=& (\vec 1' \vec y)^{-1} \cdot \dif\vec y
    - (\vec 1' \vec y)^{-2} \cdot (\vec 1' \dif\vec y) \cdot \vec y\nonumber\\
&=& (\vec 1' \vec y)^{-1} \cdot \dif\vec y
    - (\vec 1' \vec y)^{-2} \cdot \vec y \cdot (\vec 1' \dif\vec y)\nonumber\\
&=& (\vec 1' \vec y)^{-1} \cdot \dif\vec y
    - (\vec 1' \vec y)^{-2} \cdot \vec y \vec 1' \dif\vec y\label{eq:switcheroo}\\
&=& \left((\vec 1' \vec y)^{-1} \cdot I
    - (\vec 1' \vec y)^{-2} \cdot \vec y \vec 1'\right) \dif\vec y\nonumber\\
&=& \left((\vec 1' e^{{\cdot}\vec x})^{-1} \cdot I
    - (\vec 1' e^{{\cdot}\vec x})^{-2} \cdot e^{{\cdot}\vec x} \vec 1'\right)
    \dif e^{{\cdot}\vec x}\nonumber\\
&=& \left((\vec 1' e^{{\cdot}\vec x})^{-1} \cdot I
    - (\vec 1' e^{{\cdot}\vec x})^{-2} \cdot e^{{\cdot}\vec x} \vec 1'\right)
    e^{{\cdot}\vec x} \hadam \dif \vec x\nonumber\\
&=& \left((\vec 1' e^{{\cdot}\vec x})^{-1} \cdot I
    - (\vec 1' e^{{\cdot}\vec x})^{-2} \cdot e^{{\cdot}\vec x} \vec 1'\right)
    \Diag(e^{{\cdot}\vec x}) \dif \vec x\nonumber\\
&=& \left((\vec 1' e^{{\cdot}\vec x})^{-1} \cdot \Diag(e^{{\cdot}\vec x})
    - (\vec 1' e^{{\cdot}\vec x})^{-2} \cdot e^{{\cdot}\vec x} \vec 1' \Diag(e^{{\cdot}\vec x})\right)
    \dif \vec x\nonumber\\
&=& \left((\vec 1' e^{{\cdot}\vec x})^{-1} \cdot \Diag(e^{{\cdot}\vec x})
    - (\vec 1' e^{{\cdot}\vec x})^{-2} \cdot e^{{\cdot}\vec x} (e^{{\cdot}\vec x})'\right)
    \dif \vec x\nonumber\\
&=& (\Diag(\vec s(\vec x)) - \vec s(\vec x) \vec s(\vec x)') \dif\vec x\nonumber
\end{eqnarray}
Now, thanks to the identification theorem, we have found our derivative.  In the
equation~\ref{eq:switcheroo} a step has been performed that requires a bit of
justification, i.e. the fact that \(\vec y \cdot (\vec 1' \dif\vec y) = \vec y
\vec 1' \dif\vec y.\) This is simply explained if we think about the fact that
\(n \cdot \vec y = \vec y \vec n',\) where \(n\) is a vector of size 1, and how
matrix multiplication is defined.

It is interesting to note that this result can be rewritten like this
\[\Diag(\vec s(\vec x)) - \vec s(\vec x)^{{\cdot}2} \vec 1'.\] 
We will see in
Section~\ref{sec:broadcasting} that formulations involving outer products with
\(\vec 1\) can usually be implemented efficiently with broadcasting.

We have found our first derivative, of a non-trivial function, using
differentials and the identification theorem. The process was a bit long to make
sure that things were as clear as possible, but with a bit of experience, the
process can be streamlined by skipping some (or a lot) of steps.
\fi

\section{Back-propagation for Matrices}\label{sec:contribution}

Performing the back-propagation algorithm for scalars incurs too much overhead
when we need to implement functions involving matrices. In this section, we
explain how to generalize the previously introduced back-propagation algorithm
for scalars to functions of matrix value and argument. In particular, we show
how to propagate gradient through the following operations: matrix
multiplication, element-wise functions, broadcasting, dropout, and attention.

\subsection{Derivations}\label{sec:derivations}

We have now almost all the instruments we need to derive the back-propagation
formulas for matrices.

Let us revamp the previous example, now using matrices. So let \(A\) and
\(B\) be two square matrices and \(y = f(A,B) = \Vert\!\ln(A) + A B +
\sin(B)\Vert^2.\) Again say we are interested in \(\partialfrac y A\) and
\(\partialfrac y B.\) We can again evaluate the function one elementary
operation at the time

\begin{minipage}{.45\textwidth}
\begin{eqnarray*}
A = X^{[-1]} &=& 2 \\
B = X^{[0]}  &=& 5 \\
    X^{[1]}  &=& \ln\left(X^{[-1]}\right) \\
    X^{[2]}  &=& X^{[-1]} \times X^{[0]} \\
    &\cdots&  \\
\end{eqnarray*}
\end{minipage}%
\begin{minipage}{.5\textwidth}
\begin{eqnarray*}
            &\cdots&\\
    X^{[3]}  &=& \sin\left(X^{[0]}\right) \\
    X^{[4]}  &=& X^{[1]} + X^{[2]} \\
    X^{[5]}  &=& X^{[4]} + X^{[3]} \\
y = x^{[6]}  &=& \Vert X^{[5]}\Vert^2. \\
\end{eqnarray*}
\end{minipage}

Thanks to how we defined our matrix calculus, we have that:
\begin{eqnarray*}
    \partialfrac y {\vect(B)'}
    &=& \partialfrac{x^{[6]}}{\vect\left(X^{[0]}\right)'} \\
    &=& \underbrace{
        \overbrace{
            \partialfrac{x^{[6]}}{X^{[5]}} \partialfrac{X^{[5]}}{X^{[4]}}
        }^{\partialfrac{x^{[6]}}{\vect\left(X^{[4]}\right)'}}
        \partialfrac{X^{[4]}}{X^{[2]}}
    }_{\partialfrac{x^{[6]}}{\vect\left(X^{[2]}\right)'}}
    \partialfrac{X^{[2]}}{X^{[0]}}
    + \overbrace{
        \partialfrac{x^{[6]}}{X^{[5]}}
        \partialfrac{X^{[5]}}{X^{[3]}}
    }^{\partialfrac{x^{[6]}}{\vect\left(X^{[3]}\right)'}
    } \partialfrac{X^{[3]}}{X^{[0]}}
\end{eqnarray*}
\begin{eqnarray*}
    \partialfrac y {\vect(A)'}
    &=& \partialfrac{x^{[6]}}{\vect\left(X^{[-1]}\right)'} \\
    &=& \underbrace{
        \overbrace{
            \partialfrac{x^{[6]}}{X^{[5]}} \partialfrac{X^{[5]}}{X^{[4]}}
        }^{\partialfrac{x^{[6]}}{\vect\left(X^{[4]}\right)'}}
        \partialfrac{X^{[4]}}{X^{[2]}}
    }_{\partialfrac{x^{[6]}}{\vect\left(X^{[2]}\right)'}} \partialfrac{X^{[2]}}{X^{[{-1}]}}
    + \underbrace{
        \overbrace{
            \partialfrac{x^{[6]}}{X^{[5]}} \partialfrac{X^{[5]}}{X^{[4]}}
        }^{\partialfrac{x^{[6]}}{\vect\left(X^{[4]}\right)'}}
        \partialfrac{X^{[4]}}{X^{[1]}}
    }_{\partialfrac{x^{[6]}}{\vect\left(X^{[1]}\right)'}}
    \partialfrac{X^{[1]}}{X^{[-1]}}
\end{eqnarray*}
where some \(\vect\) and transpose are omitted for simplicity.

Now let us proceed with the back-propagation algorithm step by step. Let
\(\vect\left(\bar X^{[i]}\right)' = \partialfrac y {\vect\left(X^{[i]}\right)'}
= \partialfrac{x^{[6]}}{\vect\left(X^{[i]}\right)'}\) then:
\begingroup
\renewcommand{\partialfrac}[2]{\textstyle\frac{\partial#1}{\partial#2}}
\let\oldvect\vect
\renewcommand\vect[1]{\oldvect\mathchoice{\!}{}{}{}\left(#1\right)}
\setlength\arraycolsep{1pt}
\begin{minipage}{.45\textwidth}
\begin{eqnarray*}
\bar x^{[6]}          &=& \partialfrac{x^{[6]}}{x^{[6]}} = 1 \\
\left(\bar{\vec x}^{[5]}\right)' &=& \bar x^{[6]} \partialfrac{x^{[6]}}{\vect{X^{[5]}}'} \\
\vect{\bar X^{[4]}}'  &=& \left(\bar{\vec x}^{[5]}\right)' \partialfrac{\vect{X^{[5]}}}{\vect{X^{[4]}}'} \\
\vect{\bar X^{[3]}}'  &=& \left(\bar{\vec x}^{[5]}\right)' \partialfrac{\vect{X^{[5]}}}{\vect{X^{[3]}}'} \\
\vect{\bar X^{[1]}}'  &=& \vect{\bar X^{[4]}}' \partialfrac{\vect{X^{[4]}}}{\vect{X^{[1]}}'} \\
&\cdots& \\
\end{eqnarray*}
\end{minipage}%
\begin{minipage}{.5\textwidth}
\begin{eqnarray*}
&\cdots& \\
\vect{\bar X^{[2]}}' &=& \vect{\bar X^{[4]}}' \partialfrac{\vect{X^{[4]}}}{\vect{X^{[2]}}'} \\
\vect{\bar X^{[0]}}' &=& \vect{\bar X^{[3]}}' \partialfrac{\vect{X^{[3]}}}{\vect{X^{[0]}}'} \\
\vect{\bar X^{[-1]}}' &=& \vect{\bar X^{[2]}}' \partialfrac{\vect{X^{[2]}}}{\vect{X^{[-1]}}'} \\
\vect{\bar X^{[0]}}'  &\pluseq& \vect{\bar X^{[2]}}' \partialfrac{\vect{X^{[2]}}}{\vect{X^{[0]}}'} \\
\vect{\bar X^{[-1]}}' &\pluseq& \vect{\bar X^{[1]}}' \partialfrac{\vect{X^{[1]}}}{\vect{X^{[-1]}}'} \\
\end{eqnarray*}
\end{minipage}
\endgroup

In general, what we are doing is always, while calculating the derivative of  a
scalar value function \(X^{[j]} = F\!\left(X^{[i]}\right),\) where \(j > i,\)
when performing back-propagation we have: \[\vect\!\left(\bar X^{[i]}\right)' =
\vect\!\left(\bar X^{[j]}\right)' \partialfrac{X^{[j]}}{X^{[i]}} =
\partialfrac{x^{[n]}}{X^{[j]}} \partialfrac{X^{[j]}}{X^{[i]}} =
\partialfrac{x^{[n]}}{X^{[i]}}.\]

This is equivalent to doing the derivative as:
\[\vect\!\left(\bar X^{[i]}\right)'
= \mathrm{vJp}_F\!\left(X^{[i]},X^{[j]}\right)
= \partialfrac{\left\langle\bar X^{[j]},X^{[j]}\right\rangle}{\vect\!\left(X^{[i]}\right)'}
= \partialfrac{\left\langle\bar X^{[j]},F\left(X^{[i]}\right)\right\rangle}{\vect\left(X{[i]}\right)'}.\]
As we will in the next paragraphs,
this formulation involving the vector-Jacobian product is much more useful.

\subsubsection{Matrix Multiplication}

Now, let us focus on the matrix multiplication operation. Since matrix multiplication is not
commutative, we have to study two cases, the one when we multiply a constant
matrix from the left and the one when we multiply it from the right. Let us drop
the indices from \(X\) and let us call \(\bar X = G\). The two vJp we care about
are \(\partialfrac{\langle G, X B \rangle}{\vect(X)'}\) and
\(\partialfrac{\langle G, A X \rangle}{\vect(X)'}.\)

We can proceed, using the differential calculus we have developed and then
applying the appropriate identification theorem, to derive both vJp:

\begin{minipage}{.45\textwidth}
\begin{eqnarray*}
\dif\langle G, X B \rangle
&=& \dif\trace(G' (X B)) \\
&=& \trace(G' (\dif X) B) \\
&=& \trace(B G' \dif X) \\
&=& \trace((G B')' \dif X)) \\
&=& \vect(G B')' \vect(\dif X) \\
\end{eqnarray*}
\end{minipage}%
\begin{minipage}{.5\textwidth}
\begin{eqnarray*}
\dif\langle G,A X\rangle
&=& \dif\trace(G' A X) \\
&=& \trace(G' A \dif X) \\
&=& \trace((A' G)' \dif X) \\
&=& \vect(A' G)' \vect(\dif X). \\
&& \\
\end{eqnarray*}
\end{minipage}

Let us note that both expressions are functions of two \emph{small} dense matrices.
In this way we completely avoid the need to instantiate the sparse Jacobian of.

\subsubsection{Element-wise functions}

One of the most used family of functions in neural networks is the family of
element-wise functions, i.e. functions that are applied to each element of a
matrix. Many activation functions belong to this family, like ReLU and Sigmoid.
It is therefore essential to learn how to back-propagate the gradient through
this family of functions.

First, let us note that for an element-wise function \(F\) we have its
differential is \(\dif F(X) = \tilde F(X) \hadam \dif X,\) where \(\tilde F(X)\)
is the derivative of the function applied to each element. For example, if the
function applied to each element is \(\sin\), then the derivative \(\cos\) will be applied to each element. The reason is quickly explained. If we think about what
the differential does to each element of the matrix, that is, if we refer to the scalar case, we have: \(\dif f(x) = \der f(x) \dif x.\)

With the same setup as before, we now have:
\begin{eqnarray*}
\dif\langle G, F(X) \rangle
&=& \dif\trace(G' F(X)) \\
&=& \trace(\dif(G' F(X))) \\
&=& \trace(\dif G' F(X) + G' \dif F(X)) \\
&=& \trace(G' \dif F(X)) \\
&=& \trace(G' (\tilde F(X) \hadam \dif X)) \\
&=& \vect(G)' \vect(\tilde F(X) \hadam \dif X) \\
&=& \vect(G)' \diag(\vect(\tilde F(X))) \vect(\dif X) \\
&=& \vect(G \hadam \tilde F(X))' \vect(\dif X).
\end{eqnarray*}
Using the usual identification theorem, we get that the vJp for element-wise
functions is \(\vect(G \hadam \tilde F(X))'.\)

\subsubsection{Broadcasting}\label{sec:broadcasting}

% http://coldattic.info/post/116/+
% https://numpy.org/doc/stable/user/basics.broadcasting.html#:~:text=outer%20product+
% https://lesia.obspm.fr/perso/thibaut-paumard/yorick-doc/manual/yorick_50.html+
% https://stackoverflow.com/a/26950256+

Broadcasting is a convenient and efficient way to implement outer products. It
first appeared in Yorick~\cite{munro1995} and was then made popular by
NumPy~\cite{harris2020}.

When performing training on batches it is common in popular deep learning frameworks to write operations with expressions like \(S(AX + \vec
b)\) or \(S(AX + b)\). These two expressions have no well-defined meaning in the standard mathematical language, but in these frameworks it usually means adding \(\vec b\) to every row of \(AX\), for the first case, and adding \(b\) to every element of \(AX\) in the second. This behaviour has the name of broadcasting and it is
done to improve performance and reduce memory usage at the same time (since both
operations can be done without instantiating the matrices explicitly),
therefore, it is worth implementing.

This raises a question: how do we back-propagate through this kind of operation?
Luckily, the answer is easy. We can mathematically fix these expressions by
adding a few multiplications with constants that keep the result identical and
allow us to use the formalism that we have developed until now.  We can
therefore rewrite the two expressions in the following way \(S(AX + \vec b)
\Rightarrow S(AX + \vec b \vec 1'),\) and \(S(AX + b) \Rightarrow S(AX + b \cdot
\vec 1 \vec 1').\)

We now derive the back-propagation formula for the broadcasting with a scalar. To simplify our calculations, let  \(F(b) = b \cdot \vec 1 \vec 1'\). Then:

\begin{minipage}{.45\textwidth}
\begin{eqnarray*}
    \dif F(b)
    &=& \dif(b \cdot \vec 1 \vec 1') \\
    &=& \dif(b \cdot \vec 1) \vec 1' + b \vec 1 \dif \vec 1' \\
    &=& \dif(b \cdot \vec 1) \vec 1' \\
    &=& (\dif(b) \cdot \vec 1 + b \dif \vec 1) \vec 1' \\
    &=& \dif b \cdot \vec 1  \vec 1' \\
    &=& \vec 1 \vec 1' \cdot \dif b. \\
    && \\
    && \\
\end{eqnarray*}
\end{minipage}%
\begin{minipage}{.5\textwidth}
\begin{eqnarray*}
    \dif\langle G,F(b)\rangle
    &=& \dif\trace(G' F(b)) \\
    &=& \trace(\dif(G' F(b))) \\
    &=& \trace(\dif G' F(b) + G' \dif F(b)) \\
    &=& \trace(G' \dif F(b))) \\
    &=& \trace(G' \vec 1 \vec 1' \cdot \dif b)) \\
    &=& \trace(G' \vec 1 \vec 1') \cdot \dif b \\
    &=& \trace(\vec 1' G' \vec 1) \cdot \dif b \\
    &=& \vec 1' G' \vec 1 \cdot \dif b. \\
\end{eqnarray*}
\end{minipage}

That is, we just have to sum all the elements of \(G\). When we broadcast \(b\)
the result is similar. In general, we only need to sum the contributions along
the broadcast dimensions.

\subsubsection{Dropout}

The dropout layer~\cite{srivastava2014} is commonly used for regularization
techniques to avoid overfitting. It is usually described as \emph{randomly drop units (along with their connections) from the neural network during training}, and it
is usually implemented as a Hadamard multiplication with a random binary vector
\(m\) that acts as a mask for the dropped units.

Therefore for the vector case we can write \(\vec x \hadam \vec m,\) while for
the matrix case we can either promote \(\vec m\) to a random binary matrix \(M\)
and write \(X \hadam M\) or rely on broadcasting and write \(X \hadam \vec m
\vec 1'.\)

In both cases is easy to derive expressions to back-propagate through this operation: \(\dif\trace(G' (M \hadam X)) = \trace(G' (M \hadam \dif X)) =
\trace((G' \hadam M') \dif X) = \trace((G \hadam M)' \dif X)\).

\subsubsection{Attention}

The attention is the core operation behind the transformer architecture~\cite{vaswani2017}. It operates on a \emph{text} represented as a
sequence of vector embeddings in a matrix, say \(X\). It has three parameters to
 \emph{learn}, \(W^{[q]},\) \(W^{[k]},\) and \(W^{[v]}.\)

% https://en.wikipedia.org/wiki/Attention_(machine_learning)#Core_calculations
Let \(Q = X W^{[q]},\) \(K = X W^{[k]},\) \(V = X W^{[v]}\). We can define the
attention as \(V' S(1 / \sqrt{d} \cdot K Q'),\) where \(d\) is
the number of rows in \(X.\)

If we break it down into its elementary operations, we obtain:

\begin{minipage}{.45\textwidth}
\begin{eqnarray*}
    X^{[1]} &= & (X W^{[q]})' \\
    X^{[2]} &= & X W^{[k]} \\
    X^{[3]} &= & X^{[2]} X^{[1]} \\
    X^{[4]} &= & 1/\sqrt d \cdot X^{[3]} \\
    &\cdots& \\
\end{eqnarray*}
\end{minipage}%
\begin{minipage}{.5\textwidth}
\begin{eqnarray*}
    &\cdots& \\
    X^{[5]} &= & S(X^{[4]}) \\
    X^{[6]} &= & (X W^{[v]})' \\
    X^{[7]} &= & X^{[6]} X^{[5]}. \\
    && \\
\end{eqnarray*}
\end{minipage}

In order to backpropagate through it, we need to know how to propagate through
scalar multiplication, transposed matrix multiplication and the \emph{softmax}
function. 
We show how to derive the latter (the other two are easy exercises).

Let us consider \(A = \vec 1 \vec 1' e^{{\cdot}X}\). Then:
\begin{eqnarray*}
\dif\langle G,S(X)\rangle
&=& \trace\!\left(G' \dif \left(e^{{\cdot}X} \hadam A^{{\cdot}-1}\right)\right)\\
&=& \trace\!\left(G' \left(
    \dif e^{{\cdot}X} \hadam A^{{\cdot}-1}
    + e^{{\cdot}X} \hadam \dif A^{{\cdot}-1}\right)\right)\\
&=& \trace\!\left(G' \left(
    e^{{\cdot}X} \hadam \dif X \hadam A^{{\cdot}-1}
    + e^{{\cdot}X} \hadam -A^{{\cdot}-2} \hadam \dif A\right)\right)\\
&=& \trace\!\left(G' \left(e^{{\cdot}X} \hadam A^{{\cdot}-1}
    \hadam \left(\dif X - A^{{\cdot}-1} \hadam \dif A\right)\right)\right)\\
&=& \trace\!\left(G'
    \left(S(X) \hadam \left(\dif X - A^{{\cdot}-1} \hadam \dif A\right)\right)\right)\\
&=& \trace\!\left((G' \hadam S(X)')
    \left(\dif X - A^{{\cdot}-1} \hadam \dif A\right)\right)\\
&=& \trace((G \hadam S(X))' \dif X)
    - \trace\!\left((G \hadam S(X))' \left(A^{{\cdot}-1} \hadam \dif A\right)\right)
\end{eqnarray*}
Let us now consider \(B = G \hadam S(X),\) so that \(\dif\langle G,S(X)\rangle = \trace(B' \dif X) - \trace\left(B' \left(A^{{\cdot}-1} \hadam \dif A\right)\right)\). Then
\begin{eqnarray*}
\dif\langle G,S(X)\rangle
&=& \trace(B' \dif X) - \trace\!\left(\left(B' \hadam \left(A^{{\cdot}-1}\right)'\right) \dif A\right)\\
&=& \trace(B' \dif X) - \trace\!\left(
    \left(B \hadam A^{{\cdot}-1}\right)' \vec 1 \vec 1'
    \left(e^{{\cdot}X} \hadam \dif X\right)
    \right)\\
&=& \trace(B' \dif X)
    - \trace\!\left(\left(\vec 1 \vec 1' \left(B \hadam A^{{\cdot}-1}\right)\right)'
    \left(e^{{\cdot}X} \hadam \dif X\right)\right)\\
&=& \trace(B' \dif X)
    - \trace\!\left(\left(\left(\vec 1 \vec 1' \left(B \hadam A^{{\cdot}-1}\right)\right)' \hadam \left(e^{{\cdot}X}\right)'\right)
    \dif X\right)\\
&=& \trace(B' \dif X)
    - \trace\!\left(\left(\vec 1 \vec 1' \left(B \hadam A^{{\cdot}-1}\right) \hadam e^{{\cdot}X}\right)'
    \dif X\right)\\
&=& \trace\!\left(
    \left(
    B' - \left(\vec 1 \vec 1' \left(B \hadam A^{{\cdot}-1}\right) \hadam e^{{\cdot}X}\right)'
    \right)\dif X\right)\\
&=& \trace\!\left(
    \left(
    B - e^{{\cdot}X} \hadam \vec 1 \vec 1' \left(B \hadam A^{{\cdot}-1}\right)
    \right)' \dif X\right)\\
&=& \trace\!\left(
    \left(
    B - e^{{\cdot}X} \Diag\left(\left(\vec 1' \left(B \hadam A^{{\cdot}-1}\right)\right)'\right)
    \right)' \dif X\right)\!.
\end{eqnarray*}

Now, thanks to the identification theorem, we get the formula that we need to
propagate the gradient through the \emph{softmax} function.

\section{Discussion and conclusions}\label{sec:conclusion}

The purpose of this work is to present the back-propagation algorithm in matrix
form, providing the necessary tools to derive gradient propagation formulas.

We have shown how to derive all the necessary results (i.e., matrix
multiplication and element-based functions) to implement a back-propagation
library, along with some auxiliary results (broadcasting and dropout) that are
commonly used in deep neural networks.  Finally, we also showed how one can
derive a vJp for non-elementary operation \emph{softmax}.  This way, we have
shown that with the right mathematical tools available, it is relatively easy to
obtain these results.

As a result, the derived vJps can be used to produce custom-optimized
implementation on either GPU or CPU, or written as-is in some high-level
language (even if some care needs to be taken to minimize overflows and
underflow issues in softmax.) In either case, one can easily implement a
back-propagation library using these derivations as a starting point.

This is the first self-contained resource that allows any computer scientist,
with a standard mathematical background, to derive back-propagation formulas for
his or her operations. Until now, this was a major obstacle and anyone who had
to implement such code needed either the help of a mathematician or laboriously
and painfully to learn on its own.  With this work, we believe we have overcome
this difficulty due to the organization and vastness of the literature, allowing
anyone to obtain their own results more easily.

Possible future work includes making derivations even easier by using direct
sums and block matrices to represent operations involving batch matrix
multiplications, for example multi-head attention.

Finally, since in deep learning it is common to use non-differentiable functions
such as ReLU or max-pool, taking advantage of the approach used in this work,
obtaining these functions will not be a problem since it is possible to
generalize the concept of gradient to subgradient.

\begingroup
% https://texfaq.org/FAQ-compactbib
\iffalse
\let\oldbibliography\thebibliography
\renewcommand{\thebibliography}[1]{%
  \oldbibliography{#1}%
  \setlength{\itemsep}{0pt}%
}
\fi
\bibliography{bibliography-article}
\endgroup

\end{document}